\documentclass[11pt, a4paper]{article}
\usepackage[paper=a4paper, left=1.5cm, right=1.5cm, bottom=1.5cm, top=1.5cm]{geometry}
%
\usepackage[utf8]{inputenc}
\usepackage[spanish]{babel}

\usepackage{../caratuladc/caratula/caratula}

\usepackage{multicol}
\usepackage{enumitem}
\usepackage{amsmath}
\usepackage{amssymb}
\usepackage{xcolor}

% Comandos %

\newcommand{\sii}{\leftrightarrow}
\newcommand{\hacer}{\textcolor{red}{HACER!}}
\usepackage[spanish,activeacute,es-tabla]{babel}
\usepackage[utf8]{inputenc}
\usepackage{ifthen}
\usepackage{listings}
\usepackage{dsfont}
\usepackage{subcaption}
\usepackage{amsmath}
\usepackage[top=1cm,bottom=2cm,left=1cm,right=1cm]{geometry}%
\usepackage{color}%
\usepackage{changepage}
\newcommand{\tocarEspacios}{%
	\addtolength{\leftskip}{3em}%
	\setlength{\parindent}{0em}%
}

\newcommand{\Indent}{\hspace*{0.75cm}}

% Especificacion de procs

\newcommand{\In}{\textsf{in }}
\newcommand{\Out}{\textsf{out }}
\newcommand{\Inout}{\textsf{inout }}

\newcommand{\encabezadoDeProc}[3]{%
% Ponemos la palabrita problema en tt
%  \noindent%
{\normalfont\bfseries\ttfamily proc}%
% Ponemos el nombre del problema
\ %
{\normalfont\ttfamily #1}%
\
% Ponemos los parametros
(#2)%
\ifblank{#3}{}{%
	% Por ultimo, va el tipo del resultado
	\ : #3}
}

\newenvironment{proc}[4][res]{%

% El parametro 1 (opcional) es el nombre del resultado
% El parametro 2 es el nombre del problema
% El parametro 3 son los parametros
% El parametro 4 es el tipo del resultado
% Preambulo del ambiente problema
% Tenemos que definir los comandos requiere, asegura, modifica y aux
\newcommand{\requiere}[2][]{%
{\normalfont\bfseries\ttfamily requiere\;}%
\ifthenelse{\equal{##1}{variaslineas}}{\{%
\begin{adjustwidth}{+5em}{}
	\ensuremath{##2}
\end{adjustwidth}
\}
{\normalfont\bfseries\,\par}%
}
{%
\{\ensuremath{##2}\}%
{\normalfont\bfseries\,\par}%
}
}

\newcommand{\asegura}[2][]{%
{\normalfont\bfseries\ttfamily asegura\;}%
\ifthenelse{\equal{##1}{variaslineas}}{\{%
\begin{adjustwidth}{+5em}{}
	\ensuremath{##2}
\end{adjustwidth}
\}
{\normalfont\bfseries\,\par}%
}
{%
\{\ensuremath{##2}\}%
{\normalfont\bfseries\,\par}%
}
}
\renewcommand{\aux}[4]{%
	{\normalfont\bfseries\ttfamily aux\ }%
		{\normalfont\ttfamily ##1}%
	\ifthenelse{\equal{##2}{}}{}{\ (##2)}\ : ##3\, = \ensuremath{##4}%
	{\normalfont\bfseries\,;\par}%
}
\renewcommand{\pred}[3]{%
{\normalfont\bfseries\ttfamily pred }%
	{\normalfont\ttfamily ##1}%
\ifthenelse{\equal{##2}{}}{}{\ (##2) }%
\{%
\begin{adjustwidth}{+5em}{}
	\ensuremath{##3}
\end{adjustwidth}
\}%
{\normalfont\bfseries\,\par}%
}

\newcommand{\res}{#1}
\vspace{1ex}
\noindent
\encabezadoDeProc{#2}{#3}{#4}
% Abrimos la llave
\par%
\tocarEspacios
}
{
% Cerramos la llave
\vspace{1ex}
}

\newcommand{\aux}[4]{%
	{\normalfont\bfseries\ttfamily\noindent aux\ }%
		{\normalfont\ttfamily #1}%
	\ifthenelse{\equal{#2}{}}{}{\ (#2)}\ : #3\, = \ensuremath{#4}%
	{\normalfont\bfseries\,;\par}%
}

\newcommand{\pred}[3]{%
{\normalfont\bfseries\ttfamily\noindent pred }%
	{\normalfont\ttfamily #1}%
\ifthenelse{\equal{#2}{}}{}{\ (#2) }%
\{%
\begin{adjustwidth}{+2em}{}
	\ensuremath{#3}
\end{adjustwidth}
\}%
{\normalfont\bfseries\,\par}%
}

% Tipos

\newcommand{\nat}{\ensuremath{\mathbb{N}}}
\newcommand{\reals}{\ensuremath{\mathbb{R}}}
\newcommand{\ent}{\ensuremath{\mathds{Z}}}
\newcommand{\float}{\ensuremath{\mathds{R}}}
\newcommand{\bool}{\ensuremath{\mathsf{Bool}}}
\newcommand{\cha}{\ensuremath{\mathsf{Char}}}
\newcommand{\str}{\ensuremath{\mathsf{String}}}
\newcommand{\dict}[1]{\ensuremath{\mathsf{dict}\lrangle{#1}}}
\newcommand{\conj}[1]{\ensuremath{\mathsf{conj}\lrangle{#1}}}
\newcommand{\tupla}[1]{\ensuremath{\mathsf{tupla}\lrangle{#1}}}
\newcommand{\struct}[1]{\ensuremath{\mathsf{struct}\lrangle{#1}}}

% Logica

\newcommand{\True}{\ensuremath{\mathrm{true}}}
\newcommand{\False}{\ensuremath{\mathrm{false}}}
\newcommand{\Then}{\ensuremath{\rightarrow}}
\newcommand{\Iff}{\ensuremath{\leftrightarrow}}
\newcommand{\implica}{\ensuremath{\longrightarrow}}
\newcommand{\IfThenElse}[3]{\ensuremath{\mathsf{if}\ #1\ \mathsf{then}\ #2\ \mathsf{else}\ #3\ \mathsf{fi}}}
\newcommand{\If}[3]{\text{\normalfont\ttfamily IfThenElse}(\ensuremath{#1,\;#2,\;#3})}
\newcommand{\yLuego}{\land _L}
\newcommand{\oLuego}{\lor _L}
\newcommand{\thenLuego}{\Then _L}

\newcommand{\cuantificador}[5]{%
	\ensuremath{(#2 #3: #4)\ (%
		\ifthenelse{\equal{#1}{multLineas}}{
			$ % exiting math mode
				\begin{adjustwidth}{+2em}{}
					$#5$%
				\end{adjustwidth}%
			$ % entering math mode
		}{
			#5
		}
		)}
}

\newcommand{\existe}[4][]{%
	\cuantificador{#1}{\exists}{#2}{#3}{#4}
}
\newcommand{\paraTodo}[4][]{%
	\cuantificador{#1}{\forall}{#2}{#3}{#4}
}

\newcommand{\Def}[1]{\text{\normalfont\ttfamily def}(#1)}

%listas

\newcommand{\TLista}[1]{\ensuremath{seq \langle #1\rangle}}
\newcommand{\lvacia}{\ensuremath{\langle\,\rangle}}
\newcommand{\lv}{\ensuremath{\langle\,\rangle}}
\newcommand{\longitud}[1]{\ensuremath{|#1|}}
\newcommand{\cons}[1]{\ensuremath{\mathsf{addFirst}}(#1)}
\newcommand{\indice}[1]{\ensuremath{\mathsf{indice}}(#1)}
\newcommand{\conc}[1]{\ensuremath{\mathsf{concat}}(#1)}
\newcommand{\head}[1]{\ensuremath{\mathsf{head}}(#1)}
\newcommand{\tail}[1]{\ensuremath{\mathsf{tail}}(#1)}
\newcommand{\sub}[1]{\ensuremath{\mathsf{subseq}}(#1)}
\newcommand{\en}[1]{\ensuremath{\mathsf{en}}(#1)}
\newcommand{\cuenta}[2]{\mathsf{cuenta}\ensuremath{(#1, #2)}}
\newcommand{\suma}[1]{\mathsf{suma}(#1)}
\newcommand{\twodots}{\ensuremath{\mathrm{..}}}
\newcommand{\masmas}{\ensuremath{+\!+\;}}
\newcommand{\matriz}[1]{\TLista{\TLista{#1}}}
\newcommand{\seqchar}{\TLista{\cha}}
\newcommand{\lrangle}[1]{\ensuremath{\langle\text{#1}\rangle}}

%dict

\newcommand{\setKey}[1]{\ensuremath{\mathsf{setKey}(#1)}}
\newcommand{\delKey}[1]{\ensuremath{\mathsf{delKey}(#1)}}

\renewcommand{\wp}[2]{
	\ensuremath{\textit{wp}(\textbf{\lstinline{#1}}, #2)}
}

\newcommand{\hoare}[3]{\ensuremath{\{#1\} \; #2 \; \{#3\}}}

\renewcommand{\lstlistingname}{Código}
\lstset{% general command to set parameter(s)
	language=Java,
	morekeywords={func, endif, endwhile, skip, end, var, then},
	basewidth={0.47em,0.40em},
	columns=fixed, fontadjust, resetmargins, xrightmargin=5pt, xleftmargin=15pt,
	flexiblecolumns=false, tabsize=4, breaklines, breakatwhitespace=false, extendedchars=true,
	numbers=left, numberstyle=\tiny, stepnumber=1, numbersep=9pt,
	frame=l, framesep=3pt,
	captionpos=b,
}

% TADs

\newenvironment{tad}[2]{
\newcommand{\tadtype}{%
	\ensuremath{#1 \ifblank{#2}{}{\lrangle{#2}}}
}

\renewcommand{\pred}[3]{%
{\normalfont\bfseries\ttfamily pred }%
	{\normalfont\ttfamily ##1}%
\ifthenelse{\equal{##2}{}}{}{\ (##2) }%
\{%
\begin{adjustwidth}{+5em}{}
	\ensuremath{##3}
\end{adjustwidth}
\}
{\normalfont\bfseries\,\par}%
}

\vspace{1ex}
\noindent
{\normalfont\bfseries\ttfamily\large TAD #1\ifthenelse{\equal{#2}{}}{}{$<$#2$>$}} \{
% Abrimos la llave
\par%
\tocarEspacios
}{
\vspace{1ex} \par\}
}

\newcommand{\obs}[2]{
	obs #1: \ensuremath{#2}\par
}

\newenvironment{module}[4]{
\newcommand{\tadtype}{%
	\ensuremath{#3 \ifblank{#4}{}{\lrangle{#4}}}
}

\newcommand{\moduletype}{%
	\ensuremath{#1 \ifblank{#2}{}{\lrangle{#2}}}
}


\renewcommand{\pred}[3]{%
{\normalfont\bfseries\ttfamily pred }%
	{\normalfont\ttfamily ##1}%
\ifthenelse{\equal{##2}{}}{}{\ (##2) }%
\{%
\begin{adjustwidth}{+5em}{}
	\ensuremath{##3}
\end{adjustwidth}
\}
{\normalfont\bfseries\,\par}%
}

\vspace{1ex}
\noindent
{\normalfont\bfseries\ttfamily\large Módulo #1\ifthenelse{\equal{#2}{}}{}{$<$#2$>$} \ifblank{#3}{}{implementa #3\ifthenelse{\equal{#4}{}}{}{$<$#4$>$}}} \{
% Abrimos la llave
\par%
\tocarEspacios
}{
\vspace{1ex} \par\}
}

\newcommand{\var}[2]{
	var #1: \;\texto{#2}\par
}
\newcommand{\compl}[1]{Complejidad: $#1$}

% Tipos de datos de implementación
\newcommand{\Int}{\ensuremath{\mathsf{int}}}
\newcommand{\Real}{\ensuremath{\mathsf{real}}}
\newcommand{\Bool}{\ensuremath{\mathsf{bool}}}
\newcommand{\Char}{\ensuremath{\mathsf{char}}}
\newcommand{\String}{\ensuremath{\mathsf{string}}}
\newcommand{\Array}[1]{\ensuremath{\mathsf{Array<#1>}}}

\newcommand{\texto}[1]{\text{\normalfont\bfseries\ttfamily #1\;}}

\begin{document}

\titulo{Guia 3}
\materia{Algoritmos y Estructuras de Datos I}
\fecha{2do cuatrimestre 2024}

\integrante{Federico Barberón}{112/24}{jfedericobarberonj@gmail.com}
%Carátula
\maketitle
\newpage

%Indice
\tableofcontents
\newpage

\section{Guia 3}

\subsection{Ejercicio 1}
Calcular las siguientes expresiones, donde $a$, $b$ son variables reales, $i$ una variable entera y $A$ es una secuencia de reales.

\begin{multicols}{2}
    \begin{enumerate}[label=\alph*)]
        \item $\Def{a + 1} \equiv True$
        \item $\Def{a / b} \equiv b \neq 0$
        \item $\Def{\sqrt{a / b}} \equiv b \neq 0 \yLuego a / b \geq 0$
        \item $\Def{A[i] + 1} \equiv 0 \leq i < |A|$
        \item $\Def{A[i + 2]} \equiv 0 \leq i + 2 < |A| \equiv -2 \leq i < |A| - 2$
        \item $\Def{0 \leq i \leq |A| \yLuego A[i] \geq 0} \equiv i \neq |A|$
    \end{enumerate}
\end{multicols}

\subsection{Ejercicio 2}
Calcular las siguientes precondiciones más débiles, donde $a$, $b$ son variables realies, $i$ una variable entera y $A$ es una secuencia de reales.

\begin{enumerate}[label=\alph*)]
    \item \begin{align*}
              \wp{a := a+1; b := a/2}{b \geq 0} & \equiv \wp{a := a+1}{\wp{b := a/2}{b \geq 0}}      \\
                                                & \equiv \wp{a := a+1}{\Def{a/2} \yLuego a/2 \geq 0} \\
                                                & \equiv \wp{a := a+1}{a \geq 0}                     \\
                                                & \equiv \Def{a+1} \yLuego a+1 \geq 0                \\
                                                & \equiv \boxed{a \geq -1}
          \end{align*}

    \item \begin{align*}
              \wp{a := A[i] + 1; b := a*a}{b \neq 2} & \equiv \wp{a := A[i] + 1}{\wp{b := a*a}{b \neq 2}}              \\
                                                     & \equiv \wp{a := A[i] + 1}{a*a \neq 2}                           \\
                                                     & \equiv \wp{a := A[i] + 1}{|a| \neq \sqrt{2}}                    \\
                                                     & \equiv \Def{A[i] + 1} \yLuego |A[i] + 1| \neq \sqrt{2}          \\
                                                     & \equiv \boxed{0 \leq i < |A| \yLuego A[i] \neq -1 \pm \sqrt{2}}
          \end{align*}

    \item \begin{align*}
              \wp{a := A[i] + 1; a := b*b}{a \geq 0} & \equiv \wp{a := A[i] + 1}{\wp{a := b*b}{a \geq 0}} \\
                                                     & \equiv \wp{a := A[i] + 1}{b*b \geq 0}              \\
                                                     & \equiv \wp{a := A[i] + 1}{|b| \geq 0}              \\
                                                     & \equiv \wp{a := A[i] + 1}{True}                    \\
                                                     & \equiv \boxed{0 \leq i < |A|}
          \end{align*}

    \item \begin{align*}
              \wp{a := a-b; b := a+b}{a \geq 0 \land b \geq 0} & \equiv \wp{a := a-b}{\wp{b := a+b}{a \geq 0 \land b \geq 0}} \\
                                                               & \equiv \wp{a := a-b}{a \geq 0 \land a+b \geq 0}              \\
                                                               & \equiv a-b \geq 0 \land (a-b)+b \geq 0                       \\
                                                               & \equiv \boxed{a \geq b \land a \geq 0}
          \end{align*}
\end{enumerate}

\subsection{Ejercicio 3}
Sea $Q \equiv \paraTodo{j}{\ent}{0 \leq j < |A| \thenLuego A[j] \geq 0}$. Calcular las siguientes precondiciones más débiles, donde $i$ es una variable entera y $A$ es una secuencia de enteros.

\begin{enumerate}[label=\alph*)]
    \item \begin{align*}
              \wp{A[i] := 0}{Q} & \equiv \wp{A := setAt(A, i, 0)}{Q}                                                                                                    \\
                                & \equiv 0 \leq i < |A| \yLuego Q_{setAt(A, i, 0)}^A                                                                                    \\
                                & \equiv 0 \leq i < |A| \yLuego \paraTodo{j}{\ent}{0 \leq j < |A| \thenLuego setAt(A, i, 0)[j] \geq 0}                                  \\
                                & \equiv 0 \leq i < |A| \yLuego \paraTodo{j}{\ent}{0 \leq j < |A| \thenLuego (i = j \Then 0 \geq 0) \land (i \neq j \Then A[i] \geq 0)} \\
                                & \equiv \boxed{0 \leq i < |A| \yLuego \paraTodo{j}{\ent}{(0 \leq j < |A| \land i \neq j) \thenLuego A[j] \geq 0}}
          \end{align*}

    \item \begin{align*}
              \wp{A[i+2] := 0}{Q} & \equiv 0 \leq i+2 < |A| \yLuego Q_{setAt(A, i+2, 0)}^Q                                                                            \\
                                  & \equiv \ldots \yLuego \paraTodo{j}{\ent}{0 \leq j < |A| \thenLuego (i+2 = j \Then 0 \geq 0) \land (i+2 \neq j \Then A[j] \geq 0)} \\
                                  & \equiv \boxed{-2 \leq i < |A| - 2 \thenLuego \paraTodo{j}{\ent}{(0 \leq j < |A| \land i+2 \neq j) \thenLuego A[j] \geq 0}}
          \end{align*}

    \item \hacer

    \item \begin{align*}
              \wp{A[i] := 2 * A[i]}{Q} & \equiv \wp{A := setAt(A, i, 2 * A[i])}{Q}                                                                                            \\
                                       & \equiv 0 \leq i < |A| \yLuego Q_{setAt(A, i, 2 * A[i])}^A                                                                            \\
                                       & \equiv \ldots \yLuego \paraTodo{j}{\ent}{0 \leq j < |A| \thenLuego (i = j \Then 2 * A[i] \geq 0) \land (i \neq j \Then A[j] \geq 0)} \\
                                       & \equiv \ldots \yLuego \paraTodo{j}{\ent}{0 \leq j < |A| \thenLuego (i = j \Then A[j] \geq 0) \land (i \neq j \Then A[j] \geq 0)}     \\
                                       & \equiv \boxed{0 \leq i < |A| \yLuego \paraTodo{j}{\ent}{0 \leq j < |A| \thenLuego A[j] \geq 0}}
          \end{align*}

    \item \hacer
\end{enumerate}

\subsection{Ejercicio 4}
Para los siguientes pares de programas $S$ y postcondiciones $Q$

\begin{itemize}
    \item Escribir la precondición más débil $P = \wp{S}{Q}$
    \item Mostrar formalmente que la $P$ elegida es correcta
\end{itemize}

\begin{enumerate}[label=\alph*)]
    \item $S \equiv$

          \begin{lstlisting}
        if (a < 0)
            b := a
        else
            b := -a
        endif
    \end{lstlisting}

          $Q \equiv (b = - |a|)$

          \begin{align*}
              \wp{S}{Q} & \equiv (a < 0 \land \wp{b := a}{Q}) \lor (a \geq 0 \land \wp{b := -a}{Q}) \\
                        & \equiv (a < 0 \land a = - |a|) \lor (a \geq 0 \land -a = - |a|)           \\
                        & \equiv a < 0 \lor a \geq 0                                                \\
                        & \equiv True
          \end{align*}

    \item \hacer
    \item \hacer

    \item $S \equiv$

          \begin{lstlisting}
        if (i > 1)
            s[i] := s[i-1]
        else
            s[i] := 0
        endif
    \end{lstlisting}

          $Q \equiv \paraTodo{j}{\ent}{1 \leq j < |s| \thenLuego s[j] = s[j - 1]}$

          \[
              \wp{S}{Q} \equiv (i > 1 \land \wp{s[i] := s[i-1]}{Q}) \lor (i \leq 1 \land \wp{s[i] := 0}{Q})
          \]

          $
              \begin{aligned}
                  \wp{s[i] := s[i - 1]}{Q} & \equiv (0 \leq i < |s| \land 0 \leq i - 1 < |s|) \yLuego Q_{setAt(s, i, s[i-1])}^s              \\
                                           & \equiv 1 \leq i < |s| \yLuego                                                                   \\
                                           & \paraTodo{j}{\ent}{1 \leq j < |s| \thenLuego setAt(s, i, s[i-1])[j] = setAt(s, i, s[i-1])[j-1]} \\\\
                                           & \equiv 1 \leq i < |s| \yLuego                                                                   \\
                                           & \paraTodo{j}{\ent}{1 \leq j < |s| \thenLuego (i = j \Then s[j-1] = s[j-1]) \land                \\
                                           & (i = j - 1 \Then s[j] = s[j-2]) \land ((i \neq j \land i \neq j - 1) \Then s[j] = s [j-1])}     \\\\
                                           & \equiv 1 \leq i < |s| \yLuego                                                                   \\
                                           & \paraTodo{j}{\ent}{1 \leq j < |s| \yLuego (i = j - 1 \Then s[j] = s[j-2]) \land                 \\
                                           & ((i \neq j \land i \neq j - 1) \Then s[j] = s[j-1])}
              \end{aligned}
          $

          $
              \begin{aligned}
                  \wp{s[i] := 0}{Q} & \equiv 0 \leq i < |s| \yLuego Q_{setAt(s, i, 0)}^s                                                                  \\
                                    & \equiv 0 \leq i < |s| \yLuego \paraTodo{j}{\ent}{1 \leq j < |s| \thenLuego setAt(s, i, 0)[j] = setAt(s, i, 0)[j-1]} \\
                                    & \equiv 0 \leq i < |s| \yLuego \paraTodo{j}{\ent}{1 \leq j < |s| \thenLuego                                          \\
                                    & (i = j \Then 0 = s[j-1]) \land (i = j-1 \Then s[j] = 0) \land ((i \neq j \land i \neq j-1) \Then s[j] = s[j-1])}
              \end{aligned}
          $

          \begin{align*}
              \wp{S}{Q} & \equiv (1 < i < |s| \yLuego \paraTodo{j}{\ent}{1 \leq j < |s| \yLuego (i = j - 1 \Then s[j] = s[j-2]) \land        \\
                        & ((i \neq j \land i \neq j - 1) \Then s[j] = s[j-1])})\lor                                                          \\
                        & (i \leq 1 \land 0 \leq i < |s| \yLuego \paraTodo{j}{\ent}{1 \leq j < |s| \thenLuego (i = j \Then 0 = s[j-1]) \land \\
                        & (i = j-1 \Then s[j] = 0) \land ((i \neq j \land i \neq j-1) \Then s[j] = s[j-1])})
          \end{align*}

    \item \hacer
    \item \hacer
\end{enumerate}

\subsection{Ejercicio 5}
Para las siguientes especificaciones:

\begin{itemize}
    \item Poner nombre al problema que resuelven
    \item Escribir un programa $S$ sencillo en SmallLang, sin ciclos, que lo resuelva
    \item Dar la precondición más débil del programa escrito con respecto a la postcondición de su especificación
\end{itemize}

\begin{enumerate}[label=\alph*)]
    \item \begin{proc}{sumaIesimoElemento}{\In s: \TLista{\ent}, \In i: \ent, \Inout a: \ent}{}
              \requiere{0 \leq i < |s| \yLuego a = \sum_{j=0}^{i-1} s[j]}
              \asegura{a = \sum_{j=0}^{i} s[j]}
          \end{proc}

          \begin{lstlisting}
            func sumaIesimoElemento(s <int>, i int, a int) {
                a := a + s[i];
                return a
            }
          \end{lstlisting}

          \begin{align*}
              \wp{a := a + s[i]}{a = \sum_{j=0}^{i} s[j]} & \equiv 0 \leq i < |s| \yLuego a + s[i] = \sum_{j=0}^{i} s[j]    \\
                                                          & \equiv \boxed{0 \leq i < |s| \yLuego a = \sum_{j=0}^{i-1} s[j]}
          \end{align*}

    \item \hacer
    \item \hacer
\end{enumerate}

\subsection{Ejercicio 7}
Dado el siguiente condicional determinar la precondición más débil que permite hacer valer la poscondición ($Q$) propuesta. Se pide:

\begin{itemize}
    \item Describir en palabras la WP esperada
    \item Derivarla formalmente a partir de los axiomas de precondión mas débil. Para obtener el puntaje máximo deberá simplificarla lo más posible.
\end{itemize}

\begin{enumerate}[label=\alph*)]
    \item $Q \equiv \{\paraTodo{j}{\ent}{0 \leq j < |s| \thenLuego s[j] > 0}\}$

          $S \equiv$
          \begin{lstlisting}
if s[i] < 0 then
    s[i]  := -s[i]
else
    s[i] := 0
endif
    \end{lstlisting}

          Se espera que $i$ este en el rango de $s$, que $s[i] < 0$ y que todos los elementos en posisicones distintas de $i$ sean mayores a 0.

          \[\wp{S}{Q} \equiv 0 \leq i < |s| \yLuego (s[i] < 0 \land \wp{s[i] := -s[i]}{Q}) \lor (s[i] \geq 0 \land \wp{s[i] := 0}{Q})\]

          \begin{itemize}
              \item $
                        \wp{s[i] := -s[i]}{Q} \equiv 0 \leq i < |s| \yLuego \paraTodo{j}{\ent}{0 \leq j < |s| \thenLuego setAt(s,i,-s[i])[j] > 0} \\
                        \equiv 0 \leq i < |s| \yLuego \paraTodo{j}{\ent}{(0 \leq j < |s| \land j \neq i) \thenLuego s[j] > 0} \land -s[i] > 0
                    $

              \item $
                        \wp{s[i] := 0}{Q} \equiv 0 \leq i < |s| \yLuego \paraTodo{j}{\ent}{0 \leq j < |s| \thenLuego setAt(s,i,0) > 0} \\
                        \equiv 0 \leq i < |s| \yLuego \paraTodo{j}{\ent}{(0 \leq j < |s| \land j \neq i) \thenLuego s[j] > 0} \land 0 > 0 \\
                        \equiv False
                    $
          \end{itemize}

          De esta manera, juntado todo queda:

          \begin{align*}
              \wp{S}{Q} & \equiv 0 \leq i < |s| \yLuego (s[i] < 0 \land \paraTodo{j}{\ent}{(0 \leq j < |s| \land j \neq i) \thenLuego s[j] > 0} \land -s[i] > 0) \\
                        & \equiv \boxed{0 \leq i < |s| \yLuego (s[i] < 0 \land \paraTodo{j}{\ent}{(0 \leq j < |s| \land j \neq i) \thenLuego s[j] > 0})}
          \end{align*}

    \item \hacer
    \item \hacer
    \item $Q \equiv \paraTodo{j}{\ent}{0 \leq j < |s| \thenLuego s[j] \mod 3 = 0}$

          $S \equiv$
          \begin{lstlisting}
if i mod 3 == 0 then
    s[i] = s[i] + 6;
else
    s[i] = i;
endif
    \end{lstlisting}

          Se espera que $i$ este en el rango de $s$, que $i$ sea divisible por 3 y que todos los elementos de $s$ sean divisibles por 3.

          \[
              \wp{S}{Q} \equiv (i \mod 3 = 0 \land \wp{s[i] = s[i] + 6}{Q}) \lor (i \mod 3 \neq 0 \land \wp{s[i] = i}{Q})
          \]

          \begin{itemize}
              \item $
                        \wp{s[i] = s[i] + 6}{Q} \\
                        \equiv 0 \leq i < |s| \yLuego \paraTodo{j}{\ent}{0 \leq j < |s| \thenLuego (i = j \Then (s[j] + 6) \mod 3 = 0) \land (i \neq j \Then s[j] \mod 3 = 0)} \\
                        \equiv 0 \leq i < |s| \yLuego \paraTodo{j}{\ent}{0 \leq j < |s| \thenLuego s[j] \mod 3 = 0}
                    $

              \item $
                        \wp{s[i] = i}{Q} \\
                        \equiv 0 \leq i < |s| \yLuego \paraTodo{j}{\ent}{0 \leq j < |s| \thenLuego (i = j \Then j \mod 3 = 0) \land (i \neq j \Then s[j] \mod 3 = 0)} \\
                        \equiv 0 \leq i < |s| \yLuego \paraTodo{j}{\ent}{(0 \leq j < |s| \land i \neq j) \thenLuego s[j] \mod 3 = 0} \land i \mod 3 = 0
                    $
          \end{itemize}

          La segunda expresión de la disyunción de la WP es \textit{False} pues queda $(i \mod 3 \neq 0 \land \ldots \land i \mod 3 = 0)$ lo cual es una contradicción.

          Por lo que la WP queda de la siguiente manera:
          %
          \[\wp{S}{Q} \equiv i \mod 3 = 0 \land 0 \leq i < |s| \yLuego \paraTodo{j}{\ent}{0 \leq j < |s| \thenLuego s[j] \mod 3 = 0}\]

          Tal y como se esperaba.
    \item \hacer
\end{enumerate}

\end{document}

\documentclass[11pt, a4paper]{article}
\usepackage[paper=a4paper, left=1.5cm, right=1.5cm, bottom=1.5cm, top=1.5cm]{geometry}
%
\usepackage[utf8]{inputenc}
\usepackage[spanish]{babel}

\usepackage{../caratuladc/caratula/caratula}

\usepackage{multicol}
\usepackage{enumitem}
\usepackage{amsmath}
\usepackage{amssymb}
\usepackage{xcolor}

% Comandos %

\newcommand{\sii}{\leftrightarrow}
\newcommand{\hacer}{\textcolor{red}{HACER!}}

\begin{document}

\titulo{Guia 1}
\materia{Algoritmos y Estructuras de Datos I}
\fecha{2do cuatrimestre 2024}

\integrante{Federico Barberón}{112/24}{jfedericobarberonj@gmail.com}
%Carátula
\maketitle
\newpage

%Indice
\tableofcontents
\newpage

\section{Guia 1}

\subsection{Ejercicio 1}
Determinar los valores de verdad de las siguientes proposiciones cuando el valor de verdad de $a$, $b$ y $c$ es \textit{verdadero} y el de $x$ e $y$ es \textit{falso}.

\begin{multicols}{2}
    \begin{enumerate}[label=\alph*)]
        \item $(\neg x \lor b)$ True
        \item $((c \lor (y \land a)) \lor b)$ True
        \item $\neg (c \lor y)$ False
        \item $\neg (y \lor c)$ False
        \item $(\neg(c \lor y) \sii (\neg c \land \neg y))$ True
        \item $((c \lor y) \land (a \lor b))$ True
        \item $(((c \lor y) \land (a \lor b)) \sii (c \lor (y \land a) \lor b))$ True
        \item $(\neg c \land \neg y)$ False
    \end{enumerate}
\end{multicols}

\subsection{Ejercicio 2}
Considere la siguiente oración: "Si es mi cumpleaños o hay torta, entonces hay torta".

\begin{itemize}
    \item Escribir usando lógica proposicional y realizar la tabla de verdad

          $(p \lor q) \to q$

          \begin{tabular}{|c|c|c|c|}
              \hline
              $p$ & $q$ & $(p \lor q)$ & $(p \lor q) \to q$ \\
              \hline
              T   & T   & T            & T                  \\
              \hline
              T   & F   & T            & F                  \\
              \hline
              F   & T   & T            & T                  \\
              \hline
              F   & F   & F            & T                  \\
              \hline
          \end{tabular}

    \item Asumiendo que la oración es verdadera y hay una torta, qué se puede concluir?

          Se concluye que puede o no ser su cumpleaños.

    \item Asumiendo que la oración es verdadera y no hay una torta, qué se puede concluir?

          Se concluye que NO es su cumpleaños.

    \item Suponiendo que la oración es mentira (es falsa), se puede concluir algo?

          Se concluye que es su cumpleaños pero no hay torta :(
\end{itemize}

\subsection{Ejercicio 3}
Usando reglas de equivalencia (conmutatividad, asociatividad, De Morgan, etc) determinar si los siguientes pares de fórmulas son equivalencias. Indicar en cada paso qué regla se utilizó.

\begin{enumerate}[label=\alph*)]
    \item \begin{itemize}
              \item $(p \lor q) \land (p \lor r)$
              \item $\neg p \to (q \land r)$
          \end{itemize}

          \[
              \begin{array}{rl|l}
                  (p \lor q) \land (p \lor r) \sii & p \lor (q \land r)             & \text{Distributiva}           \\
                  \sii                             & \neg (\neg p) \lor (q \land r) & \text{Doble negación}         \\
                  \sii                             & \neg p \to (q \land r)         & \text{Definición condicional}
              \end{array}
          \]

          Las fórmulas son equivalentes.

    \item \begin{itemize}
              \item $\neg (\neg p) \to (\neg (\neg p \land \neg q))$
              \item $q$

          \end{itemize}

          \[
              \begin{array}{rl|l}
                  \neg (\neg p) \to (\neg (\neg p \land \neg q)) \sii & \neg (\neg p) \to (p \lor q) & \text{De Morgan}                \\
                  \sii                                                & p \to (p \lor q)             & \text{Doble negación}           \\
                  \sii                                                & \neg p \lor (p \lor q)       & \text{Definición condicional}   \\
                  \sii                                                & (\neg p \lor p) \lor q       & \text{Asociatividad}            \\
                  \sii                                                & True \lor q                                                    \\
                  \sii                                                & True                         & \text{Conjunción \textit{True}}
              \end{array}
          \]

          \textcolor{red}{Las fórmulas no son equivalentes.}

    \item \begin{itemize}
              \item $((True \land p) \land (\neg p \lor False)) \to \neg(\neg p \land q)$
              \item $p \land \neg q$
          \end{itemize}

          \[
              \begin{array}{rl|l}
                       & ((True \land p) \land (\neg p \lor False)) \to \neg(\neg p \land q)                                                               \\
                  \sii & (p \land \neg p) \to \neg (\neg p \lor q)                           & \text{Conjunción \textit{True} y disyunción \textit{False}} \\
                  \sii & False \to \neg (\neg p \lor q)                                      & \text{Contradicción}                                        \\
                  \sii & True
              \end{array}
          \]


          \textcolor{red}{Las fórmulas no son equivalentes.}

    \item \begin{itemize}
              \item $(p \lor (\neg p \land q))$
              \item $\neg p \to q$
          \end{itemize}

          \[
              \begin{array}{rl|l}
                  (p \lor (\neg p \land q)) \sii & (p \lor \neg p) \land (p \lor q) & \text{Distributiva}           \\
                  \sii                           & True \land (p \lor q)                                            \\
                  \sii                           & p \lor q                                                         \\
                  \sii                           & \neg(\neg p) \lor q              & \text{Doble negación}         \\
                  \sii                           & \neg p \to q                     & \text{Definición condicional}
              \end{array}
          \]

          Las fórmulas son equivalentes.
    \item
          \begin{itemize}
              \item $p \to (q \land \neg (q \to r))$
              \item $(\neg p \lor q) \land (\neg p \lor (q \land \neg r))$
          \end{itemize}

          \[
              \begin{array}{rl|l}
                  p \to (q \land \neg (q \to r)) \sii & \neg (\neg p) \to (q \land \neg (\neg (\neg q) \to r)) & \text{Doble negación}         \\
                  \sii                                & \neg p \lor (q \land \neg (\neg q \lor r))             & \text{Definición condicional} \\
                  \sii                                & \neg p \lor (q \land (q \land \neg r))                 & \text{De Morgan}              \\
                  \sii                                & (\neg p \lor q) \land (\neg p \lor (q \land \neg r))   & \text{Distributiva}
              \end{array}
          \]

          Las fórmulas son equivalentes.
\end{enumerate}

\subsection{Ejercicio 4}
Determinar si las siguientes fórmulas son tautologías, contradicciones o contingencias.

\begin{enumerate}[label=\alph*)]
    \item $(p \lor \neg p)$ Tautología

          \begin{tabular}{|c|c|}
              \hline
              $p$ & $(p \lor \neg p)$ \\
              \hline
              T   & \textbf{T}        \\
              F   & \textbf{T}        \\
              \hline
          \end{tabular}

    \item $(p \land \neg p)$ Contradicción

          \begin{tabular}{|c|c|}
              \hline
              $p$ & $(p \land \neg p)$ \\
              \hline
              T   & \textbf{F}         \\
              F   & \textbf{F}         \\
              \hline
          \end{tabular}

    \item $((\neg p \lor q) \sii (p \to q))$ Tautología

          \begin{tabular}{|c|c|c|c|c|}
              \hline
              $p$ & $q$ & $(\neg p \lor q)$ & $(p \to q)$ & $((\neg p \lor q) \sii (p \to q))$ \\
              \hline
              T   & T   & T                 & T           & \textbf{T}                         \\
              T   & F   & F                 & F           & \textbf{T}                         \\
              F   & T   & T                 & T           & \textbf{T}                         \\
              F   & F   & T                 & T           & \textbf{T}                         \\
              \hline
          \end{tabular}

    \item $((p \land q) \to p)$ Tautología

          \begin{tabular}{|c|c|c|c|}
              \hline
              $p$ & $q$ & $(p \land q)$ & $((p \land q) \to p)$ \\
              \hline
              T   & T   & T             & \textbf{T}            \\
              T   & F   & F             & \textbf{T}            \\
              F   & T   & F             & \textbf{T}            \\
              F   & F   & F             & \textbf{T}            \\
              \hline
          \end{tabular}

    \item $((p \land (q \lor r)) \sii ((p \land q) \lor (p \land r)))$

          \begin{tabular}{|c|c|c|c|c|c|c|c|c|}
              \hline
              $p$ & $q$ & $r$ & $(q \lor r)$ & $(p \land (q \lor r))$ & $(p \land q)$ & $(p \land r)$ & $((p \land q) \lor (p \land r))$ & Fórmula del enunciado \\
              \hline
              T   & T   & T   & T            & T                      & T             & T             & T                                & \textbf{T}            \\
              T   & T   & F   & T            & T                      & T             & F             & T                                & \textbf{T}            \\
              T   & F   & T   & T            & T                      & F             & T             & T                                & \textbf{T}            \\
              T   & F   & F   & F            & F                      & F             & F             & F                                & \textbf{T}            \\
              F   & T   & T   & T            & F                      & F             & F             & F                                & \textbf{T}            \\
              F   & T   & F   & T            & F                      & F             & F             & F                                & \textbf{T}            \\
              F   & F   & T   & T            & F                      & F             & F             & F                                & \textbf{T}            \\
              F   & F   & F   & F            & F                      & F             & F             & F                                & \textbf{T}            \\
              \hline
          \end{tabular}

    \item $((p \to (q \to r)) \to ((p \to q) \to (p \to r)))$ \hacer
\end{enumerate}

\subsection{Ejercicio 5}
Determinar la relación de fuerza de los siguientes pares de fórmulas:

\begin{enumerate}[label=\alph*)]
    \item $True, False$

          $False$ es más fuerte que $True$.

    \item $(p \land q), (p \lor q)$

          $(p \land q)$ es más fuerte que $(p \lor q)$.

    \item $p, (p \land q)$

          $(p \land q)$ es más fuerte que $p$.

    \item $p, (p \lor q)$

          $p$ es más fuerte que $(p \lor q)$.

    \item $p, q$

          No hay relación de fuerza.

    \item $p, (p \to q)$

          No hay relación de fuerza.
\end{enumerate}

\subsection{Ejercicio 6}
Asumiendo que el valor de verdad de $b$ y $c$ es \textit{verdadero}, el de $a$ es \textit{falso} y el de $x$ e $y$ es \textit{indefinido}, indicar cuáles de los operadores deben ser operadores "luego" para que la expresión no se indefina nunca:

\begin{enumerate}[label=\alph*)]
    \item $(\neg x \lor b)$

          Se indefine siempre.

    \item $((c \lor (y \land a)) \lor b)$

          $((c \lor_L (y \land a)) \lor b)$

    \item $\neg (c \lor y)$

          $\neg (c \lor_L y)$

    \item $(\neg (c \lor y) \sii (\neg c \land \neg y))$

          $(\neg (c \lor_L y) \sii (\neg c \land_L \neg y))$

    \item $((c \lor y) \land (a \lor b))$

          $((c \lor_L y) \land (a \lor b))$

    \item $(((c \lor y) \land (a \lor b)) \sii (c \lor (y \land a) \lor b))$

          $(((c \lor_L y) \land (a \lor b)) \sii (c \lor_L (y \land a) \lor b))$

    \item $(\neg c \land \neg y)$

          $(\neg c \land_L \neg y)$
\end{enumerate}

\subsection{Ejercicio 7}
Sean $p$, $q$ y $r$ tres variables de las que se sabe que:

\begin{itemize}
    \item $p$ y $q$ nunca esán indefinidas,
    \item $r$ se indefine sii $q$ es \textit{verdadera}
\end{itemize}

Proponer una fórmula que nunca se indefina, utilizando siempre las tres variables y que sea verdadera si y solo si se cumple que:

\begin{enumerate}[label=\alph*)]
    \item Al menos una es verdadera.

          $(p \lor q) \lor_L r$

    \item Ninguna es verdadera.

          $\neg (p \lor q) \land_L \neg r$

    \item Exactamente una de las tres es verdadera. \hacer

    \item Sólo $p$ y $q$ son verdaderas. \hacer

    \item No todas al mismo tiempo son verdaderas. \hacer

    \item $r$ es verdadera. \hacer
\end{enumerate}

\subsection{Ejercicio 8}
Determinar, para cada aparición de variables, si dicha aparición se encuentra libre o ligada. En cada caso de estar ligada, aclarar a qué cuantificador lo está. En los casso en que sea posible, proponer valores para las variables libres de modo tal que las expresiones sean verdaderas.

\begin{enumerate}[label=\alph*)]
    \item $(\forall x: \mathbb{Z})(0 \leq x < n \to x + y = z)$

          Ligadas: $x$ al cuantificador $(\forall x: \mathbb{Z})$.

          Libres: $n, y, z$. Posibles valores: $n = 1, y = z = 5$.

    \item $(\forall x: \mathbb{Z})((\forall y: \mathbb{Z})((0 \leq x < n \land 0 \leq y < m) \to x + y = z))$

          Ligadas: $x$ al cuantificador $(\forall x: \mathbb{Z})$, $y$ al cuantificador $(\forall y: \mathbb{Z})$.

          Libres: $n, m, z$. Posibles valores: $n = 1, m = 1, z = 0$.

    \item $(\forall j: \mathbb{Z})(0 \leq j < 10 \to j < 0)$

          Ligadas: $j$ al cuantificador $(\forall j: \mathbb{Z})$.

          En este caso la expresión es siempre falsa.

    \item $(\forall j: \mathbb{Z})(j \leq 0 \to P(j)) \land P(j)$

          Ligadas: $j$ al cuantificador $(\forall j: \mathbb{Z})$.

          Libres: $j$.

          El valor de verdad depende de $P(j)$.
\end{enumerate}
\end{document}

\documentclass[11pt, a4paper]{article}
%

\usepackage{../caratuladc/caratula/caratula}

\usepackage{multicol}
\usepackage{enumitem}
\usepackage{xcolor}
\usepackage{etoolbox}

\newcommand{\hacer}{\textcolor{red}{HACER!}}
\newcommand{\punto}{\text{.}}
\usepackage[spanish,activeacute,es-tabla]{babel}
\usepackage[utf8]{inputenc}
\usepackage{ifthen}
\usepackage{listings}
\usepackage{dsfont}
\usepackage{subcaption}
\usepackage{amsmath}
\usepackage[top=1cm,bottom=2cm,left=1cm,right=1cm]{geometry}%
\usepackage{color}%
\usepackage{changepage}
\newcommand{\tocarEspacios}{%
	\addtolength{\leftskip}{3em}%
	\setlength{\parindent}{0em}%
}

\newcommand{\Indent}{\hspace*{0.75cm}}

% Especificacion de procs

\newcommand{\In}{\textsf{in }}
\newcommand{\Out}{\textsf{out }}
\newcommand{\Inout}{\textsf{inout }}

\newcommand{\encabezadoDeProc}[4]{%
% Ponemos la palabrita problema en tt
%  \noindent%
{\normalfont\bfseries\ttfamily proc}%
% Ponemos el nombre del problema
\ %
{\normalfont\ttfamily #2}%
\
% Ponemos los parametros
(#3)%
\ifblank{#4}{}{%
	% Por ultimo, va el tipo del resultado
	\ : #4}
}

\newenvironment{proc}[4][res]{%

% El parametro 1 (opcional) es el nombre del resultado
% El parametro 2 es el nombre del problema
% El parametro 3 son los parametros
% El parametro 4 es el tipo del resultado
% Preambulo del ambiente problema
% Tenemos que definir los comandos requiere, asegura, modifica y aux
\newcommand{\requiere}[2][]{%
{\normalfont\bfseries\ttfamily requiere\;}%
\ifthenelse{\equal{##1}{variaslineas}}{\{%
\begin{adjustwidth}{+5em}{}
	\ensuremath{##2}
\end{adjustwidth}
\}
{\normalfont\bfseries\,\par}%
}
{%
\{\ensuremath{##2}\}%
{\normalfont\bfseries\,\par}%
}
}

\newcommand{\asegura}[2][]{%
{\normalfont\bfseries\ttfamily asegura\;}%
\ifthenelse{\equal{##1}{variaslineas}}{\{%
\begin{adjustwidth}{+5em}{}
	\ensuremath{##2}
\end{adjustwidth}
\}
{\normalfont\bfseries\,\par}%
}
{%
\{\ensuremath{##2}\}%
{\normalfont\bfseries\,\par}%
}
}
\renewcommand{\aux}[4]{%
	{\normalfont\bfseries\ttfamily aux\ }%
		{\normalfont\ttfamily ##1}%
	\ifthenelse{\equal{##2}{}}{}{\ (##2)}\ : ##3\, = \ensuremath{##4}%
	{\normalfont\bfseries\,;\par}%
}
\renewcommand{\pred}[3]{%
{\normalfont\bfseries\ttfamily pred }%
	{\normalfont\ttfamily ##1}%
\ifthenelse{\equal{##2}{}}{}{\ (##2) }%
\{%
\begin{adjustwidth}{+5em}{}
	\ensuremath{##3}
\end{adjustwidth}
\}%
{\normalfont\bfseries\,\par}%
}

\newcommand{\res}{#1}
\vspace{1ex}
\noindent
\encabezadoDeProc{#1}{#2}{#3}{#4}
% Abrimos la llave
\par%
\tocarEspacios
}
{
% Cerramos la llave
\vspace{1ex}
}

\newcommand{\aux}[4]{%
	{\normalfont\bfseries\ttfamily\noindent aux\ }%
		{\normalfont\ttfamily #1}%
	\ifthenelse{\equal{#2}{}}{}{\ (#2)}\ : #3\, = \ensuremath{#4}%
	{\normalfont\bfseries\,;\par}%
}

\newcommand{\pred}[3]{%
{\normalfont\bfseries\ttfamily\noindent pred }%
	{\normalfont\ttfamily #1}%
\ifthenelse{\equal{#2}{}}{}{\ (#2) }%
\{%
\begin{adjustwidth}{+2em}{}
	\ensuremath{#3}
\end{adjustwidth}
\}%
{\normalfont\bfseries\,\par}%
}

% Tipos

\newcommand{\nat}{\ensuremath{\mathbb{N}}}
\newcommand{\reals}{\ensuremath{\mathbb{R}}}
\newcommand{\ent}{\ensuremath{\mathds{Z}}}
\newcommand{\float}{\ensuremath{\mathds{R}}}
\newcommand{\bool}{\ensuremath{\mathsf{Bool}}}
\newcommand{\cha}{\ensuremath{\mathsf{Char}}}
\newcommand{\str}{\ensuremath{\mathsf{String}}}
\newcommand{\dict}[1]{\ensuremath{\mathsf{dict}\lrangle{#1}}}
\newcommand{\conj}[1]{\ensuremath{\mathsf{conj}\lrangle{#1}}}
\newcommand{\tupla}[1]{\ensuremath{\mathsf{tupla}\lrangle{#1}}}
\newcommand{\struct}[1]{\ensuremath{\mathsf{struct}\lrangle{#1}}}

% Logica

\newcommand{\True}{\ensuremath{\mathrm{true}}}
\newcommand{\False}{\ensuremath{\mathrm{false}}}
\newcommand{\Then}{\ensuremath{\rightarrow}}
\newcommand{\Iff}{\ensuremath{\leftrightarrow}}
\newcommand{\implica}{\ensuremath{\longrightarrow}}
\newcommand{\IfThenElse}[3]{\ensuremath{\mathsf{if}\ #1\ \mathsf{then}\ #2\ \mathsf{else}\ #3\ \mathsf{fi}}}
\newcommand{\If}[3]{\text{\normalfont\ttfamily IfThenElse}(\ensuremath{#1,\;#2,\;#3})}
\newcommand{\yLuego}{\land _L}
\newcommand{\oLuego}{\lor _L}
\newcommand{\thenLuego}{\Then _L}

\newcommand{\cuantificador}[5]{%
	\ensuremath{(#2 #3: #4)\ (%
		\ifthenelse{\equal{#1}{multLineas}}{
			$ % exiting math mode
				\begin{adjustwidth}{+2em}{}
					$#5$%
				\end{adjustwidth}%
			$ % entering math mode
		}{
			#5
		}
		)}
}

\newcommand{\existe}[4][]{%
	\cuantificador{#1}{\exists}{#2}{#3}{#4}
}
\newcommand{\paraTodo}[4][]{%
	\cuantificador{#1}{\forall}{#2}{#3}{#4}
}

\newcommand{\Def}[1]{\text{\normalfont\ttfamily def}(#1)}

%listas

\newcommand{\TLista}[1]{\ensuremath{seq \langle #1\rangle}}
\newcommand{\lvacia}{\ensuremath{\langle\,\rangle}}
\newcommand{\lv}{\ensuremath{\langle\,\rangle}}
\newcommand{\longitud}[1]{\ensuremath{|#1|}}
\newcommand{\cons}[1]{\ensuremath{\mathsf{addFirst}}(#1)}
\newcommand{\indice}[1]{\ensuremath{\mathsf{indice}}(#1)}
\newcommand{\conc}[1]{\ensuremath{\mathsf{concat}}(#1)}
\newcommand{\head}[1]{\ensuremath{\mathsf{head}}(#1)}
\newcommand{\tail}[1]{\ensuremath{\mathsf{tail}}(#1)}
\newcommand{\sub}[1]{\ensuremath{\mathsf{subseq}}(#1)}
\newcommand{\en}[1]{\ensuremath{\mathsf{en}}(#1)}
\newcommand{\cuenta}[2]{\mathsf{cuenta}\ensuremath{(#1, #2)}}
\newcommand{\suma}[1]{\mathsf{suma}(#1)}
\newcommand{\twodots}{\ensuremath{\mathrm{..}}}
\newcommand{\masmas}{\ensuremath{+\!+\;}}
\newcommand{\matriz}[1]{\TLista{\TLista{#1}}}
\newcommand{\seqchar}{\TLista{\cha}}
\newcommand{\lrangle}[1]{\ensuremath{\langle#1\rangle}}

%dict

\newcommand{\setKey}[1]{\ensuremath{\mathsf{setKey}(#1)}}
\newcommand{\delKey}[1]{\ensuremath{\mathsf{delKey}(#1)}}

\renewcommand{\wp}[2]{
	\ensuremath{\textit{wp}(\textbf{\lstinline{#1}}, #2)}
}

\newcommand{\hoare}[3]{\ensuremath{\{#1\} \; #2 \; \{#3\}}}

\renewcommand{\lstlistingname}{Código}
\lstset{% general command to set parameter(s)
	language=Java,
	morekeywords={func, endif, endwhile, skip, end, var, then},
	basewidth={0.47em,0.40em},
	columns=fixed, fontadjust, resetmargins, xrightmargin=5pt, xleftmargin=15pt,
	flexiblecolumns=false, tabsize=4, breaklines, breakatwhitespace=false, extendedchars=true,
	numbers=left, numberstyle=\tiny, stepnumber=1, numbersep=9pt,
	frame=l, framesep=3pt,
	captionpos=b,
}

% TADs

\newenvironment{tad}[2]{
\newcommand{\tadtype}{%
	\ensuremath{#1 \ifblank{#2}{}{\lrangle{#2}}}
}

\renewcommand{\pred}[3]{%
{\normalfont\bfseries\ttfamily pred }%
	{\normalfont\ttfamily ##1}%
\ifthenelse{\equal{##2}{}}{}{\ (##2) }%
\{%
\begin{adjustwidth}{+5em}{}
	\ensuremath{##3}
\end{adjustwidth}
\}
{\normalfont\bfseries\,\par}%
}

\vspace{1ex}
\noindent
{\normalfont\bfseries\ttfamily\large TAD #1\ifthenelse{\equal{#2}{}}{}{$<$#2$>$}} \{
% Abrimos la llave
\par%
\tocarEspacios
}{
\vspace{1ex} \par\}
}

\newcommand{\obs}[2]{
	obs #1: \ensuremath{#2}\par
}

\newenvironment{module}[4]{
\newcommand{\tadtype}{%
	\ensuremath{#3 \ifblank{#4}{}{\lrangle{#4}}}
}

\newcommand{\moduletype}{%
	\ensuremath{#1 \ifblank{#2}{}{\lrangle{#2}}}
}


\renewcommand{\pred}[3]{%
{\normalfont\bfseries\ttfamily pred }%
	{\normalfont\ttfamily ##1}%
\ifthenelse{\equal{##2}{}}{}{\ (##2) }%
\{%
\begin{adjustwidth}{+5em}{}
	\ensuremath{##3}
\end{adjustwidth}
\}
{\normalfont\bfseries\,\par}%
}

\vspace{1ex}
\noindent
{\normalfont\bfseries\ttfamily\large Módulo #1\ifthenelse{\equal{#2}{}}{}{$<$#2$>$} implementa #3\ifthenelse{\equal{#4}{}}{}{$<$#4$>$}} \{
% Abrimos la llave
\par%
\tocarEspacios
}{
\vspace{1ex} \par\}
}

\newcommand{\var}[2]{
	var #1: \;\ensuremath{#2}\par
}
\newcommand{\compl}[1]{Complejidad: $#1$}

% Tipos de datos de implementación
\newcommand{\Int}{\ensuremath{\mathsf{int}}}
\newcommand{\Real}{\ensuremath{\mathsf{real}}}
\newcommand{\Bool}{\ensuremath{\mathsf{bool}}}
\newcommand{\Char}{\ensuremath{\mathsf{char}}}
\newcommand{\String}{\ensuremath{\mathsf{string}}}
\newcommand{\Array}[1]{\ensuremath{\mathsf{Array<#1>}}}

\begin{document}

\titulo{Guia 1}
\materia{Algoritmos y Estructuras de Datos I}
\fecha{2do cuatrimestre 2024}

\integrante{Federico Barberón}{112/24}{jfedericobarberonj@gmail.com}
%Carátula
\maketitle
\newpage

%Indice
\tableofcontents
\newpage

\section{Guia 1}

\subsection{Ejercicio 1}
Determinar los valores de verdad de las siguientes proposiciones cuando el valor de verdad de $a$, $b$ y $c$ es \textit{verdadero} y el de $x$ e $y$ es \textit{falso}.

\begin{multicols}{2}
      \begin{enumerate}[label=\alph*)]
            \item $(\neg x \lor b)$ \True
            \item $((c \lor (y \land a)) \lor b)$ \True
            \item $\neg (c \lor y)$ \False
            \item $\neg (y \lor c)$ \False
            \item $(\neg(c \lor y) \Iff (\neg c \land \neg y))$ \True
            \item $((c \lor y) \land (a \lor b))$ \True
            \item $(((c \lor y) \land (a \lor b)) \Iff (c \lor (y \land a) \lor b))$ \True
            \item $(\neg c \land \neg y)$ \False
      \end{enumerate}
\end{multicols}

\subsection{Ejercicio 2}
Considere la siguiente oración: ``Si es mi cumpleaños o hay torta, entonces hay torta''.

\begin{itemize}
      \item Escribir usando lógica proposicional y realizar la tabla de verdad

            $(p \lor q) \Then q$

            \begin{tabular}{|c|c|c|c|}
                  \hline
                  $p$ & $q$ & $(p \lor q)$ & $(p \lor q) \Then q$ \\
                  \hline
                  T   & T   & T            & T                    \\
                  \hline
                  T   & F   & T            & F                    \\
                  \hline
                  F   & T   & T            & T                    \\
                  \hline
                  F   & F   & F            & T                    \\
                  \hline
            \end{tabular}

      \item Asumiendo que la oración es verdadera y hay una torta, qué se puede concluir?

            Se concluye que puede o no ser su cumpleaños.

      \item Asumiendo que la oración es verdadera y no hay una torta, qué se puede concluir?

            Se concluye que NO es su cumpleaños.

      \item Suponiendo que la oración es mentira (es falsa), se puede concluir algo?

            Se concluye que es su cumpleaños pero no hay torta :(
\end{itemize}

\subsection{Ejercicio 3}
Usando reglas de equivalencia (conmutatividad, asociatividad, De Morgan, etc) determinar si los siguientes pares de fórmulas son equivalencias. Indicar en cada paso qué regla se utilizó.

\begin{enumerate}[label=\alph*)]
      \item \begin{itemize}
                  \item $(p \lor q) \land (p \lor r)$
                  \item $\neg p \Then (q \land r)$
            \end{itemize}

            \[
                  \begin{array}{rl|l}
                        (p \lor q) \land (p \lor r) \Iff & p \lor (q \land r)             & \text{Distributiva}           \\
                        \Iff                             & \neg (\neg p) \lor (q \land r) & \text{Doble negación}         \\
                        \Iff                             & \neg p \Then (q \land r)       & \text{Definición condicional}
                  \end{array}
            \]

            Las fórmulas son equivalentes.

      \item \begin{itemize}
                  \item $\neg (\neg p) \Then (\neg (\neg p \land \neg q))$
                  \item $q$

            \end{itemize}

            \[
                  \begin{array}{rl|l}
                        \neg (\neg p) \Then (\neg (\neg p \land \neg q)) \Iff & \neg (\neg p) \Then (p \lor q) & \text{De Morgan}                \\
                        \Iff                                                  & p \Then (p \lor q)             & \text{Doble negación}           \\
                        \Iff                                                  & \neg p \lor (p \lor q)         & \text{Definición condicional}   \\
                        \Iff                                                  & (\neg p \lor p) \lor q         & \text{Asociatividad}            \\
                        \Iff                                                  & True \lor q                                                      \\
                        \Iff                                                  & True                           & \text{Conjunción \textit{True}}
                  \end{array}
            \]

            \textcolor{red}{Las fórmulas no son equivalentes.}

      \item \begin{itemize}
                  \item $((True \land p) \land (\neg p \lor False)) \Then \neg(\neg p \land q)$
                  \item $p \land \neg q$
            \end{itemize}

            \[
                  \begin{array}{rl|l}
                             & ((True \land p) \land (\neg p \lor False)) \Then \neg(\neg p \land q)                                                               \\
                        \Iff & (p \land \neg p) \Then \neg (\neg p \lor q)                           & \text{Conjunción \textit{True} y disyunción \textit{False}} \\
                        \Iff & False \Then \neg (\neg p \lor q)                                      & \text{Contradicción}                                        \\
                        \Iff & True
                  \end{array}
            \]


            \textcolor{red}{Las fórmulas no son equivalentes.}

      \item \begin{itemize}
                  \item $(p \lor (\neg p \land q))$
                  \item $\neg p \Then q$
            \end{itemize}

            \[
                  \begin{array}{rl|l}
                        (p \lor (\neg p \land q)) \Iff & (p \lor \neg p) \land (p \lor q) & \text{Distributiva}           \\
                        \Iff                           & True \land (p \lor q)                                            \\
                        \Iff                           & p \lor q                                                         \\
                        \Iff                           & \neg(\neg p) \lor q              & \text{Doble negación}         \\
                        \Iff                           & \neg p \Then q                   & \text{Definición condicional}
                  \end{array}
            \]

            Las fórmulas son equivalentes.
      \item
            \begin{itemize}
                  \item $p \Then (q \land \neg (q \Then r))$
                  \item $(\neg p \lor q) \land (\neg p \lor (q \land \neg r))$
            \end{itemize}

            \[
                  \begin{array}{rl|l}
                        p \Then (q \land \neg (q \Then r)) \Iff & \neg (\neg p) \Then (q \land \neg (\neg (\neg q) \Then r)) & \text{Doble negación}         \\
                        \Iff                                    & \neg p \lor (q \land \neg (\neg q \lor r))                 & \text{Definición condicional} \\
                        \Iff                                    & \neg p \lor (q \land (q \land \neg r))                     & \text{De Morgan}              \\
                        \Iff                                    & (\neg p \lor q) \land (\neg p \lor (q \land \neg r))       & \text{Distributiva}
                  \end{array}
            \]

            Las fórmulas son equivalentes.
\end{enumerate}

\subsection{Ejercicio 4}
Determinar si las siguientes fórmulas son tautologías, contradicciones o contingencias.

\begin{enumerate}[label=\alph*)]
      \item $(p \lor \neg p)$ Tautología

            \begin{tabular}{|c|c|}
                  \hline
                  $p$ & $(p \lor \neg p)$ \\
                  \hline
                  T   & \textbf{T}        \\
                  F   & \textbf{T}        \\
                  \hline
            \end{tabular}

      \item $(p \land \neg p)$ Contradicción

            \begin{tabular}{|c|c|}
                  \hline
                  $p$ & $(p \land \neg p)$ \\
                  \hline
                  T   & \textbf{F}         \\
                  F   & \textbf{F}         \\
                  \hline
            \end{tabular}

      \item $((\neg p \lor q) \Iff (p \Then q))$ Tautología

            \begin{tabular}{|c|c|c|c|c|}
                  \hline
                  $p$ & $q$ & $(\neg p \lor q)$ & $(p \Then q)$ & $((\neg p \lor q) \Iff (p \Then q))$ \\
                  \hline
                  T   & T   & T                 & T             & \textbf{T}                           \\
                  T   & F   & F                 & F             & \textbf{T}                           \\
                  F   & T   & T                 & T             & \textbf{T}                           \\
                  F   & F   & T                 & T             & \textbf{T}                           \\
                  \hline
            \end{tabular}

      \item $((p \land q) \Then p)$ Tautología

            \begin{tabular}{|c|c|c|c|}
                  \hline
                  $p$ & $q$ & $(p \land q)$ & $((p \land q) \Then p)$ \\
                  \hline
                  T   & T   & T             & \textbf{T}              \\
                  T   & F   & F             & \textbf{T}              \\
                  F   & T   & F             & \textbf{T}              \\
                  F   & F   & F             & \textbf{T}              \\
                  \hline
            \end{tabular}

      \item $((p \land (q \lor r)) \Iff ((p \land q) \lor (p \land r)))$

            \begin{tabular}{|c|c|c|c|c|c|c|c|c|}
                  \hline
                  $p$ & $q$ & $r$ & $(q \lor r)$ & $(p \land (q \lor r))$ & $(p \land q)$ & $(p \land r)$ & $((p \land q) \lor (p \land r))$ & Fórmula del enunciado \\
                  \hline
                  T   & T   & T   & T            & T                      & T             & T             & T                                & \textbf{T}            \\
                  T   & T   & F   & T            & T                      & T             & F             & T                                & \textbf{T}            \\
                  T   & F   & T   & T            & T                      & F             & T             & T                                & \textbf{T}            \\
                  T   & F   & F   & F            & F                      & F             & F             & F                                & \textbf{T}            \\
                  F   & T   & T   & T            & F                      & F             & F             & F                                & \textbf{T}            \\
                  F   & T   & F   & T            & F                      & F             & F             & F                                & \textbf{T}            \\
                  F   & F   & T   & T            & F                      & F             & F             & F                                & \textbf{T}            \\
                  F   & F   & F   & F            & F                      & F             & F             & F                                & \textbf{T}            \\
                  \hline
            \end{tabular}

      \item $((p \Then (q \Then r)) \Then ((p \Then q) \Then (p \Then r)))$ \hacer
\end{enumerate}

\subsection{Ejercicio 5}
Determinar la relación de fuerza de los siguientes pares de fórmulas:

\begin{enumerate}[label=\alph*)]
      \item $True, False$

            $False$ es más fuerte que $True$.

      \item $(p \land q), (p \lor q)$

            $(p \land q)$ es más fuerte que $(p \lor q)$.

      \item $p, (p \land q)$

            $(p \land q)$ es más fuerte que $p$.

      \item $p, (p \lor q)$

            $p$ es más fuerte que $(p \lor q)$.

      \item $p, q$

            No hay relación de fuerza.

      \item $p, (p \Then q)$

            No hay relación de fuerza.
\end{enumerate}

\subsection{Ejercicio 6}
Asumiendo que el valor de verdad de $b$ y $c$ es \textit{verdadero}, el de $a$ es \textit{falso} y el de $x$ e $y$ es \textit{indefinido}, indicar cuáles de los operadores deben ser operadores ``luego'' para que la expresión no se indefina nunca:

\begin{enumerate}[label=\alph*)]
      \item $(\neg x \lor b)$

            Se indefine siempre.

      \item $((c \lor (y \land a)) \lor b)$

            $((c \oLuego (y \land a)) \lor b)$

      \item $\neg (c \lor y)$

            $\neg (c \oLuego y)$

      \item $(\neg (c \lor y) \Iff (\neg c \land \neg y))$

            $(\neg (c \oLuego y) \Iff (\neg c \yLuego \neg y))$

      \item $((c \lor y) \land (a \lor b))$

            $((c \oLuego y) \land (a \lor b))$

      \item $(((c \lor y) \land (a \lor b)) \Iff (c \lor (y \land a) \lor b))$

            $(((c \oLuego y) \land (a \lor b)) \Iff (c \oLuego (y \land a) \lor b))$

      \item $(\neg c \land \neg y)$

            $(\neg c \yLuego \neg y)$
\end{enumerate}

\subsection{Ejercicio 7}
Sean $p$, $q$ y $r$ tres variables de las que se sabe que:

\begin{itemize}
      \item $p$ y $q$ nunca esán indefinidas,
      \item $r$ se indefine sii $q$ es \textit{verdadera}
\end{itemize}

Proponer una fórmula que nunca se indefina, utilizando siempre las tres variables y que sea verdadera si y solo si se cumple que:

\begin{enumerate}[label=\alph*)]
      \item Al menos una es verdadera.

            $(p \lor q) \oLuego r$

      \item Ninguna es verdadera.

            $\neg (p \lor q) \yLuego \neg r$

      \item Exactamente una de las tres es verdadera. \hacer

      \item Sólo $p$ y $q$ son verdaderas. \hacer

      \item No todas al mismo tiempo son verdaderas. \hacer

      \item $r$ es verdadera. \hacer
\end{enumerate}

\subsection{Ejercicio 8}
Determinar, para cada aparición de variables, si dicha aparición se encuentra libre o ligada. En cada caso de estar ligada, aclarar a qué cuantificador lo está. En los casso en que sea posible, proponer valores para las variables libres de modo tal que las expresiones sean verdaderas.

\begin{enumerate}[label=\alph*)]
      \item $\paraTodo{x}{\ent}{0 \leq x < n \Then x + y = z}$

            Ligadas: $x$ al cuantificador $\forall$.

            Libres: $n, y, z$. Posibles valores: $n = 1, y = z = 5$.

      \item $\paraTodo{x}{\ent}{\paraTodo{y}{\ent}{0 \leq x < n \land 0 \leq y < m} \Then x + y = z}$

            Ligadas: $x, y$ a los cuantificadores $\forall$.

            Libres: $n, m, z$. Posibles valores: $n = 1, m = 1, z = 0$.

      \item $\paraTodo{j}{\ent}{0 \leq j < 10 \Then j < 0}$

            Ligadas: $j$ al cuantificador $\forall$.

            En este caso la expresión es siempre falsa.

      \item $\paraTodo{j}{\ent}{j \leq 0 \Then P(j)} \land P(j)$

            Ligadas: $j$ al cuantificador $\forall$.

            Libres: $j$.

            El valor de verdad depende de $P(j)$.
\end{enumerate}

\subsection{Ejercicio 9}
Sea $P(x: \ent)$ y $Q(x: \ent)$ dos predicados cualquiera. Explicar cuál es el error de traducción a fórumlas de los siguientes enunciados. Dar un ejemplo en el cuál sucede el problema y luego corregirlo.

\begin{enumerate}[label=\alph*)]
      \item \textit{``Todos los naturales menores a 10 cumplen P''}

            $\paraTodo{i}{\ent}{(0 \leq i < 10) \land P(i)}$

            El error está en el operador $\land$ ya que el cuantificador universal ($\forall$) generaliza la conjunción, por lo tanto cuando $i$ está fuera del rango $[0,10)$ toda la fórmula se convierte en $False$.

            Solución: $\paraTodo{i}{\ent}{(0 \leq i < 10) \thenLuego P(i)}$

      \item \textit{``Algún natural menor a 10 cumple P''}

            $\existe{i}{\ent}{(0 \leq i < 10) \Then P(i)}$

            El error está en el operador $\Then$ y es lo contrario al item anterior. Como el cuantificador existencial ($\exists$) generaliza la disyunción, cuando $i$ está fuera del rango $[0,10)$ la implicación se vuelve $True$ toda la fórmula también.

            Solución: $\existe{i}{\ent}{(0 \leq i < 10) \yLuego P(i)}$

      \item \textit{``Todos los naturales menores a 10 que cumplen P, cumplen Q''}

            $\paraTodo{x}{\ent}{(0 \leq x < 10) \Then (P(x) \land Q(x))}$

            El error está en que si hay algun natural menor a 10 que no cumple $P$, entonces toda la fórmula se vuelve $False$.

            Solución: $\paraTodo{x}{\ent}{(0 \leq x < 10) \thenLuego (P(x) \thenLuego Q(x))}$

      \item \textit{``No hay ningún natural menor a 10 que cumpla P y Q''}

            $\neg (\existe{x}{\ent}{0 \leq x < 10 \land P(x)} \land \neg (\existe{x}{\ent}{0 \leq x < 10 \land Q(x)}))$

            El error está en que el enunciado dice que no existe natural menor a 10 que cumpla $P$ \textbf{y} $Q$, mientras que la fórmula dice que no existe natural menor a 10 que cumpla $P$ \textbf{ó} $Q$.

            Solución: $\neg (\existe{x}{\ent}{0 \leq x < 10 \yLuego (P(x) \land Q(x))})$

\end{enumerate}

\subsection{Ejercicio 10}
Sean $P(x: \ent)$ y $Q(x: \ent)$ dos predicados cualesquiera que nunca se indefinen. Escribir el predicado asociado a cada uno de los siguientes enunciados:

Sea \par
\pred{esMenorQueDiez}{x: \ent}{0 \leq x < 10}
\pred{esPar}{x: \ent}{x \text{ mod } 2 = 0}

\begin{itemize}
      \item \textit{``Existe un único número natural menor a 10 que cumple P''}

            $\existe{x}{\ent}{esMenorQueDiez(x) \yLuego (P(x) \land
                        \neg \existe{y}{\ent}{(esMenorQueDiez(y) \land (y \neq x)) \yLuego P(y)})}$
      \item \textit{``Existen al menos dos números naturales menores a 10 que cumplen P''}

            $\existe{x}{\ent}{esMenorQueDiez(x) \yLuego (P(x) \land
                        \existe{y}{\ent}{(esMenorQueDiez(y) \land y \neq x) \yLuego P(y)})}$

      \item \textit{``Existen exactamente dos números naturales menores a 10 que cumple P''}

            $\existe[multLineas]{x}{\ent}{esMenorQueDiez(x) \yLuego (P(x) \land
                        \existe[multLineas]{y}{\ent}{(esMenorQueDiez(y) \land y \neq x) \yLuego (P(y) \land
                              \neg \existe[multLineas]{z}{\ent}{esMenorQueDiez(y) \land (z \neq x \land z \neq y) \yLuego P(z)})})}$

      \item \textit{``Todos los enteros pares que cumplen P, no cumplen Q''}

            $\paraTodo{x}{\ent}{esPar(x) \thenLuego (P(x) \thenLuego \neg Q(x))}$

      \item \textit{``Si un entero cumple P y es impar, no cumple Q''}

            $\paraTodo{x}{\ent}{(P(x) \land \neg esPar(x)) \thenLuego \neg Q(x)}$

      \item \textit{``Todos los enteros pares cumplen P, y todos los enteros impares que no cumplen P cumplen Q''}

            $\paraTodo{x}{\ent}{esPar(x) \thenLuego P(x)} \land \paraTodo{x}{\ent}{(\neg esPar(x) \land \neg P(x)) \thenLuego Q(x)}$

      \item \textit{Si hay un número natural menor a 10 que no cumple P entonces ninguno natural menor a 10 cumple Q; y si todos los naturales menores a 10 cumplen P entonces hay al menos dos naturales menores a 10 que cumplen Q}

            $\existe{x}{\ent}{esMenorQueDiez(x) \yLuego \neg P(x)}
                  \thenLuego \paraTodo{x}{\ent}{esMenorQueDiez(x) \thenLuego \neg Q(x)} \land \\
                  \paraTodo{x}{\ent}{esMenorQueDiez(x) \yLuego (Q(x) \land
                        \existe{y}{\ent}{esMenorQueDiez(y) \yLuego (y \neq x \land Q(x))}
                        )}
            $
\end{itemize}

\end{document}

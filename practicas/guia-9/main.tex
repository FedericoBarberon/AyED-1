\documentclass[11pt, a4paper]{article}
%

\usepackage{../caratuladc/caratula/caratula}

\usepackage{multicol}
\usepackage{enumitem}
\usepackage{xcolor}
\usepackage{etoolbox}

\newcommand{\hacer}{\textcolor{red}{HACER!}}
\newcommand{\punto}{\text{.}}
\usepackage[spanish,activeacute,es-tabla]{babel}
\usepackage[utf8]{inputenc}
\usepackage{ifthen}
\usepackage{listings}
\usepackage{dsfont}
\usepackage{subcaption}
\usepackage{amsmath}
\usepackage[top=1cm,bottom=2cm,left=1cm,right=1cm]{geometry}%
\usepackage{color}%
\usepackage{changepage}
\newcommand{\tocarEspacios}{%
	\addtolength{\leftskip}{3em}%
	\setlength{\parindent}{0em}%
}

\newcommand{\Indent}{\hspace*{0.75cm}}

% Especificacion de procs

\newcommand{\In}{\textsf{in }}
\newcommand{\Out}{\textsf{out }}
\newcommand{\Inout}{\textsf{inout }}

\newcommand{\encabezadoDeProc}[4]{%
% Ponemos la palabrita problema en tt
%  \noindent%
{\normalfont\bfseries\ttfamily proc}%
% Ponemos el nombre del problema
\ %
{\normalfont\ttfamily #2}%
\
% Ponemos los parametros
(#3)%
\ifblank{#4}{}{%
	% Por ultimo, va el tipo del resultado
	\ : #4}
}

\newenvironment{proc}[4][res]{%

% El parametro 1 (opcional) es el nombre del resultado
% El parametro 2 es el nombre del problema
% El parametro 3 son los parametros
% El parametro 4 es el tipo del resultado
% Preambulo del ambiente problema
% Tenemos que definir los comandos requiere, asegura, modifica y aux
\newcommand{\requiere}[2][]{%
{\normalfont\bfseries\ttfamily requiere\;}%
\ifthenelse{\equal{##1}{variaslineas}}{\{%
\begin{adjustwidth}{+5em}{}
	\ensuremath{##2}
\end{adjustwidth}
\}
{\normalfont\bfseries\,\par}%
}
{%
\{\ensuremath{##2}\}%
{\normalfont\bfseries\,\par}%
}
}

\newcommand{\asegura}[2][]{%
{\normalfont\bfseries\ttfamily asegura\;}%
\ifthenelse{\equal{##1}{variaslineas}}{\{%
\begin{adjustwidth}{+5em}{}
	\ensuremath{##2}
\end{adjustwidth}
\}
{\normalfont\bfseries\,\par}%
}
{%
\{\ensuremath{##2}\}%
{\normalfont\bfseries\,\par}%
}
}
\renewcommand{\aux}[4]{%
	{\normalfont\bfseries\ttfamily aux\ }%
		{\normalfont\ttfamily ##1}%
	\ifthenelse{\equal{##2}{}}{}{\ (##2)}\ : ##3\, = \ensuremath{##4}%
	{\normalfont\bfseries\,;\par}%
}
\renewcommand{\pred}[3]{%
{\normalfont\bfseries\ttfamily pred }%
	{\normalfont\ttfamily ##1}%
\ifthenelse{\equal{##2}{}}{}{\ (##2) }%
\{%
\begin{adjustwidth}{+5em}{}
	\ensuremath{##3}
\end{adjustwidth}
\}%
{\normalfont\bfseries\,\par}%
}

\newcommand{\res}{#1}
\vspace{1ex}
\noindent
\encabezadoDeProc{#1}{#2}{#3}{#4}
% Abrimos la llave
\par%
\tocarEspacios
}
{
% Cerramos la llave
\vspace{1ex}
}

\newcommand{\aux}[4]{%
	{\normalfont\bfseries\ttfamily\noindent aux\ }%
		{\normalfont\ttfamily #1}%
	\ifthenelse{\equal{#2}{}}{}{\ (#2)}\ : #3\, = \ensuremath{#4}%
	{\normalfont\bfseries\,;\par}%
}

\newcommand{\pred}[3]{%
{\normalfont\bfseries\ttfamily\noindent pred }%
	{\normalfont\ttfamily #1}%
\ifthenelse{\equal{#2}{}}{}{\ (#2) }%
\{%
\begin{adjustwidth}{+2em}{}
	\ensuremath{#3}
\end{adjustwidth}
\}%
{\normalfont\bfseries\,\par}%
}

% Tipos

\newcommand{\nat}{\ensuremath{\mathbb{N}}}
\newcommand{\reals}{\ensuremath{\mathbb{R}}}
\newcommand{\ent}{\ensuremath{\mathds{Z}}}
\newcommand{\float}{\ensuremath{\mathds{R}}}
\newcommand{\bool}{\ensuremath{\mathsf{Bool}}}
\newcommand{\cha}{\ensuremath{\mathsf{Char}}}
\newcommand{\str}{\ensuremath{\mathsf{String}}}
\newcommand{\dict}[1]{\ensuremath{\mathsf{dict}\lrangle{#1}}}
\newcommand{\conj}[1]{\ensuremath{\mathsf{conj}\lrangle{#1}}}
\newcommand{\tupla}[1]{\ensuremath{\mathsf{tupla}\lrangle{#1}}}
\newcommand{\struct}[1]{\ensuremath{\mathsf{struct}\lrangle{#1}}}

% Logica

\newcommand{\True}{\ensuremath{\mathrm{true}}}
\newcommand{\False}{\ensuremath{\mathrm{false}}}
\newcommand{\Then}{\ensuremath{\rightarrow}}
\newcommand{\Iff}{\ensuremath{\leftrightarrow}}
\newcommand{\implica}{\ensuremath{\longrightarrow}}
\newcommand{\IfThenElse}[3]{\ensuremath{\mathsf{if}\ #1\ \mathsf{then}\ #2\ \mathsf{else}\ #3\ \mathsf{fi}}}
\newcommand{\If}[3]{\text{\normalfont\ttfamily IfThenElse}(\ensuremath{#1,\;#2,\;#3})}
\newcommand{\yLuego}{\land _L}
\newcommand{\oLuego}{\lor _L}
\newcommand{\thenLuego}{\Then _L}

\newcommand{\cuantificador}[5]{%
	\ensuremath{(#2 #3: #4)\ (%
		\ifthenelse{\equal{#1}{multLineas}}{
			$ % exiting math mode
				\begin{adjustwidth}{+2em}{}
					$#5$%
				\end{adjustwidth}%
			$ % entering math mode
		}{
			#5
		}
		)}
}

\newcommand{\existe}[4][]{%
	\cuantificador{#1}{\exists}{#2}{#3}{#4}
}
\newcommand{\paraTodo}[4][]{%
	\cuantificador{#1}{\forall}{#2}{#3}{#4}
}

\newcommand{\Def}[1]{\text{\normalfont\ttfamily def}(#1)}

%listas

\newcommand{\TLista}[1]{\ensuremath{seq \langle #1\rangle}}
\newcommand{\lvacia}{\ensuremath{\langle\,\rangle}}
\newcommand{\lv}{\ensuremath{\langle\,\rangle}}
\newcommand{\longitud}[1]{\ensuremath{|#1|}}
\newcommand{\cons}[1]{\ensuremath{\mathsf{addFirst}}(#1)}
\newcommand{\indice}[1]{\ensuremath{\mathsf{indice}}(#1)}
\newcommand{\conc}[1]{\ensuremath{\mathsf{concat}}(#1)}
\newcommand{\head}[1]{\ensuremath{\mathsf{head}}(#1)}
\newcommand{\tail}[1]{\ensuremath{\mathsf{tail}}(#1)}
\newcommand{\sub}[1]{\ensuremath{\mathsf{subseq}}(#1)}
\newcommand{\en}[1]{\ensuremath{\mathsf{en}}(#1)}
\newcommand{\cuenta}[2]{\mathsf{cuenta}\ensuremath{(#1, #2)}}
\newcommand{\suma}[1]{\mathsf{suma}(#1)}
\newcommand{\twodots}{\ensuremath{\mathrm{..}}}
\newcommand{\masmas}{\ensuremath{+\!+\;}}
\newcommand{\matriz}[1]{\TLista{\TLista{#1}}}
\newcommand{\seqchar}{\TLista{\cha}}
\newcommand{\lrangle}[1]{\ensuremath{\langle#1\rangle}}

%dict

\newcommand{\setKey}[1]{\ensuremath{\mathsf{setKey}(#1)}}
\newcommand{\delKey}[1]{\ensuremath{\mathsf{delKey}(#1)}}

\renewcommand{\wp}[2]{
	\ensuremath{\textit{wp}(\textbf{\lstinline{#1}}, #2)}
}

\newcommand{\hoare}[3]{\ensuremath{\{#1\} \; #2 \; \{#3\}}}

\renewcommand{\lstlistingname}{Código}
\lstset{% general command to set parameter(s)
	language=Java,
	morekeywords={func, endif, endwhile, skip, end, var, then},
	basewidth={0.47em,0.40em},
	columns=fixed, fontadjust, resetmargins, xrightmargin=5pt, xleftmargin=15pt,
	flexiblecolumns=false, tabsize=4, breaklines, breakatwhitespace=false, extendedchars=true,
	numbers=left, numberstyle=\tiny, stepnumber=1, numbersep=9pt,
	frame=l, framesep=3pt,
	captionpos=b,
}

% TADs

\newenvironment{tad}[2]{
\newcommand{\tadtype}{%
	\ensuremath{#1 \ifblank{#2}{}{\lrangle{#2}}}
}

\renewcommand{\pred}[3]{%
{\normalfont\bfseries\ttfamily pred }%
	{\normalfont\ttfamily ##1}%
\ifthenelse{\equal{##2}{}}{}{\ (##2) }%
\{%
\begin{adjustwidth}{+5em}{}
	\ensuremath{##3}
\end{adjustwidth}
\}
{\normalfont\bfseries\,\par}%
}

\vspace{1ex}
\noindent
{\normalfont\bfseries\ttfamily\large TAD #1\ifthenelse{\equal{#2}{}}{}{$<$#2$>$}} \{
% Abrimos la llave
\par%
\tocarEspacios
}{
\vspace{1ex} \par\}
}

\newcommand{\obs}[2]{
	obs #1: \ensuremath{#2}\par
}

\newenvironment{module}[4]{
\newcommand{\tadtype}{%
	\ensuremath{#3 \ifblank{#4}{}{\lrangle{#4}}}
}

\newcommand{\moduletype}{%
	\ensuremath{#1 \ifblank{#2}{}{\lrangle{#2}}}
}


\renewcommand{\pred}[3]{%
{\normalfont\bfseries\ttfamily pred }%
	{\normalfont\ttfamily ##1}%
\ifthenelse{\equal{##2}{}}{}{\ (##2) }%
\{%
\begin{adjustwidth}{+5em}{}
	\ensuremath{##3}
\end{adjustwidth}
\}
{\normalfont\bfseries\,\par}%
}

\vspace{1ex}
\noindent
{\normalfont\bfseries\ttfamily\large Módulo #1\ifthenelse{\equal{#2}{}}{}{$<$#2$>$} implementa #3\ifthenelse{\equal{#4}{}}{}{$<$#4$>$}} \{
% Abrimos la llave
\par%
\tocarEspacios
}{
\vspace{1ex} \par\}
}

\newcommand{\var}[2]{
	var #1: \;\ensuremath{#2}\par
}
\newcommand{\compl}[1]{Complejidad: $#1$}

% Tipos de datos de implementación
\newcommand{\Int}{\ensuremath{\mathsf{int}}}
\newcommand{\Real}{\ensuremath{\mathsf{real}}}
\newcommand{\Bool}{\ensuremath{\mathsf{bool}}}
\newcommand{\Char}{\ensuremath{\mathsf{char}}}
\newcommand{\String}{\ensuremath{\mathsf{string}}}
\newcommand{\Array}[1]{\ensuremath{\mathsf{Array<#1>}}}

\begin{document}

\titulo{Guia 9}
\materia{Algoritmos y Estructuras de Datos I}
\fecha{2do cuatrimestre 2024}

\integrante{Federico Barberón}{112/24}{jfedericobarberonj@gmail.com}
%Carátula
\maketitle
\newpage

%Indice
\tableofcontents
\newpage

\section{Guia 9}

\subsection{Ejercicio 1}
Comparar la complejidad de de los algoritmos de ordenamiento dados en la teórica para el caso en que el arreglo a ordenar se encuentre perfectamente ordenado de manera inversa a la deseada.

\hacer
% \begin{center}
%     \begin{tabular}{|c|c|l}
%         \hline
%         \textbf{Algoritmo} & \textbf{Complejidad} \\
%         \hline
%         Selection          & $O(n^2)$             \\
%         \hline
%         Insertion          & $O(n^2)$             \\
%         \hline
%         Merge              & $O(n^2)$             \\
%         \hline
%         Quick              & $O(n^2)$             \\
%         \hline
%     \end{tabular}
% \end{center}

\subsection{Ejercicio 2}
Defina la propiedad de estabilidad en un algoritmo de ordenamiento. Explique por qué el algoritmo de \textit{heapSort} no es estable

La estabilidad hace referencia a la propiedad de los algoritmos de ordenamiento de mantener el orden relativo de los elementos de igual comparación del arreglo original. En el caso del algoritmo \textit{heapSort}, esta propiedad no se cumple ya que al momento de corregir los elementos que estan mal posicionados en el heap, se puede potencialmente perder el orden relativo original de los elementos.

\subsection{Ejercicio 3}
Imagine secuencias de naturales de la forma $s = Concatenar(s', s'')$, donde $s'$ es una secuencia ordenada de naturales y $s''$ es una secuencia de naturales elegidos al azar. ¿Qué método utilizaría para ordenar $s$? Justificar. (Entiéndase que $s'$ se encuentra ordenada de la manera deseada)

\begin{itemize}
    \item Primero identifico el indice donde termina la secuencia ordenada $s'$ recorriendo $s$ hasta encontrar un elemento desordenado // $O(|s|)$
    \item Luego separo $s$ en dos subarrays $s'$ y $s''$ en base al indice encontrado // $O(|s|)$
    \item Uso MergeSort para ordenar el subarray $s''$ // $O(|s''|\log(|s''|))$
    \item Mergeo el subarray $s'$ con el subarray ya ordenado $s''$ // $O(|s|)$
\end{itemize}

Complejidad final: $O(|s| + |s| + |s''|\log(|s''|) + |s|) = O(|s| + |s''|\log(|s''|))$

\subsection{Ejercicio 4}
Escribir un algoritmo que encuentre los $k$ elementos más chicos de un arreglo de dimensión $n$, donde $k \leq n$. ¿Cuál es su complejidad temporal? ¿A partir de qué valor de $k$ es ventajoso ordenar el arreglo entero primero?

\begin{proc}{obtenerKminimos}{\In A: \Array{\Int}, \In k: \Int}{\Array{\Int}}
    \compl{O(nk)}
    (Usando un maxHeap creo que se podía hacer en $O(n\log(k))$)
    \begin{lstlisting}[numbers=none,frame=none]
    var minimos := new Array<int> (k);
    var min: int;

    for i := 0 to k do
        min := i;
        for j := i + 1 to A.size() do
            if A[j] < A[min] then
                min := j
            endif
        endfor

        minimos[i] := A[min];
        swap(A, i, min);
    endfor

    return minimos;
    \end{lstlisting}
\end{proc}

A partir de valores de $k > \log(n)$ tiene sentido ordenar el arreglo primero.

\subsection{Ejercicio 5}
Se tiene un conjunto de $n$ secuencias $\{s_1, s_2, \ldots, s_n\}$ en donde cada $s_i (i \leq i \leq n)$ es una secuencia ordenada de naturales. ¿Qué método utilizaría para obtener un arreglo que contenga todos los elementos de la unión de los $s_i$ ordenados? Describirlo. Justificar.

Este problema es similar a la segunda parte del algoritmo de MergeSort, ya que tenemos $n$ subarrays ordenados y lo único que falta hacer ahora es mergearlos. Por lo que una forma de resolverlo es recorrer los subarrays de a dos, mergando uno con el siguiente, y volver a repetir el proceso con los subarrays ya mergeados hasta que quede un solo subarray, que va a tener todos los elementos ordenados y esa va a ser la respuesta. En caso de que $n$ sea impar, se puede dejar el último subarray intacto hasta el final, donde será mergeado en el último paso.

Mergear todos los arrays en cada paso tiene una complejidad de $O(\sum_{i=1}^{n} |s_i|)$ y la cantidad de veces que tengo que mergear todos los arrays es $\log(n)$, por lo tanto la complejidad del algortimo es $O(\log(n) \cdot \sum_{i=1}^{n} |s_i|)$

Ejemplo:

{
\large
\[[\;\underbrace{[1,2,3], [5,6]}_\text{Merge}, \underbrace{[1,8,9], [2,3,7]}_\text{Merge}\;]\]
\[[\;\underbrace{[1,2,4,5,6], [1,2,3,7,8,9]}_\text{Merge}\;]\]
\[[\;[1,1,2,2,3,4,5,6,7,8,9]\;]\]
}

\subsection{Ejercicio 6}
Se tiene un arreglo de $n$ números naturales que se quiere ordenar por frecuencia, y en caso de igual frecuencia, por su valor. Por ejemplo, a partir del arreglo $[1,3,1,7,2,7,1,7,3]$ se quiere obtener $[1,1,1,7,7,7,3,3,2]$. Describa un algortimo que relice el ordenamiento descripto, utilizando las estructuras de datos intermedias que considere necesarias. Calcule el orden de complejidad temporal del algoritmo propuesto.

Para resolver este problema vamos a utilizar la metodología \textit{radix} para ordenar por varios criterios. En este caso, el criterio principal es la frecuencia y el secundario es el valor.

\begin{itemize}
    \item Ordenamos primero los numeros por valor usando MergeSort // $O(n\log(n))$
    \item Creamos una lista de tuplas donde la primer componente es un número del arreglo original y la segunda es la cantidad de apariciones de ese número. Al tener el arreglo ordenado, podemos crear este nuevo en tiempo lineal // $O(n)$
    \item Pasamos esa lista a un array y lo ordenamos usando MergeSort por la segunda componente en orden descendente // $O(n\log(n))$
    \item Recorremos este último array e insertamos en un nuevo array (o sobreescribimos el original) cada elemento tantas veces como repeticiones tenga // $O(n)$
\end{itemize}

Complejidad final: $O(n\log(n) + n + n\log(n) + n) = O(n\log(n))$

\pagebreak
\begin{proc}{ordenarPorFrecuenciaYValor}{\Inout A: \Array{\Int}}{}
    \begin{lstlisting}[numbers=none,frame=none]
    MergeSort(A); // O(nlog(n))
    
    var repeticiones := new ListaEnlazada< Tupla<int, int> >();
    
    for i := 0 to A.size() do // O(n)
        if repeticiones.size() != 0 && repeticiones.ultimo()[0] == A[i] then // O(1)
            repeticiones.ultimo()[1]++; // O(1)
        else
            repeticiones.agregarAtras(new Tupla(A[i], 1)); // O(1)
        endif
    endfor
    
    var arrRepeticiones = List2Array(repeticiones); // O(n)
    MergeSort(arrRepeticiones); // por segunda componente en orden descendente; O(nlog(n))
    
    var j := 0;
    
    for i := 0 to A.size() do // O(n)
        if arrRepeticiones[j][1] == 1 then
            j++;
        else
            arrRepeticiones[j][1]--;
        endif
    
        A[i] := arrRepeticiones[j][0];
    endfor
    \end{lstlisting}
\end{proc}

\subsection{Ejercicio 7}
Sea $A[1\ldots n]$ un arreglo que contine $n$ números naturales. Diremos que un rango de posiciones $[i\ldots j]$, con $i \leq i \leq j \leq n$, contiene una escalera en $A$ si valen las siguientes dos propiedades:
\begin{enumerate}
    \item \paraTodo{k}{\nat}{i \leq j < j \Then A[k + 1] = A[k] + 1} (esto es, los elementos no sólo están ordenados en forma creciente, sino que además el siguiente vale exactamente uno más que el anterior)..
    \item Si $1 < i$ entonces $A[i] \neq A[i - 1] + 1$ y si $j < n$ entonces $A[j + 1] \neq A[j] + 1$ (la propiedad es \textit{maximal}, es decir que el rango no puede extenderse sin que deje de ser una escalera según el punto anterior).
\end{enumerate}

Se puede verificar fácilmente que cuaqluier arreglo puede ser descompuesto de manera única como una secuencia de escaleras. Se pide escribir un algortimo para reposicionar las escaleras del arreglo original, de modo que las mismas se presenten en orden decreciente de longitud y, para las de la misma longitud, se presenten ordenadas en forma creciente por el primer valor de la escalera.

El resultado debe ser del mismo tipo de datos que el arreglo original. Calcule la complejidad temporal de la solución propuedas, y justifique dicho cálculo.

Por ejemplo, el siguiente arreglo

    {\large\[[5,6,8,9,10,7,8,9,20,15]\]}

debería ser transofrmado a

    {\large\[[7,8,9,8,9,10,5,6,15,20]\]}

\pagebreak

\begin{itemize}
    \item Construimos un arreglo de arreglos donde cada elemento es una escalera del arreglo original. // $O(n)$
    \item Ordenamos este arreglo por el primer elemento de cada escalera en forma creciente con MergeSort. // $O(n\log(n))$
    \item Ordenamos el arreglo por la longitud de cada escalera en forma decreciente con MergeSort. // $O(n\log(n))$
    \item Aplanamos el arreglo de escaleras. // $O(n)$
\end{itemize}
Complejidad final: $O(n\log(n))$

\begin{proc}{ordenarEscaleras}{\Inout A: \Array{\Int}}{}
    \compl{O(n\log(n))}
    \begin{lstlisting}[numbers=none,frame=none]
    var escaleras: ListaEnlazada <ListaEnlazada< int > > := new ListaEnlazada();
    var escaleraActual: ListaEnlazada< int > := new ListaEnlazada();

    for i := 0 to A.size() do // O(n)
        if i == 0 ||  A[i] == A[i-1] + 1 then
            escaleraActual.agregarAtras(A[i]); // O(1)
        else
            escaleras.agregarAtras(escaleraActual); // O(1)
            escaleraActual := new ListaEnlazada< int >();
            escaleraActual.agregarAtras(A[i]); // O(1)
        endif
    endfor

    escaleras.agregarAtras(escaleraActual);  // O(1)

    var arrEscaleras: Array< ListaEnlazada< int > > := List2Array(escaleras); // O(n)

    MergeSort(arrEscaleras); // por primer elemento de manera creciente; O(nlog(n))
    MergeSort(arrEscaleras); // por longitud de manera decreciente; O(nlog(n))

    var i: int := 0;

    for j := 0 to arrEscaleras.size() do // O(n) en total
        var it: IteradorBidireccional := arrEscaleras[j].iterador();

        while it.haySiguiente() do
            A[i] := it.siguiente();
            i++;
        endwhile
    endfor
    \end{lstlisting}
\end{proc}
\end{document}

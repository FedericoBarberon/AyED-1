\documentclass[11pt, a4paper]{article}
%

\usepackage{../caratuladc/caratula/caratula}

\usepackage{multicol}
\usepackage{enumitem}
\usepackage{xcolor}
\usepackage{etoolbox}

\newcommand{\hacer}{\textcolor{red}{HACER!}}
\newcommand{\punto}{\text{.}}
\usepackage[spanish,activeacute,es-tabla]{babel}
\usepackage[utf8]{inputenc}
\usepackage{ifthen}
\usepackage{listings}
\usepackage{dsfont}
\usepackage{subcaption}
\usepackage{amsmath}
\usepackage[top=1cm,bottom=2cm,left=1cm,right=1cm]{geometry}%
\usepackage{color}%
\usepackage{changepage}
\newcommand{\tocarEspacios}{%
	\addtolength{\leftskip}{3em}%
	\setlength{\parindent}{0em}%
}

\newcommand{\Indent}{\hspace*{0.75cm}}

% Especificacion de procs

\newcommand{\In}{\textsf{in }}
\newcommand{\Out}{\textsf{out }}
\newcommand{\Inout}{\textsf{inout }}

\newcommand{\encabezadoDeProc}[4]{%
% Ponemos la palabrita problema en tt
%  \noindent%
{\normalfont\bfseries\ttfamily proc}%
% Ponemos el nombre del problema
\ %
{\normalfont\ttfamily #2}%
\
% Ponemos los parametros
(#3)%
\ifblank{#4}{}{%
	% Por ultimo, va el tipo del resultado
	\ : #4}
}

\newenvironment{proc}[4][res]{%

% El parametro 1 (opcional) es el nombre del resultado
% El parametro 2 es el nombre del problema
% El parametro 3 son los parametros
% El parametro 4 es el tipo del resultado
% Preambulo del ambiente problema
% Tenemos que definir los comandos requiere, asegura, modifica y aux
\newcommand{\requiere}[2][]{%
{\normalfont\bfseries\ttfamily requiere\;}%
\ifthenelse{\equal{##1}{variaslineas}}{\{%
\begin{adjustwidth}{+5em}{}
	\ensuremath{##2}
\end{adjustwidth}
\}
{\normalfont\bfseries\,\par}%
}
{%
\{\ensuremath{##2}\}%
{\normalfont\bfseries\,\par}%
}
}

\newcommand{\asegura}[2][]{%
{\normalfont\bfseries\ttfamily asegura\;}%
\ifthenelse{\equal{##1}{variaslineas}}{\{%
\begin{adjustwidth}{+5em}{}
	\ensuremath{##2}
\end{adjustwidth}
\}
{\normalfont\bfseries\,\par}%
}
{%
\{\ensuremath{##2}\}%
{\normalfont\bfseries\,\par}%
}
}
\renewcommand{\aux}[4]{%
	{\normalfont\bfseries\ttfamily aux\ }%
		{\normalfont\ttfamily ##1}%
	\ifthenelse{\equal{##2}{}}{}{\ (##2)}\ : ##3\, = \ensuremath{##4}%
	{\normalfont\bfseries\,;\par}%
}
\renewcommand{\pred}[3]{%
{\normalfont\bfseries\ttfamily pred }%
	{\normalfont\ttfamily ##1}%
\ifthenelse{\equal{##2}{}}{}{\ (##2) }%
\{%
\begin{adjustwidth}{+5em}{}
	\ensuremath{##3}
\end{adjustwidth}
\}%
{\normalfont\bfseries\,\par}%
}

\newcommand{\res}{#1}
\vspace{1ex}
\noindent
\encabezadoDeProc{#1}{#2}{#3}{#4}
% Abrimos la llave
\par%
\tocarEspacios
}
{
% Cerramos la llave
\vspace{1ex}
}

\newcommand{\aux}[4]{%
	{\normalfont\bfseries\ttfamily\noindent aux\ }%
		{\normalfont\ttfamily #1}%
	\ifthenelse{\equal{#2}{}}{}{\ (#2)}\ : #3\, = \ensuremath{#4}%
	{\normalfont\bfseries\,;\par}%
}

\newcommand{\pred}[3]{%
{\normalfont\bfseries\ttfamily\noindent pred }%
	{\normalfont\ttfamily #1}%
\ifthenelse{\equal{#2}{}}{}{\ (#2) }%
\{%
\begin{adjustwidth}{+2em}{}
	\ensuremath{#3}
\end{adjustwidth}
\}%
{\normalfont\bfseries\,\par}%
}

% Tipos

\newcommand{\nat}{\ensuremath{\mathbb{N}}}
\newcommand{\reals}{\ensuremath{\mathbb{R}}}
\newcommand{\ent}{\ensuremath{\mathds{Z}}}
\newcommand{\float}{\ensuremath{\mathds{R}}}
\newcommand{\bool}{\ensuremath{\mathsf{Bool}}}
\newcommand{\cha}{\ensuremath{\mathsf{Char}}}
\newcommand{\str}{\ensuremath{\mathsf{String}}}
\newcommand{\dict}[1]{\ensuremath{\mathsf{dict}\lrangle{#1}}}
\newcommand{\conj}[1]{\ensuremath{\mathsf{conj}\lrangle{#1}}}
\newcommand{\tupla}[1]{\ensuremath{\mathsf{tupla}\lrangle{#1}}}
\newcommand{\struct}[1]{\ensuremath{\mathsf{struct}\lrangle{#1}}}

% Logica

\newcommand{\True}{\ensuremath{\mathrm{true}}}
\newcommand{\False}{\ensuremath{\mathrm{false}}}
\newcommand{\Then}{\ensuremath{\rightarrow}}
\newcommand{\Iff}{\ensuremath{\leftrightarrow}}
\newcommand{\implica}{\ensuremath{\longrightarrow}}
\newcommand{\IfThenElse}[3]{\ensuremath{\mathsf{if}\ #1\ \mathsf{then}\ #2\ \mathsf{else}\ #3\ \mathsf{fi}}}
\newcommand{\If}[3]{\text{\normalfont\ttfamily IfThenElse}(\ensuremath{#1,\;#2,\;#3})}
\newcommand{\yLuego}{\land _L}
\newcommand{\oLuego}{\lor _L}
\newcommand{\thenLuego}{\Then _L}

\newcommand{\cuantificador}[5]{%
	\ensuremath{(#2 #3: #4)\ (%
		\ifthenelse{\equal{#1}{multLineas}}{
			$ % exiting math mode
				\begin{adjustwidth}{+2em}{}
					$#5$%
				\end{adjustwidth}%
			$ % entering math mode
		}{
			#5
		}
		)}
}

\newcommand{\existe}[4][]{%
	\cuantificador{#1}{\exists}{#2}{#3}{#4}
}
\newcommand{\paraTodo}[4][]{%
	\cuantificador{#1}{\forall}{#2}{#3}{#4}
}

\newcommand{\Def}[1]{\text{\normalfont\ttfamily def}(#1)}

%listas

\newcommand{\TLista}[1]{\ensuremath{seq \langle #1\rangle}}
\newcommand{\lvacia}{\ensuremath{\langle\,\rangle}}
\newcommand{\lv}{\ensuremath{\langle\,\rangle}}
\newcommand{\longitud}[1]{\ensuremath{|#1|}}
\newcommand{\cons}[1]{\ensuremath{\mathsf{addFirst}}(#1)}
\newcommand{\indice}[1]{\ensuremath{\mathsf{indice}}(#1)}
\newcommand{\conc}[1]{\ensuremath{\mathsf{concat}}(#1)}
\newcommand{\head}[1]{\ensuremath{\mathsf{head}}(#1)}
\newcommand{\tail}[1]{\ensuremath{\mathsf{tail}}(#1)}
\newcommand{\sub}[1]{\ensuremath{\mathsf{subseq}}(#1)}
\newcommand{\en}[1]{\ensuremath{\mathsf{en}}(#1)}
\newcommand{\cuenta}[2]{\mathsf{cuenta}\ensuremath{(#1, #2)}}
\newcommand{\suma}[1]{\mathsf{suma}(#1)}
\newcommand{\twodots}{\ensuremath{\mathrm{..}}}
\newcommand{\masmas}{\ensuremath{+\!+\;}}
\newcommand{\matriz}[1]{\TLista{\TLista{#1}}}
\newcommand{\seqchar}{\TLista{\cha}}
\newcommand{\lrangle}[1]{\ensuremath{\langle#1\rangle}}

%dict

\newcommand{\setKey}[1]{\ensuremath{\mathsf{setKey}(#1)}}
\newcommand{\delKey}[1]{\ensuremath{\mathsf{delKey}(#1)}}

\renewcommand{\wp}[2]{
	\ensuremath{\textit{wp}(\textbf{\lstinline{#1}}, #2)}
}

\newcommand{\hoare}[3]{\ensuremath{\{#1\} \; #2 \; \{#3\}}}

\renewcommand{\lstlistingname}{Código}
\lstset{% general command to set parameter(s)
	language=Java,
	morekeywords={func, endif, endwhile, skip, end, var, then},
	basewidth={0.47em,0.40em},
	columns=fixed, fontadjust, resetmargins, xrightmargin=5pt, xleftmargin=15pt,
	flexiblecolumns=false, tabsize=4, breaklines, breakatwhitespace=false, extendedchars=true,
	numbers=left, numberstyle=\tiny, stepnumber=1, numbersep=9pt,
	frame=l, framesep=3pt,
	captionpos=b,
}

% TADs

\newenvironment{tad}[2]{
\newcommand{\tadtype}{%
	\ensuremath{#1 \ifblank{#2}{}{\lrangle{#2}}}
}

\renewcommand{\pred}[3]{%
{\normalfont\bfseries\ttfamily pred }%
	{\normalfont\ttfamily ##1}%
\ifthenelse{\equal{##2}{}}{}{\ (##2) }%
\{%
\begin{adjustwidth}{+5em}{}
	\ensuremath{##3}
\end{adjustwidth}
\}
{\normalfont\bfseries\,\par}%
}

\vspace{1ex}
\noindent
{\normalfont\bfseries\ttfamily\large TAD #1\ifthenelse{\equal{#2}{}}{}{$<$#2$>$}} \{
% Abrimos la llave
\par%
\tocarEspacios
}{
\vspace{1ex} \par\}
}

\newcommand{\obs}[2]{
	obs #1: \ensuremath{#2}\par
}

\newenvironment{module}[4]{
\newcommand{\tadtype}{%
	\ensuremath{#3 \ifblank{#4}{}{\lrangle{#4}}}
}

\newcommand{\moduletype}{%
	\ensuremath{#1 \ifblank{#2}{}{\lrangle{#2}}}
}


\renewcommand{\pred}[3]{%
{\normalfont\bfseries\ttfamily pred }%
	{\normalfont\ttfamily ##1}%
\ifthenelse{\equal{##2}{}}{}{\ (##2) }%
\{%
\begin{adjustwidth}{+5em}{}
	\ensuremath{##3}
\end{adjustwidth}
\}
{\normalfont\bfseries\,\par}%
}

\vspace{1ex}
\noindent
{\normalfont\bfseries\ttfamily\large Módulo #1\ifthenelse{\equal{#2}{}}{}{$<$#2$>$} implementa #3\ifthenelse{\equal{#4}{}}{}{$<$#4$>$}} \{
% Abrimos la llave
\par%
\tocarEspacios
}{
\vspace{1ex} \par\}
}

\newcommand{\var}[2]{
	var #1: \;\ensuremath{#2}\par
}
\newcommand{\compl}[1]{Complejidad: $#1$}

% Tipos de datos de implementación
\newcommand{\Int}{\ensuremath{\mathsf{int}}}
\newcommand{\Real}{\ensuremath{\mathsf{real}}}
\newcommand{\Bool}{\ensuremath{\mathsf{bool}}}
\newcommand{\Char}{\ensuremath{\mathsf{char}}}
\newcommand{\String}{\ensuremath{\mathsf{string}}}
\newcommand{\Array}[1]{\ensuremath{\mathsf{Array<#1>}}}

\begin{document}

\titulo{Guia 3 - Parte 2}
\materia{Algoritmos y Estructuras de Datos I}
\fecha{2do cuatrimestre 2024}

\integrante{Federico Barberón}{112/24}{jfedericobarberonj@gmail.com}
%Carátula
\maketitle
\newpage

%Indice
\tableofcontents
\newpage

\section{Guia 3 - Parte 2}

\subsection{Ejercicio 1}
Consideremos el problema de sumar los elementos de un arreglo y la siguiente implementación en SmallLang, con el invariante del ciclo.

\begin{multicols}{2}
    \begin{minipage}{0.4\textwidth}
        \large\textbf{Especificación}

        \begin{proc}{sumar}{\In s: $array\langle \ent \rangle$}{\ent}
            \requiere{True}
            \asegura{res = \sum_{j=0}^{|s| - 1} s[j]}
        \end{proc}
    \end{minipage}

    \begin{minipage}{0.4\textwidth}
        \large\textbf{Implementación en SmallLang}

        \begin{lstlisting}[numbers=none,frame=none,xleftmargin=0pt]
res := 0;
i := 0;
while (i < s.size()) do
    res := res + s[i];
    i := i + 1;
endwhile
        \end{lstlisting}
    \end{minipage}
\end{multicols}

\large\textbf{Invariante del Ciclo}

\[I \equiv 0 \leq i \leq |s| \yLuego res = \sum_{j=0}^{i-1} s[j]\]

\begin{enumerate}[label=\alph*)]
    \item Escribir la precondición y la postcondición del ciclo.

          $
              P_c \equiv \{i = 0 \land res = 0\}\\
              Q_c \equiv \{res = \sum_{j=0}^{|s| - 1} s[j]\}
          $

    \item ¿Qué punto falla en la demostración de corrección si el primer término del invariante se reemplaza por $0 \leq i < |s|$?

          Falla el punto 3 $(I \land \neg B) \Then Qc$ pues \par
          $(0 \leq i < |s| \yLuego res = \sum_{j=0}^{i-1} s[j] \land i \geq |s|) \Then Qc \equiv False \Then Qc \equiv True$

          Lo cual está mal pues nada asegura que luego de terminar el ciclo $i = |s|$.

    \item ¿Qué punto falla en la demostración de corrección si el límite superior de la sumatoria $(i - 1)$ se reemplaza por $i$?

          Falla el punto 1 $P_c \Then I$ pues cuando vale $P_c$, $I$ pide que $res = s[0]$, lo cual se contradice con $P_c$.

    \item ¿Qué punto falla en la demostración de corrección si se invierte el orden de las dos instrucciones del cuerpo del ciclo?

          Falla el punto 2 \hoare{I \land B}{S}{I} pues luego de cada iteración $res = \sum_{j=0}^{i} s[j]$ lo cual se contradice con el invariante. Además en la última iteración $res$ se indefine pues trata de acceder a $s[|s|]$.

    \item Mostrar la correción parcial del ciclo, usando los primeros puntos del teorema del invariante.

          \begin{enumerate}[label=\arabic*)]
              \item $P_c \Then I \equiv (i = 0 \land res = 0) \Then (0 \leq i \leq |s| \yLuego res = \sum_{j=0}^{i-1} s[j]) \equiv True$

              \item \hoare{I \land B}{S}{I}

                    Para probar que se cumple hay que probar que $(I \land B) \Then \wp{S}{I}$.

                    \begin{align*}
                        \wp{S}{I}                   & \equiv \wp{res := res + s[i]; i := i + 1}{I}                                                                           \\
                                                    & \equiv \wp{res := res + s[i]}{\wp{i := i + 1}{I}}                                                                      \\
                                                    & \equiv \wp{res := res + s[i]}{0 \leq i + 1 \leq |s| \yLuego res = \sum_{j=0}^{i} s[j]}                                 \\
                                                    & \equiv 0 \leq i < |s| \yLuego res + s[i] = \sum_{j=0}^{i} s[j]                                                         \\
                                                    & \equiv 0 \leq i < |s| \yLuego res = \sum_{j=0}^{i-1} s[j]                                                              \\\\
                        (I \land B) \Then \wp{S}{I} & \equiv (0 \leq i < |s| \yLuego res = \sum_{j=0}^{i-1} s[j]) \Then (0 \leq i < |s| \yLuego res = \sum_{j=0}^{i-1} s[j]) \\
                                                    & \equiv True
                    \end{align*}

              \item \begin{align*}
                        (I \land \neg B) \Then Q_c & \equiv ((0 \leq i \leq |s| \yLuego res = \sum_{j=0}^{i-1} s[j]) \land i \geq |s|) \Then res = \sum_{j=0}^{|s| - 1} s[j] \\
                                                   & \equiv (i = |s| \yLuego res = \sum_{j=0}^{i-1} s[j]) \Then res = \sum_{j=0}^{|s| - 1} s[j]                              \\
                                                   & \equiv res = \sum_{j=0}^{|s| - 1} s[j] \Then res = \sum_{j=0}^{|s| - 1} s[j]                                            \\
                                                   & \equiv True
                    \end{align*}
          \end{enumerate}

    \item Proponer una función variante y mostrar la terminación del ciclo, utilizando la función variante.

          Propongo la función variante $f_v = |s| - i$. Para probarla tengo que probar lo siguiente:

          \begin{itemize}
              \item $\hoare{I \land B \land f_v = v_0}{S}{f_v < v_0} \Iff (I \land B \land f_v = v_0) \Then \wp{S}{f_v < v_0}$

                    \begin{align*}
                        \wp{S}{f_v < v_0} & \equiv \wp{res := res + s[i]}{\wp{i := i + 1}{|s| - i < v_0}} \\
                                          & \equiv \wp{res := res + s[i]}{|s| - i - 1 < v_0}              \\
                                          & \equiv 0 \leq i < |s| \yLuego |s| - i - 1 < v_0
                    \end{align*}

                    $
                        (I \land B \land f_v = v_0) \Then \wp{S}{f_v < v_0} \\
                        \equiv (0 \leq i \leq |s| \yLuego res = \sum_{j=0}^{i-1} s[j]) \land i < |s| \land |s| - i = v_0 \Then (0 \leq i < |s| \yLuego |s| - i - 1 < v_0) \\
                        \equiv (0 \leq i < |s| \yLuego res = \sum_{j=0}^{i-1} s[j]) \land |s| - i = v_0 \Then (0 \leq i < |s| \yLuego |s| - i - 1 < |s| - i) \\
                        \equiv True
                    $

              \item \begin{align*}
                        I \land f_v \leq 0 \Then \neg B & \equiv (0 \leq i \leq |s| \yLuego res = \sum_{j=0}^{i-1} s[j]) \land |s| - i \leq 0 \Then i \geq |s| \\
                                                        & \equiv i = |s| \yLuego res = \sum_{j=0}^{i-1} s[j] \Then i \geq |s|                                  \\
                                                        & \equiv True
                    \end{align*}
          \end{itemize}

          De esta manera queda demostrado que el ciclo termina en una cantidad finita de iteraciones y que es correcto.
\end{enumerate}

\subsection{Ejercicio 2}
Dadas la especificación y la implementación del problema sumarParesHastaN

\begin{multicols}{2}
    \begin{minipage}{0.4\textwidth}
        \large\textbf{Especificación}

        \begin{proc}{sumarParesHastaN}{\In n: \ent}{\ent}
            \requiere{n \geq 0}
            \asegura{res = \sum_{j=0}^{n-1} \\ \If{j \mod 2 = 0}{j}{0}}
        \end{proc}
    \end{minipage}

    \begin{minipage}{0.4\textwidth}
        \large\textbf{Implementación en SmallLang}

        \begin{lstlisting}[numbers=none,frame=none,xleftmargin=0pt]
res := 0;
i := 0;
while (i < n) do
    res := res + i;
    i := i + 2;
endwhile
        \end{lstlisting}
    \end{minipage}
\end{multicols}

\large\textbf{Invariante del Ciclo}

\[I \equiv 0 \leq i \leq n + 1 \land i \mod 2 = 0 \land res = \sum_{j=0}^{i-1} \If{j \mod 2 = 0}{j}{0}\]

\begin{enumerate}[label=\alph*)]
    \item Escribir la precondición y la poscondición del ciclo.
    \item Mostrar la correción parcial del ciclo, usando los primeros puntos el teorema del invariante.
    \item Proponer una función variante y mostrar la terminación del ciclo, utilizando la función variante
\end{enumerate}

\begin{enumerate}[label=\alph*)]
    \item \begin{align*}
              P_c & \equiv \{res = 0 \land i = 0\}                            \\
              Q_c & \equiv \{res = \sum_{j=0}^{n-1} \If{j \mod 2 = 0}{j}{0}\}
          \end{align*}

    \item Para probar la correctitud parcial del programa tengo que probar los siguientes items:

          $P_c \Then I$:

          Con $res = 0 \land i = 0$ se tiene que
          %
          \begin{align*}
              I & \equiv 0 \leq 0 \leq n + 1 \land 0 \mod 2 = 0 \land 0 = \sum_{j=0}^{-1}\If{j \mod 2 = 0}{j}{0} \\
                & \equiv True
          \end{align*}

          Por lo tanto $P_c \Then I$ es verdadero.
          \bigskip

          \hoare{I \land B}{S}{I}

          Para probar esto tengo que ver si $(I \land B) \Then \wp{S}{I}$
          %
          \begin{align*}
              \wp{S}{I} & \equiv \wp{res := res + i}{\wp{i := i + 2}{I}}                                                                 \\
                        & \equiv \wp{res := res + i}{0 \leq i + 2 \leq n + 1 \land (i+2) \mod 2 = 0 \land res = \sum_{j=0}^{i+1} \ldots} \\
                        & \equiv -2 \leq i \leq n - 1 \land i \mod 2 = 0 \land res + i = \sum_{j=0}^{i+1} \If{j \mod 2 = 0}{j}{0}        \\
                        & \equiv -2 \leq i \leq n - 1 \land i \mod 2 = 0 \land res = \sum_{j=0}^{i-2} \If{j \mod 2 = 0}{j}{0}
          \end{align*}

          Ahora quiero ver si $(I \land B) \Then \wp{S}{I}$:
          %
          \begin{align*}
              I \land B & \equiv (0 \leq i \leq n + 1 \land i \mod 2 = 0 \land res = \sum_{j=0}^{i-1} \If{j \mod 2 = 0}{j}{0}) \land i < n \\
                        & \equiv 0 \leq i < n \land i \mod 2 = 0 \land res = \sum_{j=0}^{i-1} \If{j \mod 2 = 0}{j}{0}
          \end{align*}

          y como $(0 \leq i < n) \Then( -2 \leq i \leq n - 1)$ y \par
          $\sum_{j=0}^{i-2} \If{j \mod 2 = 0}{j}{0} = \sum_{j=0}^{i-1} \If{j \mod 2 = 0}{j}{0}$ pues $i$ es par, entonces se cumple la implicación.
          \bigskip

          $(I \land \neg B) \Then Q_c$:

          Se tiene que
          %
          \begin{align*}
              I \land \neg B & \equiv (0 \leq i \leq n + 1 \land i \mod 2 = 0 \land res = \sum_{j=0}^{i-1} \If{j \mod 2 = 0}{j}{0}) \land i \geq n \\
                             & \equiv n \leq i \leq n + 1 \land i \mod 2 = 0 \land res = \sum_{j=0}^{i-1} \If{j \mod 2 = 0}{j}{0}                  \\
                             & \equiv (n \mod 2 = 0 \Then res = \sum_{j=0}^{n-1} \If{j \mod 2 = 0}{j}{0}) \land                                    \\
                             & \mathrel{\hphantom{\equiv}} (n \mod 2 \neq 0 \Then res = \sum_{j=0}^{n} \If{j \mod 2 = 0}{j}{0})                    \\
                             & \equiv res = \sum_{j=0}^{n} \If{j \mod 2 = 0}{j}{0}
          \end{align*}

          que es exactamente igual a $Q_c$
          \medskip

          De esta manera queda probada la correctitud parcial del programa.

    \item Propongo la función variante $f_v = n - i$ y la verifico con el teorema de terminación.

          \hoare{I \land B \land f_v = v_0}{S}{f_v < v_0}

          Para probar eso tengo que probar que $(I \land B \land f_v = v_0) \Then \wp{S}{f_v < v_0}$.
          %
          \begin{align*}
              \wp{S}{f_v < v_0} & \equiv \wp{res := res + i}{\wp{i := i + 2}{n - i < v_0}} \\
                                & \equiv \wp{res := res + i}{n - i - 2 < v_0}              \\
                                & \equiv n - i - 2 < v_0
          \end{align*}

          Por lo tanto
          %
          \begin{align*}
              ((0 \leq i < n \land i \mod 2 = 0 \land res = \ldots) \land n - i = v_0) \Then n - i - 2 < v_0 & \equiv n - i - 2 < n - i  \\
                                                                                                             & \equiv -2 < 0 \equiv True
          \end{align*}

          Ahora solo queda probar $(I \land f_v \leq 0) \Then \neg B$
          %
          \begin{align*}
              (I \land f_v \leq 0) \Then \neg B & \equiv ((0 \leq i \leq n + 1 \land i \mod 2 = 0 \land res = \ldots) \land n - i \leq 0) \Then i \geq n                \\
                                                & \equiv (\underline{n \leq i} \leq n + 1 \land i \mod 2 = 0 \land res = \ldots) \Then \underline{i \geq n} \equiv True
          \end{align*}

          De esta manera queda demostrado que el programa efectivamente termina y no se cuelga, por lo tanto, la implementación es correcta con respecto a la especificación.
\end{enumerate}

\end{document}

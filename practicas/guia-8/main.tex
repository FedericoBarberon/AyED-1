\documentclass[11pt, a4paper]{article}
\usepackage[paper=a4paper, left=1.5cm, right=1.5cm, bottom=1.5cm, top=1.5cm]{geometry}
%
\usepackage[utf8]{inputenc}
\usepackage[spanish]{babel}

\usepackage{../caratuladc/caratula/caratula}

\usepackage{multicol}
\usepackage{enumitem}
\usepackage{amsmath}
\usepackage{amssymb}
\usepackage{xcolor}

% Comandos %

\newcommand{\sii}{\leftrightarrow}
\newcommand{\hacer}{\textcolor{red}{HACER!}}
\usepackage[spanish,activeacute,es-tabla]{babel}
\usepackage[utf8]{inputenc}
\usepackage{ifthen}
\usepackage{listings}
\usepackage{dsfont}
\usepackage{subcaption}
\usepackage{amsmath}
\usepackage[top=1cm,bottom=2cm,left=1cm,right=1cm]{geometry}%
\usepackage{color}%
\usepackage{changepage}
\newcommand{\tocarEspacios}{%
	\addtolength{\leftskip}{3em}%
	\setlength{\parindent}{0em}%
}

\newcommand{\Indent}{\hspace*{0.75cm}}

% Especificacion de procs

\newcommand{\In}{\textsf{in }}
\newcommand{\Out}{\textsf{out }}
\newcommand{\Inout}{\textsf{inout }}

\newcommand{\encabezadoDeProc}[3]{%
% Ponemos la palabrita problema en tt
%  \noindent%
{\normalfont\bfseries\ttfamily proc}%
% Ponemos el nombre del problema
\ %
{\normalfont\ttfamily #1}%
\
% Ponemos los parametros
(#2)%
\ifblank{#3}{}{%
	% Por ultimo, va el tipo del resultado
	\ : #3}
}

\newenvironment{proc}[4][res]{%

% El parametro 1 (opcional) es el nombre del resultado
% El parametro 2 es el nombre del problema
% El parametro 3 son los parametros
% El parametro 4 es el tipo del resultado
% Preambulo del ambiente problema
% Tenemos que definir los comandos requiere, asegura, modifica y aux
\newcommand{\requiere}[2][]{%
{\normalfont\bfseries\ttfamily requiere\;}%
\ifthenelse{\equal{##1}{variaslineas}}{\{%
\begin{adjustwidth}{+5em}{}
	\ensuremath{##2}
\end{adjustwidth}
\}
{\normalfont\bfseries\,\par}%
}
{%
\{\ensuremath{##2}\}%
{\normalfont\bfseries\,\par}%
}
}

\newcommand{\asegura}[2][]{%
{\normalfont\bfseries\ttfamily asegura\;}%
\ifthenelse{\equal{##1}{variaslineas}}{\{%
\begin{adjustwidth}{+5em}{}
	\ensuremath{##2}
\end{adjustwidth}
\}
{\normalfont\bfseries\,\par}%
}
{%
\{\ensuremath{##2}\}%
{\normalfont\bfseries\,\par}%
}
}
\renewcommand{\aux}[4]{%
	{\normalfont\bfseries\ttfamily aux\ }%
		{\normalfont\ttfamily ##1}%
	\ifthenelse{\equal{##2}{}}{}{\ (##2)}\ : ##3\, = \ensuremath{##4}%
	{\normalfont\bfseries\,;\par}%
}
\renewcommand{\pred}[3]{%
{\normalfont\bfseries\ttfamily pred }%
	{\normalfont\ttfamily ##1}%
\ifthenelse{\equal{##2}{}}{}{\ (##2) }%
\{%
\begin{adjustwidth}{+5em}{}
	\ensuremath{##3}
\end{adjustwidth}
\}%
{\normalfont\bfseries\,\par}%
}

\newcommand{\res}{#1}
\vspace{1ex}
\noindent
\encabezadoDeProc{#2}{#3}{#4}
% Abrimos la llave
\par%
\tocarEspacios
}
{
% Cerramos la llave
\vspace{1ex}
}

\newcommand{\aux}[4]{%
	{\normalfont\bfseries\ttfamily\noindent aux\ }%
		{\normalfont\ttfamily #1}%
	\ifthenelse{\equal{#2}{}}{}{\ (#2)}\ : #3\, = \ensuremath{#4}%
	{\normalfont\bfseries\,;\par}%
}

\newcommand{\pred}[3]{%
{\normalfont\bfseries\ttfamily\noindent pred }%
	{\normalfont\ttfamily #1}%
\ifthenelse{\equal{#2}{}}{}{\ (#2) }%
\{%
\begin{adjustwidth}{+2em}{}
	\ensuremath{#3}
\end{adjustwidth}
\}%
{\normalfont\bfseries\,\par}%
}

% Tipos

\newcommand{\nat}{\ensuremath{\mathbb{N}}}
\newcommand{\reals}{\ensuremath{\mathbb{R}}}
\newcommand{\ent}{\ensuremath{\mathds{Z}}}
\newcommand{\float}{\ensuremath{\mathds{R}}}
\newcommand{\bool}{\ensuremath{\mathsf{Bool}}}
\newcommand{\cha}{\ensuremath{\mathsf{Char}}}
\newcommand{\str}{\ensuremath{\mathsf{String}}}
\newcommand{\dict}[1]{\ensuremath{\mathsf{dict}\lrangle{#1}}}
\newcommand{\conj}[1]{\ensuremath{\mathsf{conj}\lrangle{#1}}}
\newcommand{\tupla}[1]{\ensuremath{\mathsf{tupla}\lrangle{#1}}}
\newcommand{\struct}[1]{\ensuremath{\mathsf{struct}\lrangle{#1}}}

% Logica

\newcommand{\True}{\ensuremath{\mathrm{true}}}
\newcommand{\False}{\ensuremath{\mathrm{false}}}
\newcommand{\Then}{\ensuremath{\rightarrow}}
\newcommand{\Iff}{\ensuremath{\leftrightarrow}}
\newcommand{\implica}{\ensuremath{\longrightarrow}}
\newcommand{\IfThenElse}[3]{\ensuremath{\mathsf{if}\ #1\ \mathsf{then}\ #2\ \mathsf{else}\ #3\ \mathsf{fi}}}
\newcommand{\If}[3]{\text{\normalfont\ttfamily IfThenElse}(\ensuremath{#1,\;#2,\;#3})}
\newcommand{\yLuego}{\land _L}
\newcommand{\oLuego}{\lor _L}
\newcommand{\thenLuego}{\Then _L}

\newcommand{\cuantificador}[5]{%
	\ensuremath{(#2 #3: #4)\ (%
		\ifthenelse{\equal{#1}{multLineas}}{
			$ % exiting math mode
				\begin{adjustwidth}{+2em}{}
					$#5$%
				\end{adjustwidth}%
			$ % entering math mode
		}{
			#5
		}
		)}
}

\newcommand{\existe}[4][]{%
	\cuantificador{#1}{\exists}{#2}{#3}{#4}
}
\newcommand{\paraTodo}[4][]{%
	\cuantificador{#1}{\forall}{#2}{#3}{#4}
}

\newcommand{\Def}[1]{\text{\normalfont\ttfamily def}(#1)}

%listas

\newcommand{\TLista}[1]{\ensuremath{seq \langle #1\rangle}}
\newcommand{\lvacia}{\ensuremath{\langle\,\rangle}}
\newcommand{\lv}{\ensuremath{\langle\,\rangle}}
\newcommand{\longitud}[1]{\ensuremath{|#1|}}
\newcommand{\cons}[1]{\ensuremath{\mathsf{addFirst}}(#1)}
\newcommand{\indice}[1]{\ensuremath{\mathsf{indice}}(#1)}
\newcommand{\conc}[1]{\ensuremath{\mathsf{concat}}(#1)}
\newcommand{\head}[1]{\ensuremath{\mathsf{head}}(#1)}
\newcommand{\tail}[1]{\ensuremath{\mathsf{tail}}(#1)}
\newcommand{\sub}[1]{\ensuremath{\mathsf{subseq}}(#1)}
\newcommand{\en}[1]{\ensuremath{\mathsf{en}}(#1)}
\newcommand{\cuenta}[2]{\mathsf{cuenta}\ensuremath{(#1, #2)}}
\newcommand{\suma}[1]{\mathsf{suma}(#1)}
\newcommand{\twodots}{\ensuremath{\mathrm{..}}}
\newcommand{\masmas}{\ensuremath{+\!+\;}}
\newcommand{\matriz}[1]{\TLista{\TLista{#1}}}
\newcommand{\seqchar}{\TLista{\cha}}
\newcommand{\lrangle}[1]{\ensuremath{\langle\text{#1}\rangle}}

%dict

\newcommand{\setKey}[1]{\ensuremath{\mathsf{setKey}(#1)}}
\newcommand{\delKey}[1]{\ensuremath{\mathsf{delKey}(#1)}}

\renewcommand{\wp}[2]{
	\ensuremath{\textit{wp}(\textbf{\lstinline{#1}}, #2)}
}

\newcommand{\hoare}[3]{\ensuremath{\{#1\} \; #2 \; \{#3\}}}

\renewcommand{\lstlistingname}{Código}
\lstset{% general command to set parameter(s)
	language=Java,
	morekeywords={func, endif, endwhile, skip, end, var, then},
	basewidth={0.47em,0.40em},
	columns=fixed, fontadjust, resetmargins, xrightmargin=5pt, xleftmargin=15pt,
	flexiblecolumns=false, tabsize=4, breaklines, breakatwhitespace=false, extendedchars=true,
	numbers=left, numberstyle=\tiny, stepnumber=1, numbersep=9pt,
	frame=l, framesep=3pt,
	captionpos=b,
}

% TADs

\newenvironment{tad}[2]{
\newcommand{\tadtype}{%
	\ensuremath{#1 \ifblank{#2}{}{\lrangle{#2}}}
}

\renewcommand{\pred}[3]{%
{\normalfont\bfseries\ttfamily pred }%
	{\normalfont\ttfamily ##1}%
\ifthenelse{\equal{##2}{}}{}{\ (##2) }%
\{%
\begin{adjustwidth}{+5em}{}
	\ensuremath{##3}
\end{adjustwidth}
\}
{\normalfont\bfseries\,\par}%
}

\vspace{1ex}
\noindent
{\normalfont\bfseries\ttfamily\large TAD #1\ifthenelse{\equal{#2}{}}{}{$<$#2$>$}} \{
% Abrimos la llave
\par%
\tocarEspacios
}{
\vspace{1ex} \par\}
}

\newcommand{\obs}[2]{
	obs #1: \ensuremath{#2}\par
}

\newenvironment{module}[4]{
\newcommand{\tadtype}{%
	\ensuremath{#3 \ifblank{#4}{}{\lrangle{#4}}}
}

\newcommand{\moduletype}{%
	\ensuremath{#1 \ifblank{#2}{}{\lrangle{#2}}}
}


\renewcommand{\pred}[3]{%
{\normalfont\bfseries\ttfamily pred }%
	{\normalfont\ttfamily ##1}%
\ifthenelse{\equal{##2}{}}{}{\ (##2) }%
\{%
\begin{adjustwidth}{+5em}{}
	\ensuremath{##3}
\end{adjustwidth}
\}
{\normalfont\bfseries\,\par}%
}

\vspace{1ex}
\noindent
{\normalfont\bfseries\ttfamily\large Módulo #1\ifthenelse{\equal{#2}{}}{}{$<$#2$>$} \ifblank{#3}{}{implementa #3\ifthenelse{\equal{#4}{}}{}{$<$#4$>$}}} \{
% Abrimos la llave
\par%
\tocarEspacios
}{
\vspace{1ex} \par\}
}

\newcommand{\var}[2]{
	var #1: \;\texto{#2}\par
}
\newcommand{\compl}[1]{Complejidad: $#1$}

% Tipos de datos de implementación
\newcommand{\Int}{\ensuremath{\mathsf{int}}}
\newcommand{\Real}{\ensuremath{\mathsf{real}}}
\newcommand{\Bool}{\ensuremath{\mathsf{bool}}}
\newcommand{\Char}{\ensuremath{\mathsf{char}}}
\newcommand{\String}{\ensuremath{\mathsf{string}}}
\newcommand{\Array}[1]{\ensuremath{\mathsf{Array<#1>}}}

\newcommand{\texto}[1]{\text{\normalfont\bfseries\ttfamily #1\;}}

\begin{document}

\titulo{Guia 8}
\materia{Algoritmos y Estructuras de Datos I}
\fecha{2do cuatrimestre 2024}

\integrante{Federico Barberón}{112/24}{jfedericobarberonj@gmail.com}
%Carátula
\maketitle
\newpage

%Indice
\tableofcontents
\newpage

\section{Guia 8}

\subsection{Ejercicio 1}
Tenemos un TAD que modela las ventas minoristas de un comercio. Cada venta es individual (una unidad de un producto) y se quieren registrar todas las ventas. El TAD tiene un único observador:
\medskip
\\
Producto es \str\\
Monto es \ent\\
Fecha es \ent (segundos desde 1/1/1970)
\bigskip

\begin{tad}{Comercio}{}
	\obs{ventasPorProducto}{\dict{Producto, \TLista{\tupla{Fecha, Monto}}}}
\end{tad}

\textbf{ventasPorProducto} contiene, para cada producto, una secuencia con todas las ventas que se hicieron de ese producto. Para cada venta, se registra la fecha y el precio. Se puede considerar que todas las fechas son diferentes. Este TAD lo vamos a implementar con la siguiente estructura:
\bigskip

\begin{module}{ComercioImpl}{}{Comercio}{}
	\var{ventas}{SecuenciaImpl\lrangle{tupla\lrangle{Producto, Fecha, Monto}}}
	\var{totalPorProducto}{DiccionarioImpl\lrangle{Producto, Monto}}
	\var{ultimoPrecio}{DiccionarioImpl\lrangle{Producto, Monto}}
\end{module}

\begin{itemize}
	\item \textbf{ventas} es una implementación de secuencia con todas las ventas realizadas, indicando producto, fecha y monto.
	\item \textbf{totalPorProducto} asocia cada producto con el dinero total obtenido por todas sus ventas.
	\item \textbf{ultimoPrecio} asocia cada producto con el monto de su última venta registrada.
\end{itemize}

Se pide:

\begin{itemize}
	\item Escribir en forma coloquial y detallada el invrep y la func abs.
	\item Escribir ambos en el lenguaje de especificación.
\end{itemize}

InvRep:
\begin{itemize}
	\item Las claves de totalPorProducto son iguales a las claves de ultimoPrecio.
	\item Una clave está en totalPorProducto si y solo si está como primer componente de algún elemento de ventas.
	\item La segunda componente de todos los elementos de ventas son distintas.
	\item La tercera componente de todos los elementos de ventas son positivas.
	\item El valor de todas las claves de totalPorProducto es igual a la suma de la tercer componente de todos los elementos de ventas que tengan como primer componente esa misma clave.
	\item El valor de todas las claves de ultimoPrecio es igual a la tercer componente del elemento de ventas que su primer componente es esa misma clave y la segunda componente es el mayor de todos los elementos de ventas con esa misma clave.
\end{itemize}

FuncAbs:
\begin{itemize}
	\item Una clave pertenece a ventasPorProducto si y solo si pertenece a totalPorProducto
	\item El valor de una clave c perteneciente a ventasPorProducto es una secuencia de tuplas con la segunda y tercer componente de todos los elementos de ventas que tienen como primer componenete a c
\end{itemize}
\pagebreak

\pred{clavesIguales}{d1: \dict{Producto, Monto}, d2: \dict{Producto, Monto}}{
	\paraTodo{c}{Producto}{c \in d1 \iff c \in d2}
}

\pred{todosFechasDistintas}{v: \TLista{\tupla{Producto, Fecha, Monto}}}{
	\paraTodo{i,j}{\ent}{
		0 \leq i < j < v.length \thenLuego v[i].fecha \neq v[j].fecha
	}
}

\pred{todosMontosPositivos}{v: \TLista{\tupla{Producto, Fecha, Monto}}}{
	\paraTodo{i}{\ent}{
		0 \leq i < v.length \thenLuego v[i].monto \geq 0
	}
}

\pred{esVentaMasRecienteDeProd}{ventas: \TLista{\tupla{Producto, Fecha, Monto}}, v: \tupla{Producto, Fecha, Monto}}{
	v \in ventas \land \paraTodo{v2}{\tupla{Producto, Fecha, Monto}}{
		v1 \in ventas \Then v1.fecha \leq v.fecha
	}
}

\aux{sumaMontosDeProducto}{v: \TLista{\tupla{Producto, Fecha, Monto}}, p: Producto}{\ent}{\\
	\Indent\displaystyle\sum_{i=0}^{v.length - 1} \If{v[i].producto = p}{v[i].monto}{0}
}
\bigskip

\pred{InvRep}{c: ComercioImpl}{
	clavesIguales(c.totalPorProducto.data, c.ultimoPrecio.data) \land \\
	\paraTodo[multLineas]{p}{Producto}{p \in c.totalPorProducto.data \iff \existe[multLineas]{v}{\tupla{Producto, Fecha, Monto}}{v \in c.ventas.s \land v.producto = p}} \land \\
	todosFechasDistintas(c.ventas.s) \land todosMontosPositivos(c.ventas.s) \land \\
	\paraTodo[multLineas]{p}{Producto}{p \in c.totalPorProducto.data \Then c.totalPorProducto.data[p] = sumaMontosDeProducto(c.ventas.s, p)} \land \\
	\paraTodo{p}{Producto}{p \in c.ultimoPrecio.data \Then \existe[multLineas]{v}{\tupla{Producto, Fecha, Monto}}{v \in c.ventas.s \land v.producto = p \land v.monto = c.ultimoPrecio.data[p] \land \\
			esVentaMasRecienteDeProd(c.ventas.data, v)}}
}

\aux{ventasDeProd}{v: \TLista{\tupla{Producto, Fecha, Monto}}, p: Producto, \\
	\Indent\Indent\Indent\Indent\Indent list: \TLista{\tupla{Fecha, Monto}}}{\TLista{\tupla{Fecha, Monto}}}{\\
	\Indent\If{v.length == 0}{list}{\\
		\Indent\Indent\If{v[0].producto = p}{\\
			\Indent\Indent\Indent ventasPorProd(tail(v), p, list \masmas \lrangle{v[0].fecha, v[0].monto})}{\\
			\Indent\Indent\Indent ventasPorProducto(tail(v), p, list)}
	}
}
\bigskip

\pred{predAbs}{c1: ComercioImpl, c2: Comercio}{
	clavesIguales(c1.totalPorProducto.data, c2.ventasPorProducto) \land \\
	\paraTodo[multLineas]{p}{Producto}{p \in c2.ventasPorProducto \Then c2.ventasPorProducto[p] = ventasPorProd(c1.ventas.s, p, [])}
}
\pagebreak

\subsection{Ejercicio 2}
Considere la siguiente especificación de una relación uno/muchos entre alarmas y sensores de una planta industrial: un sensor puede estar asociado a muchas alarmas, y una alarma puede tener muchos sensores asociados.
\bigskip

\begin{tad}{Planta}{}
	\obs{alarmas}{\conj{alarma}}
	\obs{sensores}{\conj{\lrangle{Sensor, Alarma}}}

	\begin{proc}{nuevaPlanta}{}{\tadtype}
		\asegura{res.alarmas = \{\}}
		\asegura{res.sensores = \{\}}
	\end{proc}

	\begin{proc}{agregarAlarma}{\Inout p: \tadtype, \In a: Alarma}{}
		\requiere{p = P_0}
		\requiere{a \notin p.alarmas}
		\asegura{p.alarmas = P_0.alarmas \cup \{a\}}
		\asegura{p.sensores = P_0.sensores}
	\end{proc}

	\begin{proc}{agregarSensor}{\Inout p: \tadtype, \In a: Alarma, \In s: Sensor}{}
		\requiere{p = P_0}
		\requiere{a \in p.alarmas}
		\requiere{\lrangle{s, a} \notin p.sensores}
		\asegura{p.alarmas = P_0.alarmas}
		\asegura{p.sensores = P_0.sensores \cup \{\lrangle{s, a}\}}
	\end{proc}
\end{tad}

Se decidió utilizar la siguiente estructura como representación, que permite consultar fácilmente tanto en una dirección (sensores de una alarma) como en la contraria (alarmas de un sensor)
\bigskip

\begin{module}{PlantaImpl}{}{Planta}{}
	\var{alarmas}{Diccionario\lrangle{Alarma, Conjunto\lrangle{Sensor}}}
	\var{sensores}{Diccionario\lrangle{Sensor, Conjunto\lrangle{Alarma}}}
\end{module}

Se pide:
\begin{itemize}
	\item Escribir formalmente y en castellano el invrep.
	\item Escribir la func Abs.
\end{itemize}

InvRep:

Opción 1:
\begin{itemize}
	\item Todos los elementos de los valores de todas las claves de $alarmas$ son claves de $sensores$
	\item Todos los elementos de los valores de todas las claves de $sensores$ son claves de $alarmas$
\end{itemize}
Opción 2:
\begin{itemize}
	\item Para toda clave $k$ en $alarmas$, todo elemento $e$ en $alarmas[k]$ es clave de $sensores$
	\item Para toda clave $k$ en $sensores$, todo elemento $e$ en $sensores[k]$ es clave de $alarmas$
\end{itemize}

Func Abs:

Sea $p1: PlantaImpl, p2: Planta$
\begin{itemize}
	\item Una alarma $a$ es una clave de $p1.alarmas$ si y solo si $a$ pertenece a $p2.alarmas$
	\item Un par de sensor y alarma \lrangle{s, a} pertenece a $p2.sensores$ si y solo si $s$ pertenece a $p1.sensores$ y $a$ pertenece a $p1.sensores[s]$
\end{itemize}

\subsection{Ejercicio 3}
\hacer

\subsection{Ejercicio 4}
Se desea diseñar un sistema para registrar las notas de los alumnos en una facultad. Al igual que en Exactas, los alumnos se identifican con un número de LU. A su vez, las materias tienen un nombre, y puede haber una cantidad no acotada de materias. En cada materia, las notas están entre 0 y 10, y se aprueban si la nota es mayor o igual a 7.
\bigskip

\begin{tad}{Sistema}{}
	\obs{notas}{\dict{materia, \dict{alumno, \ent}}}

	\begin{proc}{nuevoSistema}{}{\tadtype}
	\end{proc}
	\begin{proc}{registrarMateria}{\Inout s: \tadtype, \In m: materia}{}
	\end{proc}
	\begin{proc}{registrarNota}{\Inout s: \tadtype, \In m: materia, \In a: alumno, \In n: nota}{}
	\end{proc}
	\begin{proc}{notaDeAlumno}{\In s: \tadtype, \In a: alumno, \In m: materia}{nota}
	\end{proc}
	\begin{proc}{cantAlumnosConNota}{\In s: \tadtype, \In m: materia, \In n: nota}{\ent}
	\end{proc}
	\begin{proc}{cantAlumnosAprobados}{\In s: \tadtype, \In m: materia}{\ent}
	\end{proc}
\end{tad}

Dados $m = cantMaterias$ y $n = cantAlumnos$ se desea diseñar un módulo con los siguientes requerimientos de complejidad temporal:
\begin{itemize}
	\item nuevoSistema $O(1)$
	\item registrarMateria $O(\log m)$
	\item registrarNota $O(\log n + \log m)$
	\item notaDeAlumno $O(\log n + \log m)$
	\item cantAlumnosConNota y cantAlumnosAprobados $O(\log m)$
\end{itemize}

Materia es \str

Alumno es \str

Nota es \Int

\begin{module}{SistemaImpl}{}{Sistema}{}
	\var{notas}{DiccionarioLog\lrangle{Materia, DiccionarioLog\lrangle{Alumno, Nota}}}
	\var{cantAlumnosPorNota}{DiccionarioLog\lrangle{Materia, Array\lrangle{int}}}
	\bigskip

	\pred{esNota}{n: \ent}{0 \leq n \leq 10}

	\pred{InvRep}{s: \moduletype}{
		\paraTodo{m}{Materia}{m \in s.notas.data \iff m \in s.notas.cantAlumnosPorNota.data} \land \\
		\paraTodo{m}{Materia}{m \in s.notas.data \Then esNota(s.notas.data[m])} \land \\
		\paraTodo[multLineas]{m}{Materia}{m \in s.cantAlumnosPorNota.data \Then s.cantAlumnosPorNota.data[m].length = 11 \land \\
			\paraTodo{n}{\ent}{esNota(n) \thenLuego (?)}}
	}

	\pred{predAbs}{s1: \moduletype, s2: \tadtype}{
		\hacer
	}

	\begin{proc}{nuevoSistema}{}{\moduletype}
		\compl{O(1)}
		\begin{lstlisting}[numbers=none,frame=none]
		res.notas := diccionarioVacio();
		res.cantAlumnosPorNota := diccionarioVacio();

		return res;
		\end{lstlisting}
	\end{proc}

	\begin{proc}{registrarMateria}{\Inout s: \moduletype, \In m: Materia}{}
		\compl{O(\log m)}
		\begin{lstlisting}[numbers=none,frame=none]
		var alumnos: DiccionarioLog < Alumno, Nota >;
		var notas: Array < int >;
		var i := 0;

		alumnos := diccionarioVacio();
		notas := new Array < int > (11);

		while i < 11 do
			notas[i] := 0;
			i := i + 1;
		endwhile

		s.notas.definir(m, alumnos);
		s.cantAlumnosPorNota.definir(m, notas);
		\end{lstlisting}
	\end{proc}

	\begin{proc}{registrarNota}{\Inout s: \moduletype, \In m: Materia, \In a: Alumno, \In n: Nota}{}
		\compl{O(\log n + \log m)}
		\begin{lstlisting}[numbers=none,frame=none]
		var notasDeMateria: DiccionarioLog < Materia, Array < int >>;

		s.notas.obtener(m).definir(a, n);
		
		notasDeMateria := s.cantAlumnosPorNota.obtener(m);
		notasDeMateria[n] := notasDeMateria[n] + 1;
		\end{lstlisting}
	\end{proc}

	\begin{proc}{notaDeAlumno}{\In s: \moduletype, \In a: Alumno, \In m: Materia}{Nota}
		\compl{O(\log n + \log m)}
		\begin{lstlisting}[numbers=none,frame=none]
		return s.notas.obtener(m).obtener(a);
		\end{lstlisting}
	\end{proc}

	\begin{proc}{cantAlumnosConNota}{\In s: \moduletype, \In m: Materia, \In n: Nota}{\Int}
		\compl{O(\log m)}
		\begin{lstlisting}[numbers=none,frame=none]
		return s.cantAlumnosPorNota.obtener(m)[n];
		\end{lstlisting}
	\end{proc}

	\begin{proc}{cantAlumnosAprobados}{\In s: \moduletype, \In m: Materia}{\Int}
		\compl{O(\log m)}
		\begin{lstlisting}[numbers=none,frame=none]
		var cant: int;
		var nota: int;
		var notasDeMateria: DiccionarioLog < Materia, Array < int >>;
		cant := 0;
		nota := 7;
		notasDeMateria := s.cantAlumnosPorNota.obtener(m);

		while nota < 11 do
			cant := cant + notasDeMateria[nota];
			nota := nota + 1;
		endwhile

		return cant;
		\end{lstlisting}
	\end{proc}
\end{module}

\subsection{Ejercicio 5}
El TAD Matriz infinita de booleanos tiene las siguientes operaciones:
\begin{itemize}
	\item Crear, que crea una matriz donde todos los valores son falsos.
	\item Asignar, que toma una matriz, dos naturales (fila y columna) y un booleano, y asigna a este último en esa coordenada. (Como la matriz es infinita, no hay restricciones sobore la magnitud de fila y columna).
	\item Ver, que dadas una matriz, una fila y una columna devuelve el valor de esa coordenada. (Idem.)
	\item Complementar, que invierte todos los valores de la matriz.
\end{itemize}
\bigskip

\begin{minipage}[t]{0.45\textwidth}
	Ejemplo de uso del módulo:
	\begin{lstlisting}[numbers=none,frame=none]
MatrizInfinita M := Crear()
bool b1 := Ver(M , 0, 0)
Asignar(M , 1, 3, False)
Asignar(M , 100, 5000, True)
bool b2 := M .Ver(100, 5000)
Complementar( M )
bool b3 := Ver(M , 394, 788)
bool b4 := Ver(M , 100, 5000)
	\end{lstlisting}
\end{minipage}
\begin{minipage}[t]{0.45\textwidth}
	Tras lo ue deberíamos tener
	\begin{lstlisting}[numbers=none,frame=none]
b1 = False
b2 = True
b3 = True
b4 = False
	\end{lstlisting}
\end{minipage}

Elija la estructura y escriba los algoritmos de modod que las operaciones Crear, Ver y Complementar tomen $O(1)$ tiempo en peor caso.

\begin{module}{MatrizInfinita}{}{MatrizInfinitaBooleanos}{}
	\var{matriz}{Vector\lrangle{Vector\lrangle{bool}}}
	\var{complementar}{bool}

	\begin{proc}{Crear}{}{\moduletype}
		\begin{lstlisting}[numbers=none,frame=none]
		res.matriz := vectorVacio();
		res.complementar := false;
		return res;
		\end{lstlisting}
	\end{proc}

	\begin{proc}{Asignar}{\Inout m: \moduletype, \In f: \Int, \In c: \Int, \In b: \bool}{}
		\hacer
	\end{proc}

	\begin{proc}{Ver}{\In m: \moduletype, \In f: \Int, \In c: \Int}{\bool}
		\begin{lstlisting}[numbers=none,frame=none]
		if m.matriz.longitud() == 0 || f >= m.matriz.longitud() || c >= m.matriz[0].longitud() then
			return m.complementar;
		else if m.complementar then
			return !m.matriz[f][c];
		else
			return m.matriz[f][c];
		endif
		\end{lstlisting}
	\end{proc}

	\begin{proc}{Complementar}{\Inout m: \moduletype}{}
		\begin{lstlisting}[numbers=none,frame=none]
		m.complementar := !m.complementar;
		\end{lstlisting}
	\end{proc}
\end{module}

\subsection{Ejercicio 6}
Una matriz finita posee las siguientes operaciones:
\begin{itemize}
	\item Crear, con la cnatidad de filas y columnas que albergará la matriz.
	\item Definir, que permite definir el valor para una posición válida.
	\item \#Filas, que retorna la cantidad de filas de la matriz.
	\item \#Columnas, que retorna la cantidad de columas de la matriz.
	\item Obtener, que devuelve el valor de una posición válida de la matriz (si nunca se definió la matriz en la posición solicitada devuelve cero).
	\item SumarMatrices, que permite sumar dos matrices de iguales dimensiones.
\end{itemize}

Dado $n$ y $m$ son la cantidad de elementos no nulos de $A$ y $B$, respectivamente, diseñe un módulo (elegir una estructura y escribir los algoritmos) para el TAD MatrizFinita de modo tal que dados dos matrices finitas $A$ y $B$,
\begin{enumerate}[label=(\alph*)]
	\item Definir y Obtener aplicadas a $A$ se realicen cada una en $\Theta(n)$ en peor caso, y
	\item SumarMatrices aplicada a $A$ y $B$ se realice en $\Theta(n + m)$ en peor caso.
\end{enumerate}

\hacer

\subsection{Ejercicio 8}
Se desea diseñar un sistema de estadísticas para la cantidad de personas que ingresan a un banco. Al final del día, un empleado del banco ingresa en el sistema el total de ingresantes para ese día. Se desea saber, en cualquier intervalo de días, la cantidad total de personas que ingresaron al banco. La siguiente es una especificación del problema.

\begin{tad}{IngresosAlBanco}{}
	\obs{totales}{\TLista{\ent}}

	\begin{proc}{nuevoIngresos}{}{\tadtype}
		\asegura{res.totales = []}
	\end{proc}

	\begin{proc}{registrarNuevoDia}{\Inout i: \tadtype, \In cant: \ent}{}
		\requiere{cant \geq 0 \land i = I_0}
		\asegura{i.totales = I_0.totales.concat(cant)}
	\end{proc}

	\begin{proc}{cantDias}{\In i: \tadtype}{\ent}
		\asegura{res = |i.totales|}
	\end{proc}

	\begin{proc}{cantPersonas}{\In i: \tadtype, \In desde: \ent, \In hasta: \ent}{\ent}
		\requiere{0 \leq desde \leq hasta < |i.totales|}
		\asegura{res = \sum_{j=desde}^{hasta} i.totales[j]}
	\end{proc}
\end{tad}

\begin{enumerate}
	\item Dar una estructura de representación que permita que la función \textit{cantPersonas} tome $O(1)$.
	\item Calcular cómo crece el tamaño de la estructura en función de la cantidad de días que pasaron.
	\item Si el cálculo del punto anterior fue una función que no es $O(n)$, piense otra estructura que permita resolver el problema utilizando $O(n)$ memoria.
	\item Agregue al diseño del punto anterior una operación \textit{mediana} que devuelva el último (mayor) día $d$ tal que $cantPersonas(i, 0, d) \leq cantPersonas(i, d + 1, cantDias(i) - 1)$, restringiendo la operación a los casos donde dicho día existe.
\end{enumerate}

\begin{module}{IngresosAlBancoImpl}{}{IngresosAlBanco}{}
	\var{totalesAcumulados}{Vector\lrangle{int}} \par
	Crece proporcionalmente a la cantidad de días que pasaron

	\begin{proc}{nuevoIngresos}{}{\moduletype}
		\begin{lstlisting}[numbers=none,frame=none]
		res.totales := new Vector < int > ();
		return res;
		\end{lstlisting}
	\end{proc}

	\begin{proc}{registrarNuevoDia}{\Inout i: \moduletype, \In cant: \Int}{}
		\begin{lstlisting}[numbers=none,frame=none]
		if i.cantDias() == 0 then
			i.totalesAcumulados.add(cant);
		else
			i.totalesAcumulados.add(cant + i.totalesAcumulados[i.cantDias() - 1]);
		endif
		\end{lstlisting}
	\end{proc}

	\begin{proc}{cantDias}{\In i: \moduletype}{\Int}
		\begin{lstlisting}[numbers=none,frame=none]
		return i.totalesAcumulados.size();
		\end{lstlisting}
	\end{proc}

	\begin{proc}{cantPersonas}{\In i: \moduletype, \In desde: \Int, \In hasta: \Int}{\Int}
		\compl{O(1)}
		\begin{lstlisting}[numbers=none,frame=none]
		if desde == 0 then
			return i[hasta];
		endif

		return i[hasta] - i[desde - 1];
		\end{lstlisting}
	\end{proc}

	\begin{proc}{mediana}{\In i: \moduletype}{\Int}
		\begin{lstlisting}[numbers=none,frame=none]
		var mediana: int;
		var j: int;
		mediana := -1;
		j := 0

		while j < i.cantDias() do
			if i.cantPersonas(0, j) <= i.cantPersonas(j + 1, i.cantDias() - 1) && j > mediana then
				mediana := j;
			endif

			j := j + 1;
		endwhile

		return mediana;
		\end{lstlisting}
	\end{proc}
\end{module}

\subsection{Ejercicio 9}
La Maderera San Blas vende, entre otras cosas, listones de madera. Los compra en aserraderos de la zona, los cepilla y acondiciona, y los vende por menor del largo que el cliente necesite.
Tienen un sistema un poco particular y ciertamente no muy eficiente: Cuando ingresa un pedido, buscan el listón más largo que tiene en el depósito, realizan el corte del tamaño que el cliente pidió, y devuelven el trozo que queda al depósito.
Por otra parte, identifican a cada cliente con un código alfanumérico de 10 dígitos y cuentan con un fichero en el que registran todas las compras que hizo cada cliente (con la fecha de la compra y el tamaño del listón vendido).
Este sería el TAD simplificado del sistema:
\pagebreak

\texto{Cliente es string}
\begin{tad}{Maderera}{}
	\begin{proc}{comprarUnListon}{\Inout m: \tadtype, \In tamaño: \ent}{}
		\textit{// comprar en el aserradero un listón de un determinado tamaño}
	\end{proc}

	\begin{proc}{venderUnListon}{\Inout m: \tadtype, \In tamaño: \ent, \In cli: Cliente, \In f: Fecha}{}
		\textit{// vender un listón de un determinado tamaño a un cliente particular en una fecha determinada}
	\end{proc}

	\begin{proc}{ventasACliente}{\In m: \tadtype, \In cli: Cliente}{}
		\textit{// devolver el conjunto de todas las ventas que se le hicieron a un cliente} \\
		\textit{// (para cada venta, se quiere saber la fecha y el tamaño del listón)}
	\end{proc}
\end{tad}

Se pide:

\begin{itemize}
	\item Escriba una estructura que permita realizar las operaciones indicadas con las siguientes complejidades:

	      \begin{itemize}
		      \item comprarUnListon en $O(\log(m))$
		      \item venderUnListon en $O(\log(m))$
		      \item ventasACliente en $O(1)$
	      \end{itemize}

	      donde m es la cantidad de pedazos de listón que hay en el depósito

	\item Escriba el algoritmo para la operación venderUnListon
\end{itemize}

\begin{module}{MadereraImpl}{}{Maderera}{}
	\var{listones}{ConjuntoLog\lrangle{\Int}}
	\var{ventasPorCliente}{DiccDigital\lrangle{Cliente, ListaEnlazada\lrangle{\struct{tamaño: int, fecha: Fecha}}}} \textit{// Como las claves están acotadas a 10 dígitos, las operaciones son O(1)}

	\begin{proc}{venderUnListon}{\Inout m: \moduletype, \In tamaño: \Int, \In cli: Cliente, \In f: Fecha}{}
		\compl{O(\log(m))}
		\begin{lstlisting}[numbers=none,frame=none]
		var venta: Struct < tamano: int, fecha: Fecha >;
		venta := new Struct < tamano: int, fecha: Fecha >;
		venta.tamano := tamano;
		venta.fecha := f;

		m.listones.sacar(tamano); // O( log m )
		
		if m.ventasPorCliente.esta(cli) then // O(1)
			m.ventasPorCliente.obtener(cli).agregarAtras(venta); // O(1)
		else
			var ventas: ListaEnlazada< Struct < tamano: int, fecha: Fecha > >;
			ventas := new ListaEnlazada();
			ventas.agregarAtras(venta); // O(1)
			m.ventasPorCliente.definir(cli, ventas); // O(1)
		endif
		\end{lstlisting}
	\end{proc}
\end{module}

\subsection{Ejercicio 10}
Se quiere implementar el TAD Biblioteca que modela una biblioteca con su colección de libros. Se modela la biblioteca como una sola estantería de capacidad arbitraria, dentro de la cual cada libro ocupa una posición. La biblioteca cuenta con un registro de socios que pueden retirar y devolver libros en cualquier momento. Por restricciones del sistema que se utiliza, un socio no puede registrarse con un nombre de más de 50 caracteres.
Cuando la biblioteca adquiere un nuevo libro o cuando un libro es devuelto, éste es insertado en el primer espacio libre de la estantería. Es decir, si los lugares ocupados son 1, 2, 3, 4 y se presta el libro en la posición 2, al agregar un nuevo libro al catálogo éste será ubicado en la posición 2. Cuando el libro que estaba originalmente en la posición 2 sea devuelto, será ubicado en la primera posición libre, que será la 5.
El TAD tiene las siguientes operaciones, para las que se nos piden las complejidades temporales de peor caso indicadas, donde

\begin{itemize}
	\item $L$ es la cantidad de libros en la colección
	\item $r$ es la cantidad de libros que el socio en cuestión tiene retirados
	\item $k$ es la cantidad de posiciones libres en la estantería
\end{itemize}

Socio es \str; Libro, Posicion es \Int
\begin{module}{Biblioteca}{}{}{}
	\var{socios}{DiccionarioDigital\lrangle{Socio, ConjuntoLog\lrangle{Libro}}}
	\textit{// Socios y los lirbos que tienen prestados} \par
	\bigskip
	\var{catalogo}{DiccionarioLog\lrangle{Libro, Posicion}}
	\textit{// Libros y su posición en la estantería} \par
	\bigskip
	\var{posicionesLibres}{ColaDePrioridadLog\lrangle{Posicion}}
	\textit{// Posiciones vacías en la estantería}
	\textit{// Las posiciones mas chicas tienen mayor prioridad en la cola}
	\bigskip

	\begin{proc}{AgregarLibroAlCatálogo}{\Inout b: \moduletype, \In l: idLibro}{}
		\begin{adjustwidth}{+5em}{}
			\begin{itemize}
				\item \textbf{Requiere}: $l$ no pertenece a la colección de libros de $b$
				\item \textbf{Descripción}: la biblioteca adquiere un nuevo libro, lo suma a su catálogo y lo pone en la estantería en el primer espacio disponible
				\item \textbf{Complejidad}: $O(\log(k) + \log(L))$
			\end{itemize}
		\end{adjustwidth}

		\begin{lstlisting}[numbers=none,frame=none]
		var pos: Posicion;

		if b.posicionesLibres.size() == 0 then
			pos := b.catalogo.size(); // O(1)
		else
			pos := b.posicionesLibres.desencolar(); // O(log(k))
		endif // O(log(k))

		b.catalogo.definir(l, pos); // O(log(L))
		\end{lstlisting}
		Complejidad final: $O(\log(k) + \log(L))$
	\end{proc}

	\begin{proc}{PedirLibro}{\Inout b: \moduletype, \In l: idLibro, \In s: Socio}{}
		\begin{adjustwidth}{+5em}{}
			\begin{itemize}
				\item \textbf{Requiere}: $s$ es socio de la biblioteca y el libro $l$ no está entre los libros prestados
				\item \textbf{Descripción}: el socio pasa a retirar un libro que se retira de la estantería y se acumula en sus libros prestados.
				\item \textbf{Complejidad}: $O(\log(r) + \log(k) + \log(L))$
			\end{itemize}
		\end{adjustwidth}

		\begin{lstlisting}[numbers=none,frame=none]
		var pos: Posicion := b.catalogo.obtener(l); // O(log(L))
		b.catalogo.borrar(l); // O(log(L))
		b.posicionesLibres.encolar(pos); // O(log(k))
		b.socios.obtener(s).agregar(l); // O(1) + O(log(r))
		\end{lstlisting}
		Complejidad final: $O(2\log(L) + \log(k) + 1 + \log(r)) = O(\log(r) + \log(k) + \log(L))$
	\end{proc}

	\pagebreak

	\begin{proc}{DevolverLibro}{\Inout b: \moduletype, \In l: idLibro, \In s: Socio}{}
		\begin{adjustwidth}{+5em}{}
			\begin{itemize}
				\item \textbf{Requiere}: $s$ es socio de la biblioteca y el libro $l$ está entre sus libros prestados
				\item \textbf{Descripción}: el socio pasa a devolver un libro que previamente había tomado prestado. Vuelve a la estantería en el primer espacio disponible.
				\item \textbf{Complejidad}: $O(\log(r) + \log(k) + \log(L))$
			\end{itemize}
		\end{adjustwidth}

		\begin{lstlisting}[numbers=none,frame=none]
		b.socios.obtener(s).sacar(l); // O(1) + O(log(r))
		var pos: Posicion;

		if b.posicionesLibres.size() == 0 then
			pos := b.catalogo.size(); // O(1)
		else
			pos := b.posicionesLibres.desencolar(); // O(log(k))
		endif // O(log(k))

		b.catalogo.definir(l, pos); // O(log(L))
		\end{lstlisting}
		Complejidad final: $O(1 + \log(r) + \log(k) + \log(L)) = O(\log(r) + \log(k) + \log(L))$
	\end{proc}

	\begin{proc}{Prestados}{\Inout b: \moduletype, \In s: Socio}{Conjunto\lrangle{Libro}}
		\begin{adjustwidth}{+5em}{}
			\begin{itemize}
				\item \textbf{Requiere}: $s$ es socio de la biblioteca
				\item \textbf{Descripción}: este procedimiento retorna los libros que el socio tomó prestados de la biblioteca y aún no devolvió
				\item \textbf{Complejidad}: $O(1)$
			\end{itemize}
		\end{adjustwidth}

		\begin{lstlisting}[numbers=none,frame=none]
		return b.socios.obtener(s); // O(1)
		\end{lstlisting}
	\end{proc}

	\begin{proc}{UbicacionDeLibro}{\Inout b: \moduletype, \In l: idLibro}{Posicion}
		\begin{adjustwidth}{+5em}{}
			\begin{itemize}
				\item \textbf{Requiere}: $l$ pertenece a la colección de libros de $b$ y no está prestado
				\item \textbf{Descripción}: obtiene la posición del libro en la estantería
				\item \textbf{Complejidad}: $O(\log(L))$
			\end{itemize}
		\end{adjustwidth}

		\begin{lstlisting}[numbers=none,frame=none]
		return b.coleccion.obtener(l); // O(log(L))
		\end{lstlisting}
	\end{proc}

\end{module}

\end{document}

\documentclass[11pt, a4paper]{article}
\usepackage[paper=a4paper, left=1.5cm, right=1.5cm, bottom=1.5cm, top=1.5cm]{geometry}
%
\usepackage[utf8]{inputenc}
\usepackage[spanish]{babel}

\usepackage{../caratuladc/caratula/caratula}

\usepackage{multicol}
\usepackage{enumitem}
\usepackage{amsmath}
\usepackage{amssymb}
\usepackage{xcolor}

% Comandos %

\newcommand{\sii}{\leftrightarrow}
\newcommand{\hacer}{\textcolor{red}{HACER!}}
\usepackage[spanish,activeacute,es-tabla]{babel}
\usepackage[utf8]{inputenc}
\usepackage{ifthen}
\usepackage{listings}
\usepackage{dsfont}
\usepackage{subcaption}
\usepackage{amsmath}
\usepackage[top=1cm,bottom=2cm,left=1cm,right=1cm]{geometry}%
\usepackage{color}%
\usepackage{changepage}
\newcommand{\tocarEspacios}{%
	\addtolength{\leftskip}{3em}%
	\setlength{\parindent}{0em}%
}

\newcommand{\Indent}{\hspace*{0.75cm}}

% Especificacion de procs

\newcommand{\In}{\textsf{in }}
\newcommand{\Out}{\textsf{out }}
\newcommand{\Inout}{\textsf{inout }}

\newcommand{\encabezadoDeProc}[3]{%
% Ponemos la palabrita problema en tt
%  \noindent%
{\normalfont\bfseries\ttfamily proc}%
% Ponemos el nombre del problema
\ %
{\normalfont\ttfamily #1}%
\
% Ponemos los parametros
(#2)%
\ifblank{#3}{}{%
	% Por ultimo, va el tipo del resultado
	\ : #3}
}

\newenvironment{proc}[4][res]{%

% El parametro 1 (opcional) es el nombre del resultado
% El parametro 2 es el nombre del problema
% El parametro 3 son los parametros
% El parametro 4 es el tipo del resultado
% Preambulo del ambiente problema
% Tenemos que definir los comandos requiere, asegura, modifica y aux
\newcommand{\requiere}[2][]{%
{\normalfont\bfseries\ttfamily requiere\;}%
\ifthenelse{\equal{##1}{variaslineas}}{\{%
\begin{adjustwidth}{+5em}{}
	\ensuremath{##2}
\end{adjustwidth}
\}
{\normalfont\bfseries\,\par}%
}
{%
\{\ensuremath{##2}\}%
{\normalfont\bfseries\,\par}%
}
}

\newcommand{\asegura}[2][]{%
{\normalfont\bfseries\ttfamily asegura\;}%
\ifthenelse{\equal{##1}{variaslineas}}{\{%
\begin{adjustwidth}{+5em}{}
	\ensuremath{##2}
\end{adjustwidth}
\}
{\normalfont\bfseries\,\par}%
}
{%
\{\ensuremath{##2}\}%
{\normalfont\bfseries\,\par}%
}
}
\renewcommand{\aux}[4]{%
	{\normalfont\bfseries\ttfamily aux\ }%
		{\normalfont\ttfamily ##1}%
	\ifthenelse{\equal{##2}{}}{}{\ (##2)}\ : ##3\, = \ensuremath{##4}%
	{\normalfont\bfseries\,;\par}%
}
\renewcommand{\pred}[3]{%
{\normalfont\bfseries\ttfamily pred }%
	{\normalfont\ttfamily ##1}%
\ifthenelse{\equal{##2}{}}{}{\ (##2) }%
\{%
\begin{adjustwidth}{+5em}{}
	\ensuremath{##3}
\end{adjustwidth}
\}%
{\normalfont\bfseries\,\par}%
}

\newcommand{\res}{#1}
\vspace{1ex}
\noindent
\encabezadoDeProc{#2}{#3}{#4}
% Abrimos la llave
\par%
\tocarEspacios
}
{
% Cerramos la llave
\vspace{1ex}
}

\newcommand{\aux}[4]{%
	{\normalfont\bfseries\ttfamily\noindent aux\ }%
		{\normalfont\ttfamily #1}%
	\ifthenelse{\equal{#2}{}}{}{\ (#2)}\ : #3\, = \ensuremath{#4}%
	{\normalfont\bfseries\,;\par}%
}

\newcommand{\pred}[3]{%
{\normalfont\bfseries\ttfamily\noindent pred }%
	{\normalfont\ttfamily #1}%
\ifthenelse{\equal{#2}{}}{}{\ (#2) }%
\{%
\begin{adjustwidth}{+2em}{}
	\ensuremath{#3}
\end{adjustwidth}
\}%
{\normalfont\bfseries\,\par}%
}

% Tipos

\newcommand{\nat}{\ensuremath{\mathbb{N}}}
\newcommand{\reals}{\ensuremath{\mathbb{R}}}
\newcommand{\ent}{\ensuremath{\mathds{Z}}}
\newcommand{\float}{\ensuremath{\mathds{R}}}
\newcommand{\bool}{\ensuremath{\mathsf{Bool}}}
\newcommand{\cha}{\ensuremath{\mathsf{Char}}}
\newcommand{\str}{\ensuremath{\mathsf{String}}}
\newcommand{\dict}[1]{\ensuremath{\mathsf{dict}\lrangle{#1}}}
\newcommand{\conj}[1]{\ensuremath{\mathsf{conj}\lrangle{#1}}}
\newcommand{\tupla}[1]{\ensuremath{\mathsf{tupla}\lrangle{#1}}}
\newcommand{\struct}[1]{\ensuremath{\mathsf{struct}\lrangle{#1}}}

% Logica

\newcommand{\True}{\ensuremath{\mathrm{true}}}
\newcommand{\False}{\ensuremath{\mathrm{false}}}
\newcommand{\Then}{\ensuremath{\rightarrow}}
\newcommand{\Iff}{\ensuremath{\leftrightarrow}}
\newcommand{\implica}{\ensuremath{\longrightarrow}}
\newcommand{\IfThenElse}[3]{\ensuremath{\mathsf{if}\ #1\ \mathsf{then}\ #2\ \mathsf{else}\ #3\ \mathsf{fi}}}
\newcommand{\If}[3]{\text{\normalfont\ttfamily IfThenElse}(\ensuremath{#1,\;#2,\;#3})}
\newcommand{\yLuego}{\land _L}
\newcommand{\oLuego}{\lor _L}
\newcommand{\thenLuego}{\Then _L}

\newcommand{\cuantificador}[5]{%
	\ensuremath{(#2 #3: #4)\ (%
		\ifthenelse{\equal{#1}{multLineas}}{
			$ % exiting math mode
				\begin{adjustwidth}{+2em}{}
					$#5$%
				\end{adjustwidth}%
			$ % entering math mode
		}{
			#5
		}
		)}
}

\newcommand{\existe}[4][]{%
	\cuantificador{#1}{\exists}{#2}{#3}{#4}
}
\newcommand{\paraTodo}[4][]{%
	\cuantificador{#1}{\forall}{#2}{#3}{#4}
}

\newcommand{\Def}[1]{\text{\normalfont\ttfamily def}(#1)}

%listas

\newcommand{\TLista}[1]{\ensuremath{seq \langle #1\rangle}}
\newcommand{\lvacia}{\ensuremath{\langle\,\rangle}}
\newcommand{\lv}{\ensuremath{\langle\,\rangle}}
\newcommand{\longitud}[1]{\ensuremath{|#1|}}
\newcommand{\cons}[1]{\ensuremath{\mathsf{addFirst}}(#1)}
\newcommand{\indice}[1]{\ensuremath{\mathsf{indice}}(#1)}
\newcommand{\conc}[1]{\ensuremath{\mathsf{concat}}(#1)}
\newcommand{\head}[1]{\ensuremath{\mathsf{head}}(#1)}
\newcommand{\tail}[1]{\ensuremath{\mathsf{tail}}(#1)}
\newcommand{\sub}[1]{\ensuremath{\mathsf{subseq}}(#1)}
\newcommand{\en}[1]{\ensuremath{\mathsf{en}}(#1)}
\newcommand{\cuenta}[2]{\mathsf{cuenta}\ensuremath{(#1, #2)}}
\newcommand{\suma}[1]{\mathsf{suma}(#1)}
\newcommand{\twodots}{\ensuremath{\mathrm{..}}}
\newcommand{\masmas}{\ensuremath{+\!+\;}}
\newcommand{\matriz}[1]{\TLista{\TLista{#1}}}
\newcommand{\seqchar}{\TLista{\cha}}
\newcommand{\lrangle}[1]{\ensuremath{\langle\text{#1}\rangle}}

%dict

\newcommand{\setKey}[1]{\ensuremath{\mathsf{setKey}(#1)}}
\newcommand{\delKey}[1]{\ensuremath{\mathsf{delKey}(#1)}}

\renewcommand{\wp}[2]{
	\ensuremath{\textit{wp}(\textbf{\lstinline{#1}}, #2)}
}

\newcommand{\hoare}[3]{\ensuremath{\{#1\} \; #2 \; \{#3\}}}

\renewcommand{\lstlistingname}{Código}
\lstset{% general command to set parameter(s)
	language=Java,
	morekeywords={func, endif, endwhile, skip, end, var, then},
	basewidth={0.47em,0.40em},
	columns=fixed, fontadjust, resetmargins, xrightmargin=5pt, xleftmargin=15pt,
	flexiblecolumns=false, tabsize=4, breaklines, breakatwhitespace=false, extendedchars=true,
	numbers=left, numberstyle=\tiny, stepnumber=1, numbersep=9pt,
	frame=l, framesep=3pt,
	captionpos=b,
}

% TADs

\newenvironment{tad}[2]{
\newcommand{\tadtype}{%
	\ensuremath{#1 \ifblank{#2}{}{\lrangle{#2}}}
}

\renewcommand{\pred}[3]{%
{\normalfont\bfseries\ttfamily pred }%
	{\normalfont\ttfamily ##1}%
\ifthenelse{\equal{##2}{}}{}{\ (##2) }%
\{%
\begin{adjustwidth}{+5em}{}
	\ensuremath{##3}
\end{adjustwidth}
\}
{\normalfont\bfseries\,\par}%
}

\vspace{1ex}
\noindent
{\normalfont\bfseries\ttfamily\large TAD #1\ifthenelse{\equal{#2}{}}{}{$<$#2$>$}} \{
% Abrimos la llave
\par%
\tocarEspacios
}{
\vspace{1ex} \par\}
}

\newcommand{\obs}[2]{
	obs #1: \ensuremath{#2}\par
}

\newenvironment{module}[4]{
\newcommand{\tadtype}{%
	\ensuremath{#3 \ifblank{#4}{}{\lrangle{#4}}}
}

\newcommand{\moduletype}{%
	\ensuremath{#1 \ifblank{#2}{}{\lrangle{#2}}}
}


\renewcommand{\pred}[3]{%
{\normalfont\bfseries\ttfamily pred }%
	{\normalfont\ttfamily ##1}%
\ifthenelse{\equal{##2}{}}{}{\ (##2) }%
\{%
\begin{adjustwidth}{+5em}{}
	\ensuremath{##3}
\end{adjustwidth}
\}
{\normalfont\bfseries\,\par}%
}

\vspace{1ex}
\noindent
{\normalfont\bfseries\ttfamily\large Módulo #1\ifthenelse{\equal{#2}{}}{}{$<$#2$>$} \ifblank{#3}{}{implementa #3\ifthenelse{\equal{#4}{}}{}{$<$#4$>$}}} \{
% Abrimos la llave
\par%
\tocarEspacios
}{
\vspace{1ex} \par\}
}

\newcommand{\var}[2]{
	var #1: \;\texto{#2}\par
}
\newcommand{\compl}[1]{Complejidad: $#1$}

% Tipos de datos de implementación
\newcommand{\Int}{\ensuremath{\mathsf{int}}}
\newcommand{\Real}{\ensuremath{\mathsf{real}}}
\newcommand{\Bool}{\ensuremath{\mathsf{bool}}}
\newcommand{\Char}{\ensuremath{\mathsf{char}}}
\newcommand{\String}{\ensuremath{\mathsf{string}}}
\newcommand{\Array}[1]{\ensuremath{\mathsf{Array<#1>}}}

\newcommand{\texto}[1]{\text{\normalfont\bfseries\ttfamily #1\;}}

\begin{document}

\titulo{Guia 7}
\materia{Algoritmos y Estructuras de Datos I}
\fecha{2do cuatrimestre 2024}

\integrante{Federico Barberón}{112/24}{jfedericobarberonj@gmail.com}
%Carátula
\maketitle
\newpage

%Indice
\tableofcontents
\newpage

\section{Guia 7}

\subsection{Ejercicio 1}
Implementamos un \textbf{Árbol Binario} (AB)

\begin{itemize}
	\item Escriba en castellano el invariante de representación para este módulo
	\item Escriba en lógica el invrep usando preds recursivos
	\item Escriba los algoritmos para los siguientes procs y, de ser posible, calcule su complejidad
	      \begin{itemize}
		      \item altura(in ab: ArbolBinario$<$T$>$): int
		      \item cantidadHojas(in ab: ArbolBinario$<$T$>$): int
		      \item está(in ab: ArbolBinario$<$T$>$, in t: T): bool
		      \item cantidadApariciones(in ab: ArbolBinario$<$T$>$, in t: T): int
	      \end{itemize}
\end{itemize}

\newcommand{\nodo}{\text{Nodo$<$T$>$}}

\nodo es \struct{dato: T, izq: Nodo, der: Nodo}

\begin{module}{ArbolBinario}{T}{Arbol Binario}{T}
	\var{raiz}{\nodo}
	\medskip
	InvRep: No tiene ciclos y la raiz es null o el subarbol derecho y el izquierdo son AB

	\pred{esAB}{r: \nodo}{
		r = null \oLuego (esAB(r.der) \land esAB(r.izq))
	}

	\pred{InvRep}{ab: \moduletype}{
		sinCiclos(ab.raiz) \land esAB(ab.raiz)
	}

	\begin{proc}{altura}{\In ab: \moduletype}{\Int}
		\compl{O(n) \leftarrow n = cantNodos}
		\begin{lstlisting}[numbers=none,frame=none]
		return altura(ab.raiz);
		\end{lstlisting}
	\end{proc}

	\begin{proc}{alturaAux}{\In r: \nodo}{\Int}
		\begin{lstlisting}[numbers=none,frame=none]
		if r == null then
			return 0;
		endif
	
		return 1 + max(alturaAux(r.izq), alturaAux(r.der));
		\end{lstlisting}
	\end{proc}

	\begin{proc}{cantidadHojas}{\In ab: \moduletype}{\Int}
		\compl{O(n)}
		\begin{lstlisting}[numbers=none,frame=none]
		return cantidadHojasAux(ab.raiz);
		\end{lstlisting}
	\end{proc}

	\begin{proc}{cantidadHojasAux}{\In r: \nodo}{\Int}
		\begin{lstlisting}[numbers=none,frame=none]
		if r == null then
			return 0;
		else if r.izq == null && r.der == null then
			return 1;
		endif

		return cantidadHojasAux(r.izq) + cantidadHojasAux(r.der);
		\end{lstlisting}
	\end{proc}

	\pagebreak

	\begin{proc}{esta}{\In ab: \moduletype}{\bool}
		\compl{O(n)}
		\begin{lstlisting}[numbers=none,frame=none]
		return estaAux(r.raiz, e);
		\end{lstlisting}
	\end{proc}

	\begin{proc}{estaAux}{\In r: \nodo, \In e: T}{\bool}
		\begin{lstlisting}[numbers=none,frame=none]
		if r == null then
			return false;
		else if r.dato == e then
			return true;
		else
			return estaAux(r.izq, e) || estaAux(r.der, e);
		endif
		\end{lstlisting}
	\end{proc}

	\begin{proc}{cantidadApariciones}{\In ab: \moduletype, \In e: T}{\Int}
		\compl{O(n)}
		\begin{lstlisting}[numbers=none,frame=none]
		return cantidadAparicionesAux(ab.raiz, e);
		\end{lstlisting}
	\end{proc}

	\begin{proc}{cantidadAparicionesAux}{\In r: \nodo, \In e: T}{\Int}
		\begin{lstlisting}[numbers=none,frame=none]
		var cant: int;
		cant := 0;

		if r == null then
			return 0;
		else if r.dato == e then
			cant := 1;
		endif
		
		return cant + cantidadAparicionesAux(r.izq, e) + cantidadAparicionesAux(r.der, e);
		\end{lstlisting}
	\end{proc}

\end{module}

\subsection{Ejercicio 2}
Un \textbf{Árbol Binario de Búsqueda} (ABB) es un árbol binario que cumple que para cualquier nodo N, todos los elementos del árbol a la izquierda son menores o iguales al valor del nodo y todos los elementos del árbol a la derecha son mayores al valor del nodo, es decir

\pred{esABB}{a: \nodo}{
	a = null \lor (\\
	\paraTodo{e}{T}{e \in elems(a.izq) \Then e \leq a.dato} \land \paraTodo{e}{T}{e \in elems(a.der) \Then e > a.dato} \land \\
	esABB(a.izq) \land esABB(a.der) \\
	)
}

\aux{elems}{a: \nodo}{\conj{T}}{
	\If{a = null}{\emptyset}{\{a.dato\} \cup elems(a.izq) \cup elems(a.der)}
}

\begin{itemize}
	\item Implemene los algoritmos para los siguientes procs y calcule su complejidad en mejor y peor caso
	      \begin{enumerate}
		      \item está(in ab: ABB$<$T$>$, in t: T): bool
		      \item cantidadApariciones(in ab: ABB$<$T$>$, in t: T): int
		      \item insertar(inout ab: ABB$<$T$>$, in t: T)
		      \item eliminar(inout ab: ABB$<$T$>$, in t: T)
		      \item inOrder(in ab: ABB$<$T$>$): Array$<$T$>$
	      \end{enumerate}

	\item Asumiendo que el árbol está balanceado, recalcule, si es necesario, las complejidads en peor caso de los algoritmos del ítem anterior
	\item ¿Qué pasa en un ABB cuando se insertan valores repetidos? Proponga una modificación del módulo que resuelva este problema
\end{itemize}

\begin{module}{ABB}{T}{Arbol Binario De Busqueda}{T}
	\var{raiz}{\nodo}
	\var{size}{\Int}

	\pred{InvRep}{ab: \moduletype}{
		esABB(ab.raiz) \land ab.size = cantNodos(ab.raiz)
	}

	\aux{cantNodos}{r: \nodo}{\Int}{
		\If{r = null}{0}{1 + cantNodos(r.izq) + cantNodos(r.der)}
	}

	\begin{proc}{nuevoABB}{}{\moduletype}
		\compl{O(1)}
		\begin{lstlisting}[numbers=none,frame=none]
		var ab: ABB < T >;
		ab := new ABB < T >;
		ab.raiz := new Nodo < T >;
		ab.size := 0;
		return ab;
		\end{lstlisting}
	\end{proc}

	\begin{proc}{esta}{\In ab: \moduletype, \In t: T}{\bool}
		Mejor caso: $\Theta(1)$\\
		Peor caso: $\Theta(n)$
		\begin{lstlisting}[numbers=none,frame=none]
		return estaAux(ab.raiz, t);
		\end{lstlisting}
	\end{proc}

	\begin{proc}{estaAux}{\In r: \nodo, \In t: T}{\bool}
		\begin{lstlisting}[numbers=none,frame=none]
		if r.dato == t then
			return true;
		else if t > r.dato then
			return estaAux(r.der, t);
		else
			return estaAux(r.izq, t);
		endif
		\end{lstlisting}
	\end{proc}

	\begin{proc}{cantidadApariciones}{\In ab: \moduletype, \In t: T}{\Int}
		Mejor caso: $\Theta(1)$ \\
		Peor caso: $\Theta(n)$
		\begin{lstlisting}[numbers=none,frame=none]
		return cantidadAparicionesAux(ab.raiz, t);
		\end{lstlisting}
	\end{proc}

	\begin{proc}{cantidadAparicionesAux}{\In r: \nodo, \In t: T}{\Int}
		\begin{lstlisting}[numbers=none,frame=none]
		if r == null then
			return 0;
		else if r.dato == t then
			return 1 + cantidadAparicionesAux(r.izq, t);
		else if r.dato > t then
			return cantidadAparicionesAux(r.der, t);
		else
			return cantidadAparicionesAux(r.izq, t);
		endif
		\end{lstlisting}
	\end{proc}

	\begin{proc}{insertar}{\Inout ab: \moduletype, \In t: T}{}
		Mejor caso: $\Theta(1)$ \\
		Peor caso: $\Theta(n)$
		\begin{lstlisting}[numbers=none,frame=none]
		ab.raiz := insertarAux(ab.raiz, t);
		ab.size := ab.size + 1;
		\end{lstlisting}
	\end{proc}

	\pagebreak

	\begin{proc}{insertarAux}{\Inout r: \nodo, \In t: T}{\nodo}
		\begin{lstlisting}[numbers=none,frame=none]
		if r == null then
			r := new Nodo < T >;
			r.dato := t;
		else if t <= r.dato then
			r.izq := insertarAux(r.izq, t);
		else
			r.der := insertarAux(r.der, t);
		endif

		return r;
		\end{lstlisting}
	\end{proc}

	\begin{proc}{eliminar}{\Inout ab: \moduletype, \In t: T}{}
		Mejor caso: $\Theta(1)$\\
		Peor caso: $\Theta(n)$
		\begin{lstlisting}[numbers=none,frame=none]
		if ab.esta(t) then
			ab.raiz := eliminarAux(ab.raiz, t);
			ab.size := ab.size - 1;
		endif
		\end{lstlisting}
	\end{proc}

	\begin{proc}{eliminarAux}{\Inout r: \nodo, \In t: T}{\nodo}
		\begin{lstlisting}[numbers=none,frame=none]
		if r == null then
			return null
		else if r.dato == t then
			if r.izq != null && r.der != null then
				r.dato = minimo(r.der);
				r.der = eliminarAux(r.der,r.dato);
			else if r.izq != null then
				return r.izq;
			else
				return r.der;
			endif
		else if t > r.dato then
			r.der := eliminarAux(r.der, t);
		else
			r.izq := eliminarAux(r.izq, t);
		endif

		return r;
		\end{lstlisting}
	\end{proc}

	\begin{proc}{minimo}{\In r: \nodo}{T}
		\begin{lstlisting}[numbers=none,frame=none]
		while r.izq != null do
			r := r.izq;
		endwhile

		return r.dato;
		\end{lstlisting}
	\end{proc}

	\begin{proc}{inOrder}{\In ab: \moduletype}{\Array{T}}
		Mejor caso: $\Theta(1)$\\
		Peor caso: $\Theta(n)$
		\begin{lstlisting}[numbers=none,frame=none]
		var cola: ColaSobreLista < T >;
		cola := colaVacia();

		inOrderAux(cola, ab.raiz);

		return colaAArray(cola, ab.size);
		\end{lstlisting}
	\end{proc}

	\begin{proc}{inOrderAux}{\Inout c: ColaSobreLista$<$T$>$,\In r: \nodo}{}
		\begin{lstlisting}[numbers=none,frame=none]
		if r == null then
			return;
		endif

		inOrderAux(c, r.izq);
		c.encolar(r.dato);
		inOrderAux(c, r.der);
		\end{lstlisting}
	\end{proc}

	\begin{proc}{colaAArray}{\Inout c: ColaSobreLista$<$T$>$, \In size: \Int}{\Array{T}}
		\begin{lstlisting}[numbers=none,frame=none]
		var res: Array < T >;
		var i: int;
		res := new Array < T >(size);
		i := 0;

		while !c.vacia() do
			res[i] := c.desencolar();
			i := i + 1;
		endwhile

		return res;
		\end{lstlisting}
	\end{proc}

	\begin{proc}{cantidadNodos}{\In ab: \moduletype}{\Int}
		\begin{lstlisting}[numbers=none,frame=none]
		return ab.size;
		\end{lstlisting}
	\end{proc}
\end{module}

Si el arbol está balanceado entonces la complejidad en el peor caso de los algoritmos está, insertar y eliminar pasa a ser $\Theta(\log{n})$
\medskip

Si se insertar valores repetidos, según el enunciado, estos se insertaran en el subarbol izquierdo del nodo con ese mismo valor (no se cual sería el problema)

\subsection{Ejercicio 3}
Implementar los siguientes TADs sobre ABB. Calcule las complejidades de los procs en mejor y peor caso
\begin{enumerate}
	\item Conjunto$<$T$>$
	\item Diccionario$<$K, V$>$
	\item ColaDePrioridad$<$T$>$
\end{enumerate}

Recalcule, si es necesario, las complejidades en peor caso de los algoritmos de los TADs considerando que se implementan sobre AVL en vez de ABB.

\begin{module}{ConjuntoABB}{T}{Conjunto}{T}
	var elems: ABB$<$T$>$

	\begin{proc}{conjVacio}{}{\moduletype}
		\compl{O(1)}
		\begin{lstlisting}[numbers=none,frame=none]
		var conj: ConjuntoABB < T >;
		conj := new ConjuntoABB < T >;
		conj.elems := nuevoABB();
		return conj;
		\end{lstlisting}
	\end{proc}

	\begin{proc}{pertenece}{\In c: \moduletype, \In e: T}{\bool}
		Mejor caso: $\Theta(1)$\\
		Peor caso: $\Theta(n)$
		\begin{lstlisting}[numbers=none,frame=none]
		return c.elems.esta(e);
		\end{lstlisting}
	\end{proc}

	\begin{proc}{agregar}{\Inout c: \moduletype, \In e: T}{}
		Mejor caso: $\Theta(1)$\\
		Peor caso: $\Theta(n)$
		\begin{lstlisting}[numbers=none,frame=none]
		if !c.elems.pertenece(e) then
			c.elems.insertar(e);
		endif
		\end{lstlisting}
	\end{proc}

	\begin{proc}{sacar}{\Inout c: \moduletype, \In e: T}{}
		Mejor caso: $\Theta(1)$\\
		Peor caso: $\Theta(n)$
		\begin{lstlisting}[numbers=none,frame=none]
		c.elems.eliminar(e);
		\end{lstlisting}
	\end{proc}

	\begin{proc}{unir}{\Inout c1: \moduletype, \In c2: \moduletype}{}
		Mejor caso: $\Theta(1)$\\
		Peor caso: $\Theta(nm)$
		\begin{lstlisting}[numbers=none,frame=none]
		var elems: Array < T >;
		var i: int;
		elems := c2.elems.inOrder();
		i := 0;

		while i < elems.length() do
			c1.elems.agregar(elems[i]);
			i := i + 1;
		endwhile
		\end{lstlisting}
	\end{proc}

	\begin{proc}{restar}{\Inout c1: \moduletype, \In c2: \moduletype}{}
		Mejor caso: $\Theta(1)$\\
		Peor caso: $\Theta(nm)$
		\begin{lstlisting}[numbers=none,frame=none]
		var elems: Array < T >;
		var i: int;
		elems := c2.elems.inOrder();
		i := 0;

		while i < elems.length() do
			c1.elems.eliminar(elems[i]);
			i := i + 1;
		endwhile
		\end{lstlisting}
	\end{proc}

	\pagebreak

	\begin{proc}{intersecar}{\Inout c1: \moduletype, \In c2: \moduletype}{}
		Mejor caso: $\Theta(1)$\\
		Peor caso: $\Theta(n^2 + nm)$
		\begin{lstlisting}[numbers=none,frame=none]
		var elems: Array < T >;
		var i: int;
		elems := c1.elems.inOrder();
		i := 0;

		while i < elems.length() do
			if !c2.elems.pertenece(elems[i]) then
				c1.elems.eliminar(elems[i]);
			endif
			i := i + 1;
		endwhile
		\end{lstlisting}
	\end{proc}

	\begin{proc}{agregarRapido}{\Inout c: \moduletype, \In e: T}{}
		Mejor caso: $\Theta(1)$\\
		Peor caso: $\Theta(n)$
		\begin{lstlisting}[numbers=none,frame=none]
		c.elems.insertar(e);
		\end{lstlisting}
	\end{proc}

	\begin{proc}{tamano}{\In c: \moduletype}{\Int}
		\compl{\Theta(1)}
		\begin{lstlisting}[numbers=none,frame=none]
		return c.elems.cantidadNodos();
		\end{lstlisting}
	\end{proc}
\end{module}

El diccionario sobre ABB tiene basicamente los mismos procs que el conjunto solo que en vez de usar ABB$<$T$>$ usamos ABB$<$\struct{clave: K, valor: V}$>$ con K un tipo comparable.

\end{document}

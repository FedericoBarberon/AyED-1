\documentclass[11pt, a4paper]{article}
\usepackage[paper=a4paper, left=1.5cm, right=1.5cm, bottom=1.5cm, top=1.5cm]{geometry}
%
\usepackage[utf8]{inputenc}
\usepackage[spanish]{babel}

\usepackage{../caratuladc/caratula/caratula}

\usepackage{multicol}
\usepackage{enumitem}
\usepackage{amsmath}
\usepackage{amssymb}
\usepackage{xcolor}

% Comandos %

\newcommand{\sii}{\leftrightarrow}
\newcommand{\hacer}{\textcolor{red}{HACER!}}
\usepackage[spanish,activeacute,es-tabla]{babel}
\usepackage[utf8]{inputenc}
\usepackage{ifthen}
\usepackage{listings}
\usepackage{dsfont}
\usepackage{subcaption}
\usepackage{amsmath}
\usepackage[top=1cm,bottom=2cm,left=1cm,right=1cm]{geometry}%
\usepackage{color}%
\usepackage{changepage}
\newcommand{\tocarEspacios}{%
	\addtolength{\leftskip}{3em}%
	\setlength{\parindent}{0em}%
}

\newcommand{\Indent}{\hspace*{0.75cm}}

% Especificacion de procs

\newcommand{\In}{\textsf{in }}
\newcommand{\Out}{\textsf{out }}
\newcommand{\Inout}{\textsf{inout }}

\newcommand{\encabezadoDeProc}[3]{%
% Ponemos la palabrita problema en tt
%  \noindent%
{\normalfont\bfseries\ttfamily proc}%
% Ponemos el nombre del problema
\ %
{\normalfont\ttfamily #1}%
\
% Ponemos los parametros
(#2)%
\ifblank{#3}{}{%
	% Por ultimo, va el tipo del resultado
	\ : #3}
}

\newenvironment{proc}[4][res]{%

% El parametro 1 (opcional) es el nombre del resultado
% El parametro 2 es el nombre del problema
% El parametro 3 son los parametros
% El parametro 4 es el tipo del resultado
% Preambulo del ambiente problema
% Tenemos que definir los comandos requiere, asegura, modifica y aux
\newcommand{\requiere}[2][]{%
{\normalfont\bfseries\ttfamily requiere\;}%
\ifthenelse{\equal{##1}{variaslineas}}{\{%
\begin{adjustwidth}{+5em}{}
	\ensuremath{##2}
\end{adjustwidth}
\}
{\normalfont\bfseries\,\par}%
}
{%
\{\ensuremath{##2}\}%
{\normalfont\bfseries\,\par}%
}
}

\newcommand{\asegura}[2][]{%
{\normalfont\bfseries\ttfamily asegura\;}%
\ifthenelse{\equal{##1}{variaslineas}}{\{%
\begin{adjustwidth}{+5em}{}
	\ensuremath{##2}
\end{adjustwidth}
\}
{\normalfont\bfseries\,\par}%
}
{%
\{\ensuremath{##2}\}%
{\normalfont\bfseries\,\par}%
}
}
\renewcommand{\aux}[4]{%
	{\normalfont\bfseries\ttfamily aux\ }%
		{\normalfont\ttfamily ##1}%
	\ifthenelse{\equal{##2}{}}{}{\ (##2)}\ : ##3\, = \ensuremath{##4}%
	{\normalfont\bfseries\,;\par}%
}
\renewcommand{\pred}[3]{%
{\normalfont\bfseries\ttfamily pred }%
	{\normalfont\ttfamily ##1}%
\ifthenelse{\equal{##2}{}}{}{\ (##2) }%
\{%
\begin{adjustwidth}{+5em}{}
	\ensuremath{##3}
\end{adjustwidth}
\}%
{\normalfont\bfseries\,\par}%
}

\newcommand{\res}{#1}
\vspace{1ex}
\noindent
\encabezadoDeProc{#2}{#3}{#4}
% Abrimos la llave
\par%
\tocarEspacios
}
{
% Cerramos la llave
\vspace{1ex}
}

\newcommand{\aux}[4]{%
	{\normalfont\bfseries\ttfamily\noindent aux\ }%
		{\normalfont\ttfamily #1}%
	\ifthenelse{\equal{#2}{}}{}{\ (#2)}\ : #3\, = \ensuremath{#4}%
	{\normalfont\bfseries\,;\par}%
}

\newcommand{\pred}[3]{%
{\normalfont\bfseries\ttfamily\noindent pred }%
	{\normalfont\ttfamily #1}%
\ifthenelse{\equal{#2}{}}{}{\ (#2) }%
\{%
\begin{adjustwidth}{+2em}{}
	\ensuremath{#3}
\end{adjustwidth}
\}%
{\normalfont\bfseries\,\par}%
}

% Tipos

\newcommand{\nat}{\ensuremath{\mathbb{N}}}
\newcommand{\reals}{\ensuremath{\mathbb{R}}}
\newcommand{\ent}{\ensuremath{\mathds{Z}}}
\newcommand{\float}{\ensuremath{\mathds{R}}}
\newcommand{\bool}{\ensuremath{\mathsf{Bool}}}
\newcommand{\cha}{\ensuremath{\mathsf{Char}}}
\newcommand{\str}{\ensuremath{\mathsf{String}}}
\newcommand{\dict}[1]{\ensuremath{\mathsf{dict}\lrangle{#1}}}
\newcommand{\conj}[1]{\ensuremath{\mathsf{conj}\lrangle{#1}}}
\newcommand{\tupla}[1]{\ensuremath{\mathsf{tupla}\lrangle{#1}}}
\newcommand{\struct}[1]{\ensuremath{\mathsf{struct}\lrangle{#1}}}

% Logica

\newcommand{\True}{\ensuremath{\mathrm{true}}}
\newcommand{\False}{\ensuremath{\mathrm{false}}}
\newcommand{\Then}{\ensuremath{\rightarrow}}
\newcommand{\Iff}{\ensuremath{\leftrightarrow}}
\newcommand{\implica}{\ensuremath{\longrightarrow}}
\newcommand{\IfThenElse}[3]{\ensuremath{\mathsf{if}\ #1\ \mathsf{then}\ #2\ \mathsf{else}\ #3\ \mathsf{fi}}}
\newcommand{\If}[3]{\text{\normalfont\ttfamily IfThenElse}(\ensuremath{#1,\;#2,\;#3})}
\newcommand{\yLuego}{\land _L}
\newcommand{\oLuego}{\lor _L}
\newcommand{\thenLuego}{\Then _L}

\newcommand{\cuantificador}[5]{%
	\ensuremath{(#2 #3: #4)\ (%
		\ifthenelse{\equal{#1}{multLineas}}{
			$ % exiting math mode
				\begin{adjustwidth}{+2em}{}
					$#5$%
				\end{adjustwidth}%
			$ % entering math mode
		}{
			#5
		}
		)}
}

\newcommand{\existe}[4][]{%
	\cuantificador{#1}{\exists}{#2}{#3}{#4}
}
\newcommand{\paraTodo}[4][]{%
	\cuantificador{#1}{\forall}{#2}{#3}{#4}
}

\newcommand{\Def}[1]{\text{\normalfont\ttfamily def}(#1)}

%listas

\newcommand{\TLista}[1]{\ensuremath{seq \langle #1\rangle}}
\newcommand{\lvacia}{\ensuremath{\langle\,\rangle}}
\newcommand{\lv}{\ensuremath{\langle\,\rangle}}
\newcommand{\longitud}[1]{\ensuremath{|#1|}}
\newcommand{\cons}[1]{\ensuremath{\mathsf{addFirst}}(#1)}
\newcommand{\indice}[1]{\ensuremath{\mathsf{indice}}(#1)}
\newcommand{\conc}[1]{\ensuremath{\mathsf{concat}}(#1)}
\newcommand{\head}[1]{\ensuremath{\mathsf{head}}(#1)}
\newcommand{\tail}[1]{\ensuremath{\mathsf{tail}}(#1)}
\newcommand{\sub}[1]{\ensuremath{\mathsf{subseq}}(#1)}
\newcommand{\en}[1]{\ensuremath{\mathsf{en}}(#1)}
\newcommand{\cuenta}[2]{\mathsf{cuenta}\ensuremath{(#1, #2)}}
\newcommand{\suma}[1]{\mathsf{suma}(#1)}
\newcommand{\twodots}{\ensuremath{\mathrm{..}}}
\newcommand{\masmas}{\ensuremath{+\!+\;}}
\newcommand{\matriz}[1]{\TLista{\TLista{#1}}}
\newcommand{\seqchar}{\TLista{\cha}}
\newcommand{\lrangle}[1]{\ensuremath{\langle\text{#1}\rangle}}

%dict

\newcommand{\setKey}[1]{\ensuremath{\mathsf{setKey}(#1)}}
\newcommand{\delKey}[1]{\ensuremath{\mathsf{delKey}(#1)}}

\renewcommand{\wp}[2]{
	\ensuremath{\textit{wp}(\textbf{\lstinline{#1}}, #2)}
}

\newcommand{\hoare}[3]{\ensuremath{\{#1\} \; #2 \; \{#3\}}}

\renewcommand{\lstlistingname}{Código}
\lstset{% general command to set parameter(s)
	language=Java,
	morekeywords={func, endif, endwhile, skip, end, var, then},
	basewidth={0.47em,0.40em},
	columns=fixed, fontadjust, resetmargins, xrightmargin=5pt, xleftmargin=15pt,
	flexiblecolumns=false, tabsize=4, breaklines, breakatwhitespace=false, extendedchars=true,
	numbers=left, numberstyle=\tiny, stepnumber=1, numbersep=9pt,
	frame=l, framesep=3pt,
	captionpos=b,
}

% TADs

\newenvironment{tad}[2]{
\newcommand{\tadtype}{%
	\ensuremath{#1 \ifblank{#2}{}{\lrangle{#2}}}
}

\renewcommand{\pred}[3]{%
{\normalfont\bfseries\ttfamily pred }%
	{\normalfont\ttfamily ##1}%
\ifthenelse{\equal{##2}{}}{}{\ (##2) }%
\{%
\begin{adjustwidth}{+5em}{}
	\ensuremath{##3}
\end{adjustwidth}
\}
{\normalfont\bfseries\,\par}%
}

\vspace{1ex}
\noindent
{\normalfont\bfseries\ttfamily\large TAD #1\ifthenelse{\equal{#2}{}}{}{$<$#2$>$}} \{
% Abrimos la llave
\par%
\tocarEspacios
}{
\vspace{1ex} \par\}
}

\newcommand{\obs}[2]{
	obs #1: \ensuremath{#2}\par
}

\newenvironment{module}[4]{
\newcommand{\tadtype}{%
	\ensuremath{#3 \ifblank{#4}{}{\lrangle{#4}}}
}

\newcommand{\moduletype}{%
	\ensuremath{#1 \ifblank{#2}{}{\lrangle{#2}}}
}


\renewcommand{\pred}[3]{%
{\normalfont\bfseries\ttfamily pred }%
	{\normalfont\ttfamily ##1}%
\ifthenelse{\equal{##2}{}}{}{\ (##2) }%
\{%
\begin{adjustwidth}{+5em}{}
	\ensuremath{##3}
\end{adjustwidth}
\}
{\normalfont\bfseries\,\par}%
}

\vspace{1ex}
\noindent
{\normalfont\bfseries\ttfamily\large Módulo #1\ifthenelse{\equal{#2}{}}{}{$<$#2$>$} \ifblank{#3}{}{implementa #3\ifthenelse{\equal{#4}{}}{}{$<$#4$>$}}} \{
% Abrimos la llave
\par%
\tocarEspacios
}{
\vspace{1ex} \par\}
}

\newcommand{\var}[2]{
	var #1: \;\texto{#2}\par
}
\newcommand{\compl}[1]{Complejidad: $#1$}

% Tipos de datos de implementación
\newcommand{\Int}{\ensuremath{\mathsf{int}}}
\newcommand{\Real}{\ensuremath{\mathsf{real}}}
\newcommand{\Bool}{\ensuremath{\mathsf{bool}}}
\newcommand{\Char}{\ensuremath{\mathsf{char}}}
\newcommand{\String}{\ensuremath{\mathsf{string}}}
\newcommand{\Array}[1]{\ensuremath{\mathsf{Array<#1>}}}

\newcommand{\texto}[1]{\text{\normalfont\bfseries\ttfamily #1\;}}

\begin{document}

\titulo{Guia 2}
\materia{Algoritmos y Estructuras de Datos I}
\fecha{2do cuatrimestre 2024}

\integrante{Federico Barberón}{112/24}{jfedericobarberonj@gmail.com}
%Carátula
\maketitle
\newpage

%Indice
\tableofcontents
\newpage

\section{Guia 2}

\subsection{Ejercicio 1}
Nombrar los siguientes predicados sobre enteros:

\begin{enumerate}[label=\alph*)]
      \item \pred{esCuadrado}{x: \ent}{\existe{c}{\ent}{c > 0 \land (c * c = x)}}
      \item \pred{esPrimo}{x: \ent}{\paraTodo{n}{\ent}{(1 < n < x) \thenLuego (x \mod n \neq 0)}}
\end{enumerate}

\subsection{Ejercicio 2}
Escriba los siguientes predicados sobre números enteros en lenguaje de especifiación:

\begin{enumerate}[label=\alph*)]
      \item Que sea verdadero si y sólo si $x$ e $y$ son coprimos.

            \pred{sonCoprimos}{x, y: \ent}{
                  \paraTodo{i}{\ent}{i > 1 \thenLuego \neg (x \mod i = 0 \land y \mod i = 0)}
            }

      \item Que sea verdadero si $y$ es el mayor primo que divide a $x$.

            \pred{mayorPrimoQueDivide}{x, y: \ent}{
                  (esPrimo(y) \yLuego x \mod y = 0) \land \neg \existe{i}{\ent}{esPrimo(i) \yLuego (x \mod i = 0 \land i > y)}
            }
\end{enumerate}

\subsection{Ejercicio 3}
Nombre los siguientes predicados auxiliares sobre secuencias de enteros:

\begin{enumerate}[label=\alph*)]
      \item \pred{todoPositivos}{s: \TLista{\ent}}{
                  \paraTodo{i}{\ent}{(0 \leq i < |s|) \thenLuego s[i] \geq 0}
            }

      \item \pred{todosDistintos}{s: \TLista{\ent}}{
                  \paraTodo{i}{\ent}{(0 \leq i < |s|) \thenLuego
                        \paraTodo{j}{\ent}{(0 \leq j < |s| \land i \neq j) \thenLuego (s[i] \neq s[j])}}
            }
\end{enumerate}

\subsection{Ejercicio 4}
Escriba los siguientes predicados auxiliares sobre secuencias de enteros, aclarando los tipos de los parámetros que recibe:

\begin{enumerate}[label=\alph*)]
      \item \textit{esPrefijo}, que determina si una secuencia es prefijo de otra.

            \pred{esPrefijo}{s1, s2: \TLista{\ent}}{
                  (|s1| \leq |s2|) \yLuego \paraTodo{i}{\ent}{
                        (0 \leq i < |s1|) \thenLuego (s1[i] = s2[i])
                  }
            }

      \item \textit{estáOrdenada}, que determina si la secuencia está ordenada de menor a mayor.

            \pred{estáOrdenada}{s: \TLista{\ent}}{
                  \paraTodo{i}{\ent}{(0 \leq i < |s| - 1) \thenLuego (s[i] \leq s[i + 1])}
            }

      \item \textit{hayUnoParQueDivideAlResto}, que determina si hay un elemento par en la secuencia que divide a todos los otros elementos de la secuencia.

            \pred{divideA}{d, n: \ent}{(d \neq 0) \yLuego n \mod d = 0}

            \pred{hayUnoParQueDivideAlResto}{s: \TLista{\ent}}{
                  \existe{i}{\ent}{(0 \leq i < |s|) \yLuego esPar(s[i]) \land \paraTodo{j}{\ent}{(0 \leq j < |s|) \yLuego divideA(s[i], s[j])}}
            }

      \item \textit{enTresPartes}, que determina si en la secuencia aparecen (de izquieda a derecha) primero 0s, despúes 1s y por último 2s. Por ejemplo $\langle 0, 0, 1, 1, 1, 1, 2 \rangle$ cumple, pero $\langle 0, 1, 3, 0 \rangle$ o $\langle 0, 0, 0, 1, 1 \rangle$ no. ¿Cómo modificaría la expresión para que se admitan cero apariciones de 0s, 1s y 2s (es decir, para que por ejemplo $\langle 0, 0, 0, 1, 1 \rangle$ o $\langle \rangle$ sí cumplan)?

            \pred{tieneSoloCeroUnoYDos}{s: \TLista{\ent}}{
                  \paraTodo{i}{\ent}{(0 \leq i < |s|) \thenLuego (s[i] = 0 \lor s[i] = 1 \lor s[i] = 2)}
            }

            \pred{enTresPartes}{s: \TLista{\ent}}{
                  tieneSoloCeroUnoYDos(s) \land estaOrdenada(s)
            }
\end{enumerate}

\end{document}

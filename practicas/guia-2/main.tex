\documentclass[11pt, a4paper]{article}
%

\usepackage{../caratuladc/caratula/caratula}

\usepackage{multicol}
\usepackage{enumitem}
\usepackage{xcolor}
\usepackage{etoolbox}

\newcommand{\hacer}{\textcolor{red}{HACER!}}
\newcommand{\punto}{\text{.}}
\usepackage[spanish,activeacute,es-tabla]{babel}
\usepackage[utf8]{inputenc}
\usepackage{ifthen}
\usepackage{listings}
\usepackage{dsfont}
\usepackage{subcaption}
\usepackage{amsmath}
\usepackage[top=1cm,bottom=2cm,left=1cm,right=1cm]{geometry}%
\usepackage{color}%
\usepackage{changepage}
\newcommand{\tocarEspacios}{%
	\addtolength{\leftskip}{3em}%
	\setlength{\parindent}{0em}%
}

\newcommand{\Indent}{\hspace*{0.75cm}}

% Especificacion de procs

\newcommand{\In}{\textsf{in }}
\newcommand{\Out}{\textsf{out }}
\newcommand{\Inout}{\textsf{inout }}

\newcommand{\encabezadoDeProc}[4]{%
% Ponemos la palabrita problema en tt
%  \noindent%
{\normalfont\bfseries\ttfamily proc}%
% Ponemos el nombre del problema
\ %
{\normalfont\ttfamily #2}%
\
% Ponemos los parametros
(#3)%
\ifblank{#4}{}{%
	% Por ultimo, va el tipo del resultado
	\ : #4}
}

\newenvironment{proc}[4][res]{%

% El parametro 1 (opcional) es el nombre del resultado
% El parametro 2 es el nombre del problema
% El parametro 3 son los parametros
% El parametro 4 es el tipo del resultado
% Preambulo del ambiente problema
% Tenemos que definir los comandos requiere, asegura, modifica y aux
\newcommand{\requiere}[2][]{%
{\normalfont\bfseries\ttfamily requiere\;}%
\ifthenelse{\equal{##1}{variaslineas}}{\{%
\begin{adjustwidth}{+5em}{}
	\ensuremath{##2}
\end{adjustwidth}
\}
{\normalfont\bfseries\,\par}%
}
{%
\{\ensuremath{##2}\}%
{\normalfont\bfseries\,\par}%
}
}

\newcommand{\asegura}[2][]{%
{\normalfont\bfseries\ttfamily asegura\;}%
\ifthenelse{\equal{##1}{variaslineas}}{\{%
\begin{adjustwidth}{+5em}{}
	\ensuremath{##2}
\end{adjustwidth}
\}
{\normalfont\bfseries\,\par}%
}
{%
\{\ensuremath{##2}\}%
{\normalfont\bfseries\,\par}%
}
}
\renewcommand{\aux}[4]{%
	{\normalfont\bfseries\ttfamily aux\ }%
		{\normalfont\ttfamily ##1}%
	\ifthenelse{\equal{##2}{}}{}{\ (##2)}\ : ##3\, = \ensuremath{##4}%
	{\normalfont\bfseries\,;\par}%
}
\renewcommand{\pred}[3]{%
{\normalfont\bfseries\ttfamily pred }%
	{\normalfont\ttfamily ##1}%
\ifthenelse{\equal{##2}{}}{}{\ (##2) }%
\{%
\begin{adjustwidth}{+5em}{}
	\ensuremath{##3}
\end{adjustwidth}
\}%
{\normalfont\bfseries\,\par}%
}

\newcommand{\res}{#1}
\vspace{1ex}
\noindent
\encabezadoDeProc{#1}{#2}{#3}{#4}
% Abrimos la llave
\par%
\tocarEspacios
}
{
% Cerramos la llave
\vspace{1ex}
}

\newcommand{\aux}[4]{%
	{\normalfont\bfseries\ttfamily\noindent aux\ }%
		{\normalfont\ttfamily #1}%
	\ifthenelse{\equal{#2}{}}{}{\ (#2)}\ : #3\, = \ensuremath{#4}%
	{\normalfont\bfseries\,;\par}%
}

\newcommand{\pred}[3]{%
{\normalfont\bfseries\ttfamily\noindent pred }%
	{\normalfont\ttfamily #1}%
\ifthenelse{\equal{#2}{}}{}{\ (#2) }%
\{%
\begin{adjustwidth}{+2em}{}
	\ensuremath{#3}
\end{adjustwidth}
\}%
{\normalfont\bfseries\,\par}%
}

% Tipos

\newcommand{\nat}{\ensuremath{\mathbb{N}}}
\newcommand{\reals}{\ensuremath{\mathbb{R}}}
\newcommand{\ent}{\ensuremath{\mathds{Z}}}
\newcommand{\float}{\ensuremath{\mathds{R}}}
\newcommand{\bool}{\ensuremath{\mathsf{Bool}}}
\newcommand{\cha}{\ensuremath{\mathsf{Char}}}
\newcommand{\str}{\ensuremath{\mathsf{String}}}
\newcommand{\dict}[1]{\ensuremath{\mathsf{dict}\lrangle{#1}}}
\newcommand{\conj}[1]{\ensuremath{\mathsf{conj}\lrangle{#1}}}
\newcommand{\tupla}[1]{\ensuremath{\mathsf{tupla}\lrangle{#1}}}
\newcommand{\struct}[1]{\ensuremath{\mathsf{struct}\lrangle{#1}}}

% Logica

\newcommand{\True}{\ensuremath{\mathrm{true}}}
\newcommand{\False}{\ensuremath{\mathrm{false}}}
\newcommand{\Then}{\ensuremath{\rightarrow}}
\newcommand{\Iff}{\ensuremath{\leftrightarrow}}
\newcommand{\implica}{\ensuremath{\longrightarrow}}
\newcommand{\IfThenElse}[3]{\ensuremath{\mathsf{if}\ #1\ \mathsf{then}\ #2\ \mathsf{else}\ #3\ \mathsf{fi}}}
\newcommand{\If}[3]{\text{\normalfont\ttfamily IfThenElse}(\ensuremath{#1,\;#2,\;#3})}
\newcommand{\yLuego}{\land _L}
\newcommand{\oLuego}{\lor _L}
\newcommand{\thenLuego}{\Then _L}

\newcommand{\cuantificador}[5]{%
	\ensuremath{(#2 #3: #4)\ (%
		\ifthenelse{\equal{#1}{multLineas}}{
			$ % exiting math mode
				\begin{adjustwidth}{+2em}{}
					$#5$%
				\end{adjustwidth}%
			$ % entering math mode
		}{
			#5
		}
		)}
}

\newcommand{\existe}[4][]{%
	\cuantificador{#1}{\exists}{#2}{#3}{#4}
}
\newcommand{\paraTodo}[4][]{%
	\cuantificador{#1}{\forall}{#2}{#3}{#4}
}

\newcommand{\Def}[1]{\text{\normalfont\ttfamily def}(#1)}

%listas

\newcommand{\TLista}[1]{\ensuremath{seq \langle #1\rangle}}
\newcommand{\lvacia}{\ensuremath{\langle\,\rangle}}
\newcommand{\lv}{\ensuremath{\langle\,\rangle}}
\newcommand{\longitud}[1]{\ensuremath{|#1|}}
\newcommand{\cons}[1]{\ensuremath{\mathsf{addFirst}}(#1)}
\newcommand{\indice}[1]{\ensuremath{\mathsf{indice}}(#1)}
\newcommand{\conc}[1]{\ensuremath{\mathsf{concat}}(#1)}
\newcommand{\head}[1]{\ensuremath{\mathsf{head}}(#1)}
\newcommand{\tail}[1]{\ensuremath{\mathsf{tail}}(#1)}
\newcommand{\sub}[1]{\ensuremath{\mathsf{subseq}}(#1)}
\newcommand{\en}[1]{\ensuremath{\mathsf{en}}(#1)}
\newcommand{\cuenta}[2]{\mathsf{cuenta}\ensuremath{(#1, #2)}}
\newcommand{\suma}[1]{\mathsf{suma}(#1)}
\newcommand{\twodots}{\ensuremath{\mathrm{..}}}
\newcommand{\masmas}{\ensuremath{+\!+\;}}
\newcommand{\matriz}[1]{\TLista{\TLista{#1}}}
\newcommand{\seqchar}{\TLista{\cha}}
\newcommand{\lrangle}[1]{\ensuremath{\langle#1\rangle}}

%dict

\newcommand{\setKey}[1]{\ensuremath{\mathsf{setKey}(#1)}}
\newcommand{\delKey}[1]{\ensuremath{\mathsf{delKey}(#1)}}

\renewcommand{\wp}[2]{
	\ensuremath{\textit{wp}(\textbf{\lstinline{#1}}, #2)}
}

\newcommand{\hoare}[3]{\ensuremath{\{#1\} \; #2 \; \{#3\}}}

\renewcommand{\lstlistingname}{Código}
\lstset{% general command to set parameter(s)
	language=Java,
	morekeywords={func, endif, endwhile, skip, end, var, then},
	basewidth={0.47em,0.40em},
	columns=fixed, fontadjust, resetmargins, xrightmargin=5pt, xleftmargin=15pt,
	flexiblecolumns=false, tabsize=4, breaklines, breakatwhitespace=false, extendedchars=true,
	numbers=left, numberstyle=\tiny, stepnumber=1, numbersep=9pt,
	frame=l, framesep=3pt,
	captionpos=b,
}

% TADs

\newenvironment{tad}[2]{
\newcommand{\tadtype}{%
	\ensuremath{#1 \ifblank{#2}{}{\lrangle{#2}}}
}

\renewcommand{\pred}[3]{%
{\normalfont\bfseries\ttfamily pred }%
	{\normalfont\ttfamily ##1}%
\ifthenelse{\equal{##2}{}}{}{\ (##2) }%
\{%
\begin{adjustwidth}{+5em}{}
	\ensuremath{##3}
\end{adjustwidth}
\}
{\normalfont\bfseries\,\par}%
}

\vspace{1ex}
\noindent
{\normalfont\bfseries\ttfamily\large TAD #1\ifthenelse{\equal{#2}{}}{}{$<$#2$>$}} \{
% Abrimos la llave
\par%
\tocarEspacios
}{
\vspace{1ex} \par\}
}

\newcommand{\obs}[2]{
	obs #1: \ensuremath{#2}\par
}

\newenvironment{module}[4]{
\newcommand{\tadtype}{%
	\ensuremath{#3 \ifblank{#4}{}{\lrangle{#4}}}
}

\newcommand{\moduletype}{%
	\ensuremath{#1 \ifblank{#2}{}{\lrangle{#2}}}
}


\renewcommand{\pred}[3]{%
{\normalfont\bfseries\ttfamily pred }%
	{\normalfont\ttfamily ##1}%
\ifthenelse{\equal{##2}{}}{}{\ (##2) }%
\{%
\begin{adjustwidth}{+5em}{}
	\ensuremath{##3}
\end{adjustwidth}
\}
{\normalfont\bfseries\,\par}%
}

\vspace{1ex}
\noindent
{\normalfont\bfseries\ttfamily\large Módulo #1\ifthenelse{\equal{#2}{}}{}{$<$#2$>$} implementa #3\ifthenelse{\equal{#4}{}}{}{$<$#4$>$}} \{
% Abrimos la llave
\par%
\tocarEspacios
}{
\vspace{1ex} \par\}
}

\newcommand{\var}[2]{
	var #1: \;\ensuremath{#2}\par
}
\newcommand{\compl}[1]{Complejidad: $#1$}

% Tipos de datos de implementación
\newcommand{\Int}{\ensuremath{\mathsf{int}}}
\newcommand{\Real}{\ensuremath{\mathsf{real}}}
\newcommand{\Bool}{\ensuremath{\mathsf{bool}}}
\newcommand{\Char}{\ensuremath{\mathsf{char}}}
\newcommand{\String}{\ensuremath{\mathsf{string}}}
\newcommand{\Array}[1]{\ensuremath{\mathsf{Array<#1>}}}

\begin{document}

\titulo{Guia 2}
\materia{Algoritmos y Estructuras de Datos I}
\fecha{2do cuatrimestre 2024}

\integrante{Federico Barberón}{112/24}{jfedericobarberonj@gmail.com}
%Carátula
\maketitle
\newpage

%Indice
\tableofcontents
\newpage

\section{Guia 2}

\subsection{Ejercicio 1}
Nombrar los siguientes predicados sobre enteros:

\begin{enumerate}[label=\alph*)]
      \item \pred{esCuadrado}{x: \ent}{\existe{c}{\ent}{c > 0 \land (c * c = x)}}
      \item \pred{esPrimo}{x: \ent}{\paraTodo{n}{\ent}{(1 < n < x) \thenLuego (x \mod n \neq 0)}}
\end{enumerate}

\subsection{Ejercicio 2}
Escriba los siguientes predicados sobre números enteros en lenguaje de especifiación:

\begin{enumerate}[label=\alph*)]
      \item Que sea verdadero si y sólo si $x$ e $y$ son coprimos.

            \pred{sonCoprimos}{x, y: \ent}{
                  \paraTodo{i}{\ent}{i > 1 \thenLuego \neg (x \mod i = 0 \land y \mod i = 0)}
            }

      \item Que sea verdadero si $y$ es el mayor primo que divide a $x$.

            \pred{mayorPrimoQueDivide}{x, y: \ent}{
                  (esPrimo(y) \yLuego x \mod y = 0) \land \neg \existe{i}{\ent}{esPrimo(i) \yLuego (x \mod i = 0 \land i > y)}
            }
\end{enumerate}

\subsection{Ejercicio 3}
Nombre los siguientes predicados auxiliares sobre secuencias de enteros:

\begin{enumerate}[label=\alph*)]
      \item \pred{todoPositivos}{s: \TLista{\ent}}{
                  \paraTodo{i}{\ent}{(0 \leq i < |s|) \thenLuego s[i] \geq 0}
            }

      \item \pred{todosDistintos}{s: \TLista{\ent}}{
                  \paraTodo{i}{\ent}{(0 \leq i < |s|) \thenLuego
                        \paraTodo{j}{\ent}{(0 \leq j < |s| \land i \neq j) \thenLuego (s[i] \neq s[j])}}
            }
\end{enumerate}

\subsection{Ejercicio 4}
Escriba los siguientes predicados auxiliares sobre secuencias de enteros, aclarando los tipos de los parámetros que recibe:

\begin{enumerate}[label=\alph*)]
      \item \textit{esPrefijo}, que determina si una secuencia es prefijo de otra.

            \pred{esPrefijo}{s1, s2: \TLista{\ent}}{
                  (|s1| \leq |s2|) \yLuego \paraTodo{i}{\ent}{
                        (0 \leq i < |s1|) \thenLuego (s1[i] = s2[i])
                  }
            }

      \item \textit{estáOrdenada}, que determina si la secuencia está ordenada de menor a mayor.

            \pred{estáOrdenada}{s: \TLista{\ent}}{
                  \paraTodo{i}{\ent}{(0 \leq i < |s| - 1) \thenLuego (s[i] \leq s[i + 1])}
            }

      \item \textit{hayUnoParQueDivideAlResto}, que determina si hay un elemento par en la secuencia que divide a todos los otros elementos de la secuencia.

            \pred{divideA}{d, n: \ent}{(d \neq 0) \yLuego n \mod d = 0}

            \pred{hayUnoParQueDivideAlResto}{s: \TLista{\ent}}{
                  \existe{i}{\ent}{(0 \leq i < |s|) \yLuego esPar(s[i]) \land \paraTodo{j}{\ent}{(0 \leq j < |s|) \yLuego divideA(s[i], s[j])}}
            }

      \item \textit{enTresPartes}, que determina si en la secuencia aparecen (de izquieda a derecha) primero 0s, despúes 1s y por último 2s. Por ejemplo $\langle 0, 0, 1, 1, 1, 1, 2 \rangle$ cumple, pero $\langle 0, 1, 3, 0 \rangle$ o $\langle 0, 0, 0, 1, 1 \rangle$ no. ¿Cómo modificaría la expresión para que se admitan cero apariciones de 0s, 1s y 2s (es decir, para que por ejemplo $\langle 0, 0, 0, 1, 1 \rangle$ o $\langle \rangle$ sí cumplan)?

            \pred{tieneSoloCeroUnoYDos}{s: \TLista{\ent}}{
                  \paraTodo{i}{\ent}{(0 \leq i < |s|) \thenLuego (s[i] = 0 \lor s[i] = 1 \lor s[i] = 2)}
            }

            \pred{enTresPartes}{s: \TLista{\ent}}{
                  tieneSoloCeroUnoYDos(s) \land estaOrdenada(s)
            }
\end{enumerate}

\subsection{Ejercicio 5}
Sea $s$ una secuencia de elementos de tipo $\ent$. Escribir una expresión (utilizando sumatoria y productoria) tal que:

\begin{enumerate}[label=\alph*)]
      \item Cuente la cantidad de veces que aparece el elemento $e$ de tipo $\ent$ en la secuencia $s$.

            $\sum\limits_{i=0}\limits^{|s| - 1} \IfThenElse{s[i] = e}{1}{0}$

      \item Sume los elementos en las posiciones impares de la secuencia $s$.

            $\sum\limits_{i=0}\limits^{|s| - 1} \IfThenElse{\neg esPar(s[i])}{s[i]}{0}$

      \item Sume los elementos mayores a 0 contenidos en la secuencia $s$.

            $\sum\limits_{i=0}\limits^{|s| - 1} \IfThenElse{s[i] > 0}{s[i]}{0}$

      \item Sume los inverso multiplicativos $(\frac{1}{x})$ de los elementos contenidos en la secuencia $s$ distintos a 0.

            $\sum\limits_{i=0}\limits^{|s| - 1} \IfThenElse{s[i] \neq 0}{\frac{1}{s[i]}}{0}$
\end{enumerate}

\subsection{Ejercicio 6}
Las siguientes especifiaciones no son correctas. Indicar por qué y corregirlas para que describan correctamente el problema.

\begin{enumerate}[label=\alph*)]
      \item progresionGeometricaFactor2: Indica si la secuencia $l$ representa una progresión geométrica factor 2. Es decir, si cada elemento de la secuencia es el doble del elemento anterior.

            \begin{proc}{progresionGeometricaFactor2}{\In l: \TLista{\ent}}{\bool}
                  \requiere{True}
                  \asegura{res = True \Iff (\paraTodo{i}{\ent}{0 \leq i < |l| \thenLuego l[i] = 2 * l[i - 1]})}
            \end{proc}

            El asegura se indefine con $i = 0$ pues trata de acceder a $l[-1]$.

            Solucion: \begin{proc}{progresionGeometricaFactor2}{\In l: \TLista{\ent}}{\bool}
                  \requiere{True}
                  \asegura{res = True \Iff (\paraTodo{i}{\ent}{0 < i < |l| \thenLuego l[i] = 2 * l[i - 1]})}
            \end{proc}

      \item minimo: Devuelve en $res$ el menor elemento de $l$.

            \begin{proc}{minimo}{\In l: \TLista{\ent}}{\ent}
                  \requiere{True}
                  \asegura{\paraTodo{y}{\ent}{(y \in l \land y \neq x) \Then y > res}}
            \end{proc}

            En el asegura se hace referencia a $x$ que no está definida. La lista no puede estar vacía y $res$ tiene que estar en la lista.

            Solución: \begin{proc}{minimo}{\In l: \TLista{\ent}}{\ent}
                  \requiere{|l| > 0}
                  \asegura{res \in l \land \paraTodo{y}{\ent}{(y \in l \land y \neq res) \Then y > res}}
            \end{proc}
\end{enumerate}

\subsection{Ejercicio 7}
Para los siguientes problemas, dar todas las soluciones posibles a las entradas dadas:

\begin{enumerate}[label=\alph*)]
      \item \begin{proc}{indiceDelMaximo}{\In l: \TLista{\float}}{\ent}
                  \requiere{|l| > 0}
                  \asegura{0 \leq res < |l| \yLuego (\paraTodo{i}{\ent}{0 \leq i < |l| \thenLuego l[i] \leq l[res]})}
            \end{proc}

            \begin{enumerate}[label=\roman*)]
                  \item $\langle 1, 2, 3, 4 \rangle$ $res = 3$
                  \item $\langle 15 \punto 5, -18, 4 \punto 215, 15 \punto 5, -1 \rangle$ $res = 0 \lor res = 3$
                  \item $\langle 0, 0, 0, 0, 0, 0 \rangle$ $res \in [0, |l|)$
            \end{enumerate}

      \item \begin{proc}{indiceDelPrimerMaximo}{\In l: \TLista{\float}}{\ent}
                  \requiere{|l| > 0}
                  \asegura{0 \leq res < |l| \yLuego (\paraTodo{i}{\ent}{0 \leq i < |l| \thenLuego (l[i] \leq l[res] \lor (l[i] = l[res] \land i >= res))})}
            \end{proc}

            \begin{enumerate}[label=\roman*)]
                  \item $\langle 1, 2, 3, 4 \rangle$ $res = 3$
                  \item $\langle 15 \punto 5, -18, 4 \punto 215, 15 \punto 5, -1 \rangle$ $res = 0$
                  \item $\langle 0, 0, 0, 0, 0, 0 \rangle$ $res = 0$
            \end{enumerate}

      \item ¿Para qué valores de entrada indiceDelPrimerMaximo y indiceDelMaximo tienen necesariamente la misma salida?

            Tienen la misma salida sii el máximo de la lista no está repetido.
\end{enumerate}

\subsection{Ejercicio 8}
Sea $f: \float  \times \float  \to \float$ definida como:

\[
      f(a, b) = \begin{cases}
            2b    & \text{si } a < 0    \\
            b - 1 & \text{en otro caso}
      \end{cases}
\]

Indicar cuáles de las siguientes especifiaciones son correctas para el problema de calcular $f(a, b)$. Para aquellas que no lo son, indicar por qué.

\begin{enumerate}[label=\alph*)]
      \item \begin{proc}{f}{\In a, b: \float}{\float}
                  \requiere{True}
                  \asegura{(a < 0 \land res = 2 * b) \land (a \geq 0 \land res = b - 1)}
            \end{proc}

            Falla pues el asegura es siempre falso ya que, por la asociatividad, se tiene que $(a < 0 \land a \geq 0 \land \dots)$ lo cual es siempre falso.

      \item \begin{proc}{f}{\In a, b: \float}{\float}
                  \requiere{True}
                  \asegura{(a < 0 \land res = 2 * b) \lor (a \geq 0 \land res = b - 1)}
            \end{proc}

            Es correcta.

      \item \begin{proc}{f}{\In a, b: \float}{\float}
                  \requiere{True}
                  \asegura{(a < 0 \Then res = 2 * b) \lor (a \geq 0 \Then res = b - 1)}
            \end{proc}

            Falla pues siempre es verdadera.

      \item \begin{proc}{f}{\In a, b: \float}{\float}
                  \requiere{True}
                  \asegura{res = \If{a < 0}{2 * b}{b - 1}}
            \end{proc}

            Es correcta.
\end{enumerate}

\subsection{Ejercicio 9}
Considerar la siguiente especificación, junto con un algoritmo que dado $x$ devuelve $x^2$.

\begin{proc}{unoMasGrande}{\In x: \float}{\float}
      \requiere{True}
      \asegura{res > x}
\end{proc}

\begin{enumerate}[label=\alph*)]
      \item ¿Qué devuelve el algoritmo si recibe $x = 3$? ¿El resultado hace verdadera la posrtcondición de unoMasGrande?

            El algoritmo devuelve 9 y hace verdadera la postcondición pues 9 > 3.

      \item ¿Qué sucede para las entradas $x = 0\punto5, x = 1, x = -0\punto2$ y $x = -7$?

            Para los primeros 2 falla y para los útlimos 2 cumple.

      \item Teniendo en cuenta lo respondido en los puntos anteriores, escribir una \textbf{precondición} para unoMasGrande, de manera tal que el algoritmo cumpla con la especifiación.

            requiere \{$|x| > 1$\}
\end{enumerate}

\subsection{Ejercicio 10}
Sean $x$ y $r$ variables de tipo \float. Considerar los siguientes predicados:

\begin{multicols}{2}
      \begin{itemize}
            \item P1: \{$x \leq 0$\}
            \item P2: \{$x \leq 10$\}
            \item P3: \{$x \leq -10$\}
            \item Q1: \{$r \geq x^2$\}
            \item Q2: \{$r \geq 0$\}
            \item Q3: \{$r = x^2$\}
      \end{itemize}
\end{multicols}

\begin{enumerate}[label=\alph*)]
      \item Indicar la relación de fuerza entre P1, P2 y P3

            P1 es más fuerte que P2

            P3 es más fuerte que P1 y P2

      \item Indicar la relación de fuerza entre Q1, Q2 y Q3

            Q1 es más fuerte que Q2

            Q3 es más fuerte que Q2 y Q1

      \item Escribir 2 programas que cumplan con la siguiente especifiación:

            \begin{proc}{hagoAlgo}{\In x: \float}{\float}
                  \requiere{x \leq 0}
                  \asegura{res \geq x^2}
            \end{proc}

            $S1 \equiv res := x^2$

            $S2 \equiv res := x^2 + 1$

      \item Sea A un algoritmo que cumple con la especifiación del ítme anterior. Decidir si necesariamente cumple las siguientes especificaciones:

            \begin{enumerate}
                  \item requiere \{$x \leq -10$\}, asegura \{$r \geq x^2$\} Cumple.
                  \item requiere \{$x \leq 10$\}, asegura \{$r \geq x^2$\} No Cumple.
                  \item requiere \{$x \leq 0$\}, asegura \{$r \geq 0$\} Cumple.
                  \item requiere \{$x \leq 0$\}, asegura \{$r = 0$\} No Cumple.
                  \item requiere \{$x \leq -10$\}, asegura \{$r \geq 0$\} Cumple.
                  \item requiere \{$x \leq 10$\}, asegura \{$r = 0$\} No Cumple.
            \end{enumerate}

      \item ¿Qué conclusión pueden sacar? ¿Qué debe cumplirse con respecto a las precondiciones y postcondiciones para que sea seguro \textbf{reemplazar la especificación}?

            Se concluye que para poder reemplazar una especificación por otra debe ocurrir que la nueva precondición sea igual o más fuerte que la anterior, y la postcondición sea igual o más débil que la anterior.
\end{enumerate}

\subsection{Ejercicio 11}
Considerar las siguientes dos especificaciones, junto con un algoritmo $a$ que satisface la especificación de p2.

\begin{proc}{p1}{\In x: \float, \In n: \ent}{\ent}
      \requiere{x \neq 0}
      \asegura{x^n - 1 < res \leq x^n}
\end{proc}

\begin{proc}{p2}{\In x: \float, \In n: \ent}{\ent}
      \requiere{n \leq 0 \Then x \neq 0}
      \asegura{res = \lfloor x^n \rfloor }
\end{proc}

\begin{enumerate}[label=\alph*)]
      \item Dados valores de $x$ y $n$ que hacen verdadera la precondición de p1, demostrar que hacen también verdadera la precondición de p2.

            Basta probar que $x \neq 0 \Then (n \leq 0 \Then x \neq 0)$

            \begin{align*}
                  x \neq 0 \Then (n \leq 0 \Then x \neq 0) \\
                  x = 0 \lor (n > 0 \lor x \neq 0)         \\
                  (x = 0 \lor x \neq 0) \lor n > 0         \\
                  True \lor n > 0                          \\
                  True
            \end{align*}

            Por lo tanto, los valores que cumplen la precondición de p1 cumplen la precondición de p2.

      \item Ahora, dados estos valores de $x$ y $n$, supongamos que se ejecuta $a$: llegamos a un valor de $res$ que hace verdadera la postcondición de p2. ¿Será también verdadera la postcondición de p1 con este valor de $res$?

            Si $res$ satisface la postcondición de p2 esto es que $res = \lfloor x^n \rfloor$. Para ver si satisface la precondición de p1 hay que ver si es verdadero que $x^n - 1 < \lfloor x^n \rfloor \leq x^n$, lo cuál es trivial. Por lo tanto satisface ambas postcondiciones.

      \item ¿Podemos concluir que $a$ satisface la especificación de p1?

            Si. Como cualquier valor que cumple la precondición de p1, cumple a su vez la precondición de p2, entonces como $a$ satisface la especificación de p2, ejecutar $a$ con cualquier valor que satisfaga la precondiciónde p1 devolverá un valor que satisface ambas postcondiciones, en particular la de p1, por lo tanto, $a$ satisface la especifiación de p1.
\end{enumerate}

\subsection{Ejercicio 12}
Especificar los siguientes problemas:

\begin{enumerate}[label=\alph*]
      \item Dado un entero, decidir si es par

            \begin{proc}{esPar}{\In n: \ent}{\bool}
                  \requiere{True}
                  \asegura{res = \True \Iff n \mod 2 = 0}
            \end{proc}

      \item Dado un entero $n$ y otro $m$, decidir si $n$ es un múltiplo de $m$.

            \begin{proc}{esMultiplo}{\In n,m: \ent}{\bool}
                  \requiere{True}
                  \asegura{res = \True \Iff \existe{i}{\ent}{n = i * m}}
            \end{proc}

      \item Dado un entero, listar todos sus divisores positivos (sin duplicados)

            \begin{proc}{divisoresPos}{\In n: \ent}{\TLista{\ent}}
                  \requiere{True}
                  \asegura{sinDuplicados(res) \land sonDivisoresPos(n, res) \land noHayOtroDivisorPos(n, res)}

                  \pred{sonDivisoresPos}{n: \ent, divsPos: \TLista{\ent}}{
                        \paraTodo{i}{\ent}{i \in divsPos \thenLuego i > 0 \land divideA(i, n)}
                  }

                  \pred{noHayOtroDivisorPos}{n: \ent, divsPos: \TLista{\ent}}{
                        \lnot \existe{i}{\ent}{i > 0 \land i \notin divsPos \land divideA(i, n)}
                  }

                  \pred{sinDuplicados}{list: \TLista{\ent}}{
                        \paraTodo{i}{\ent}{
                              i \in list \thenLuego cantApariciones(i, list) = 1
                        }
                  }

                  \pred{divideA}{d, n: \ent}{
                        \existe{k}{\ent}{n = d * k}
                  }

                  \aux{cantApariciones}{e: \ent, list: \TLista{\ent}}{\ent}{
                        \sum\limits_{i = 0}\limits^{|list| - 1} \If{list[i] = e}{1}{0}
                  }
            \end{proc}

      \item Dado un entero positivo, obtener su descomposición en factores primos. Devolver una secuencia de tuplas $(p, e)$, donde $p$ es un primo y $e$ es su exponente, ordenada en forma creciente con respecto a $p$.

            \begin{proc}{descomposicionEnPrimos}{\In n: \ent}{\TLista{\ent \times \nat}}
                  \requiere{n \geq 0}
                  \asegura{estaOrdenada(res) \land \\
                        \paraTodo{i}{\ent}{0 \leq i < |res| \thenLuego esPrimo(res[i]_0)} \land \\
                        \left(\displaystyle\prod_{i = 0}^{|res| - 1} res[i]_0^{res[i]_1}\right) = n}

                  \pred{estaOrdenada}{list: \TLista{\ent \times \nat}}{
                        \paraTodo{i}{\ent}{0 \leq i < |list| - 1 \thenLuego list[i]_0 <= list[i + 1]_0 }
                  }

                  \pred{esPrimo}{n: \ent}{
                        \paraTodo{i}{\ent}{(i > 0 \yLuego n \mod i = 0) \thenLuego (i = 1 \lor i = n)}
                  }
            \end{proc}
\end{enumerate}

\subsection{Ejercicio 13}
Especificar los siguientes problemas sobre secuencias:

\begin{enumerate}[label=\alph*)]
      \item Dados dos secuencias $s$ y $t$, decidir si $s$ está \textit{incluida} en $t$, es decir, si todos los elementos de $s$ aparecen en $t$ en igual o mayor cantidad.

            \begin{proc}{estaIncluida}{\In s, t: \TLista{\ent}}{\bool}
                  \requiere{True}
                  \asegura{res = \True \Iff \paraTodo{i}{\ent}{i \in s \Then cantApariciones(i, s) \leq cantApariciones(i, t)}}

                  \aux{cantApariciones}{e: \ent, s: \TLista{\ent}}{\ent}{
                        \displaystyle\sum_{i = 0}^{|s| - 1} \If{s[i] = e}{1}{0}
                  }
            \end{proc}

      \item Dadas dos secuencias $s$ y $t$, devolver su \textit{intersección}, es decir, una secuencia con todos los elementos que aparecen en ambas. Si un mismo elemento tiene repetidos, la secuencia retornada debe contener la cantidad mínima de apariciones del elemento en $s$ y $t$.

            \begin{proc}{interseccion}{\In s, t: \TLista{\ent}}{\TLista{\ent}}
                  \requiere{True}
                  \asegura{todoElementoPerteneceAAmbas(res, s, t) \land \\
                        todoElementoConLaMinimaAparicion(res, s, t)}

                  \pred{todoElementoPerteneceAAmbas}{list, s, t: \TLista{\ent}}{
                        \paraTodo{n}{\ent}{n \in list \Iff (n \in s \land n \in t)}
                  }

                  \pred{todoElementoConLaMinimaAparicion}{list, s, t: \TLista{\ent}}{
                        \paraTodo[multLineas]{n}{\ent}{n \in list \Then cantApariciones(n, list) = min(cantApariciones(n, s), cantApariciones(n, t))}
                  }

                  \aux{cantApariciones}{e: \ent, s: \TLista{\ent}}{\ent}{
                        \displaystyle\sum_{i = 0}^{|s| - 1} \If{s[i] = e}{1}{0}
                  }
            \end{proc}

      \item Dada una secuencia de números enteros, devolver aquel que divida a más elementos de la secuencia. Ele elemento tiene que pertenecer a la secuencia original. Si existe más de un elemento que cumple esta propiedad, devolver alguno de ellos.

            \begin{proc}{elQueMasDivide}{\In s: \TLista{\ent}}{\ent}
                  \requiere{True}
                  \asegura{res \in s \land \paraTodo[multLineas]{n}{\ent}{n \in s \Then cantidadDeMultiplosEnLista(n, s) \leq cantidadDeMultiplosEnLista(res, s)}}

                  \pred{divideA}{d, n: \ent}{
                        \existe{k}{\ent}{n = d * k}
                  }

                  \aux{cantidadDeMultiplosEnLista}{e: \ent, s: \TLista{\ent}}{\ent}{
                        \displaystyle\sum_{i = 0}^{|s| - 1} \If{divideA(e, s[i])}{1}{0}
                  }
            \end{proc}

      \item Dada una secuencia de secuencias de enteros $l$, devolver una secuencia de $l$ que contenga el máximo valor. Por ejemplo, si $l = \langle \langle 2, 3, 5 \rangle, \langle 8, 1 \rangle, \langle 2, 8, 4, 3 \rangle \rangle$, devolver $\langle 8, 1 \rangle$ o $\langle 2, 8, 4, 3 \rangle$

            \begin{proc}{secuenciaConMax}{\In s: \TLista{\TLista{\ent}}}{\TLista{\ent}}
                  \requiere{True}
                  \asegura{res \in s \land \\
                        \existe{max}{\ent}{max \in res \land \paraTodo{i}{\ent}{0 \leq i < |s| \thenLuego \paraTodo{n}{\ent}{n \in s[i] \Then n \leq max}}}
                  }
            \end{proc}

      \item Dada una secuencia $l$ con todos sus elementos distintos, devolver la secuencia de $partes$, es decir, la secuencia de todas las secuencias incluidas en $l$, cada una con sus elementos en el mismo orden en que aparecen en $l$.

            \begin{proc}{partes}{\In l: \TLista{\ent}}{\TLista{\TLista{\ent}}}
                  \requiere{todosDistintos(l)}
                  \asegura{} \hacer

                  \pred{todosDistintos}{l: \TLista{\ent}}{
                        \paraTodo{n}{\ent}{n \in l \Then cantApariciones(n, l) = 1}
                  }

                  \aux{cantApariciones}{e: \ent, s: \TLista{\ent}}{\ent}{
                        \displaystyle\sum_{i = 0}^{|s| - 1} \If{s[i] = e}{1}{0}
                  }
            \end{proc}
\end{enumerate}

\subsection{Ejercicio 14}
Dados dos enteros $a$ y $b$, se necesita calcular la suma y retornarla en un entero $c$. ¿Cuáles de las siguientes especificaciones son correctas para ese problema? Para las que no lo son, indicar por qué.

\begin{enumerate}[label=\alph*)]
      \item \begin{proc}{sumar}{\Inout a, b, c: \ent}{}
                  \requiere{True}
                  \asegura{a + b = c}
            \end{proc}

            Esta especificación está mal porque los valores de $a$ y $b$ pueden modificarse durante la ejecución, lo cuál produciria un resultado no deseado.

      \item \begin{proc}{sumar}{\In a, b: \ent, \Inout c: \ent}{}
                  \requiere{True}
                  \asegura{c = a + b}
            \end{proc}

            Esta especifiación es correcta.

      \item \begin{proc}{sumar}{\Inout a, b, c: \ent}{}
                  \requiere{a = A_0 \land b = B_0}
                  \asegura{a = A_0 \land b = B_0 \land c = a + b}
            \end{proc}

            Esta especifiación es correcta.
\end{enumerate}

\subsection{Ejercicio 15}
Dada una secuencia $l$, se desea sacar su primer elemento y devolverlo. Decidir cuáles de estas especificaciones son correctas. Para las que no lo son, indicar por qué y justificar con ejemplos.

\begin{enumerate}[label=\alph*)]
      \item \begin{proc}{tomarPrimero}{\Inout l: \TLista{\float}}{\float}
                  \requiere{|l| > 0}
                  \asegura{res = head(l)}
            \end{proc}

            Está mal porque el asegura referencia al estado final de $l$, por lo que $head(l)$ sería el segundo elemento de la lista. Ademas no pone condiciones sobre que hay que sacar el primer elemento de la lista.

      \item \begin{proc}{tomarPrimero}{\Inout l: \TLista{\float}}{\float}
                  \requiere{|l| > 0 \land l = L_0}
                  \asegura{res = head(L_0)}
            \end{proc}

            Está mal pues no especifica que se debe sacar el primer elemento de la lista.

      \item \begin{proc}{tomarPrimero}{\Inout l: \TLista{\float}}{\float}
                  \requiere{|l| > 0}
                  \asegura{res = head(L_0) \land |l| = |L_0| - 1}
            \end{proc}

            Está mal pues $L_0$ no está definida previamente en el requiere. Que la longitud sea 1 menor no necesariamente implica que se haya eliminado el primer elemento.

      \item \begin{proc}{tomarPrimero}{\Inout l: \TLista{\float}}{\float}
                  \requiere{|l| > 0 \land l = L_0}
                  \asegura{res = head(L_0) \land l = tail(L_0)}
            \end{proc}

            Esta especifiación es correcta.
\end{enumerate}

\subsection{Ejercicio 16}
Dada una secuencia de enteros, se requiere multiplicar por 2 aquéllos valores que se encuentran en posiciones pares. Indicar por qué son incorrectas las siguientes especificaciones y proponer una alternativa correcta.

\begin{enumerate}[label=\alph*)]
      \item \begin{proc}{duplicarPares}{\Inout l: \TLista{\ent}}{}
                  \requiere{l = L_0}
                  \asegura{|l| = |L_0| \land \\
                  \paraTodo{i}{\ent}{(0 \leq i < |l| \land i \mod 2 = 0) \thenLuego l[i] = 2 * L_0[i]}
                  }
            \end{proc}

            Está mál pues no especifica que los elementos en posiciones impares se mantienen igual

      \item \begin{proc}{duplicarPares}{\Inout l: \TLista{\ent}}{}
                  \requiere{l = L_0}
                  \asegura{
                  \paraTodo{i}{\ent}{(0 \leq i < |l| \land i \mod 2 \neq 0) \thenLuego l[i] = L_0[i]} \land \\
                  \paraTodo{i}{\ent}{(0 \leq i < |l| \land i \mod 2 = 0) \thenLuego l[i] = 2 * L_0[i]}
                  }
            \end{proc}

            Está mal pues no especifica que la secuencia tenga la misma cantidad de elementos.

      \item \begin{proc}{duplicarPares}{\Inout l: \TLista{\ent}}{\TLista{\ent}}
                  \requiere{True}
                  \asegura{
                        |l| = |res| \land \\
                        \paraTodo{i}{\ent}{(0 \leq i < |l| \land i \mod 2 \neq 0) \thenLuego res[i] = l[i]} \land \\
                        \paraTodo{i}{\ent}{(0 \leq i < |l| \land i \mod 2 = 0) \thenLuego res[i] = 2 * l[i]}
                  }
            \end{proc}

            Está mal pues no especifica que tiene que modificar la secuencia original.
\end{enumerate}

Solución propuesta:

\begin{proc}{duplicarPares}{\Inout l: \TLista{\ent}}{}
      \requiere{l = L_0}
      \asegura{|l| = |L_0| \yLuego \\
            \paraTodo{i}{\ent}{0 \leq i < |l| \thenLuego (i \mod 2 = 0 \land l[i] = 2 * L_0[i]) \lor (i \mod 2 \neq 0 \land l[i] = L_0[i])}}
\end{proc}

\subsection{Ejercicio 17}
Especificar los siguientes problemas de modificación de secuencias:

\begin{enumerate}[label=\alph*)]
      \item Dada una secuencia de enteros mayores a dos, reemplaza dichos valores por el número primo menor más cercano. Por ejemplo, si $l = \langle 6, 5, 9, 14 \rangle$, luego de aplicar $primosHermanos(l), l = \langle 5, 3, 7, 13 \rangle$

            \begin{proc}{primosHermanos}{\Inout l: \TLista{\ent}}{}
                  \requiere{\paraTodo{n}{\ent}{n \in l \thenLuego n > 2} \land l = L_0}
                  \asegura{|l| = |L_0| \yLuego \\
                  \paraTodo{i}{\ent}{0 \leq i < |l| \thenLuego esElPrimoMasCercano(l[i], L_0[i])}
                  }

                  \pred{esElPrimoMasCercano}{p, n: \ent}{
                        esPrimo(p) \land p < n \land \paraTodo{q}{\ent}{(esPrimo(q) \land q < n) \thenLuego q \leq p}
                  }
            \end{proc}

      \item Reemplaza todas las apariciones de $a$ en $l$ por $b$.

            \begin{proc}{reemplazar}{\Inout l: \seqchar, \In a,b: \cha}{}
                  \requiere{l = L_0}
                  \asegura{|l| = |L_0| \land \\
                  \paraTodo{i}{\ent}{0 \leq i < |l| \thenLuego l[i] = \If{L_0[i] = a}{b}{L_0[i]}}
                  }
            \end{proc}

      \item Elimina los elementos duplicados de $l$ dejando sólo su primera aparición (en el orden original). Devuelva además una secuencia con todas las apariciones eliminadas (en cualquier orden)

            \begin{proc}{limpiarDuplicados}{\Inout l: \seqchar}{\seqchar}
                  \requiere{l = L_0}
                  \asegura{} \hacer
            \end{proc}
\end{enumerate}

\subsection{Ejercicio 18}
Especificar los siguientes problemas. En todos los casos es recomendable ayudarse escribiendo predicados y funciones auxiliares.

\begin{enumerate}[label=\alph*)]
      \item Se desea especificar el problema $reemplazarNumerosPerfectos$, que dada una secuencia de enteros devuelve la secuencia pero con los valores que se corresponden con números perfectos reemplazados por el índice donde se encuentran. Se llama números perfectos a aquellos naturales mayores a cero que son iguales a la suma de sus divisores positivos propios (divisores incluyendo al 1 y sin incluir al propio número). Por ejemplo, $reemplazarNumerosPerfectos([0, 3, 9, 6, 4, 28, 7]) = [0, 3, 9, 3, 4, 5, 7]$, donde los únicos números reemplazados son el 6 y el 28 porque son los únicos números perfectos de la secuencia.

            \begin{proc}{reemplazarNumerosPerfectos}{\In s: \TLista{\ent}}{\TLista{\ent}}
                  \requiere{True}
                  \asegura{|res| = |s| \yLuego \paraTodo{i}{\ent}{0 \leq i < |s| \thenLuego res[i] = \IfThenElse{esPerfecto(s[i])}{i}{s[i]}}}

                  \pred{esPerfecto}{n: \ent}{
                        n > 0 \land n = sumaDivisores(n)
                  }

                  \aux{sumaDivisores}{n: \ent}{\ent}{
                        \sum_{i=1}^{n-1} \IfThenElse{n \mod i = 0}{i}{0}
                  }
            \end{proc}

      \item Se desea especificar el problema ordenarYBuscarMayor que dada una secuencia s de enteros (que puede tener repetidos) ordena dicha secuencia en orden creciente de valor absoluto y devuelve el valor del máximo elemento. Por ejemplo,

            \begin{itemize}
                  \item $ordenarYBuscarMayor([1,4,3,5,6,2,7]) = [1,2,3,4,5,6,7], 7$
                  \item $ordenarYBuscarMayor([1,-2,2,5,1,4,-2,-10]) = [1,1,-2,-2,2,4,5,-10], 5$
                  \item $ordenarYBuscarMayor([-10,-3,-7,-9]) = [-3,-7,-9,-10], -3$
            \end{itemize}

            \begin{proc}{ordenarYBuscarMayor}{\Inout s: \TLista{\ent}}{\ent}
                  \requiere{s = S_0}
                  \asegura{|s| = |S_0| \land mismosElementos(s, S_0) \land esElMayor(res, s) \land ordenadaPorAbs(s)}

                  \pred{mismosElementos}{s1: \TLista{\ent}, s2: \TLista{\ent}}{
                        \paraTodo{e}{\ent}{e \in s1 \Iff e \in s2}
                  }

                  \pred{esElMayor}{m: \ent, s: \TLista{\ent}}{
                        m \in s \land \paraTodo{e}{\ent}{e \in s \Then e \leq m}
                  }

                  \pred{ordenadaPorAbs}{s: \TLista{\ent}}{
                  \paraTodo{i}{\ent}{0 \leq i < |s| - 1 \thenLuego |\,s[i]\,| \leq |\,s[i+1]\,|}
                  }
            \end{proc}

      \item Se desea especificar el problema $primosEnCero$ que dada una secuencia $s$ de enteros devuelve la secuencia pero con los valores que se encuentran en posiciones correspondientes a un número primo reemplazados por 0. Por ejemplo,

            \begin{itemize}
                  \item $primosEnCero([0,1,2,3,4,5,6]) = [0,1,0,0,4,0,6]$
                  \item $primosEnCero([5,7,-2,13,-9,1]) = [5,7,0,0,-9,0]$
            \end{itemize}

            \begin{proc}{primosEnCero}{\In \TLista{\ent}}{\TLista{\ent}}
                  \requiere{True}
                  \asegura{|res| = |s| \yLuego \paraTodo{i}{\ent}{0 \leq i < |s| \thenLuego res[i] = \If{esPrimo[s[i]]}{0}{s[i]}}}

                  \pred{esPrimo}{n: \ent}{
                        \paraTodo{i}{\ent}{1 < i < |n| \thenLuego n \mod i \neq 0}
                  }
            \end{proc}

      \item Se desea especificar el problema positivosAumentados que dada una secuencia s de enteros devuelve la secuencia pero con los valores positivos reemplazados por su valor multiplicado por la posición en que se encuentra.

            \begin{itemize}
                  \item $positivosAumentados([0,1,2,3,4,5]) = [0,1,4,9,16,25]$
                  \item $positivosAumentados([-2,-1,5,3,0,-4,7]) = [-2,-1,10,9,0,-4,42]$
            \end{itemize}

            \begin{proc}{positivosAumentados}{\In s: \TLista{\ent}}{\TLista{\ent}}
                  \requiere{True}
                  \asegura{|res| = |s| \yLuego \paraTodo{i}{\ent}{0 \leq i < |s| \thenLuego res[i] = aumentaPositivos(s, i)}}

                  \aux{aumentaPositivos}{s: \TLista{\ent}, i: \ent}{\ent}{
                        \If{s[i] > 0}{s[i] * i}{s[i]}
                  }
            \end{proc}

      \item Se desea especificar el problema $procesarPrefijos$ que dada una secuencia $s$ de palabras y una palabra $p$, remueve todas las palabras de $s$ que no tengan como prefijo a $p$ y además retorna la longitud de la palabra más larga que tiene de prefijo a $p$. Por ejemplo, dados:

            $s = ["casa", "calamar", "banco", "recuperatorio", "aprobar", "cansado"]$ y $p = "ca"$ un posible valor para la secuencia s luego de aplicar $procesarPrefijos(s, p)$ puede ser $["casa", "calamar", "cansado"]$ y el valor devuelto será 7.

            \begin{proc}{procesarPrefijos}{\Inout s: \TLista{\str}, \In p: \str}{\ent}
                  \requiere{s = S_0 \land \existe{s1}{\str}{s1 \in s \land esPrefijo(p, s1)}}
                  \asegura{|s| \leq |S_0| \yLuego \paraTodo{s1}{\str}{s1 \in s \Iff (s1 \in S_0 \land esPrefijo(p, s1))}}
                  \asegura{\existe{i}{\ent}{0 \leq i < |s| \yLuego |s[i]| = res}}
                  \asegura{\paraTodo{i}{\ent}{0 \leq i < |s| \thenLuego |s[i]| \leq res}}

                  \pred{esPrefijo}{p: \str, s: \str}{
                        \paraTodo{i}{\ent}{0 \leq i < |p| \thenLuego s[i] = p[i]}
                  }
            \end{proc}
\end{enumerate}

\end{document}
